% !TEX root = multiplayer_reach_avoid_games.tex
\section{Introduction}
% Applications of differential games/reach-avoid games
% - CTF
% - warehouse robots
% - air traffic control
% - etc.

Multiplayer reach-avoid games are differential games between two adversarial teams of cooperative players playing on a compact domain with obstacles. The ``attacking team'' aims to send as many team members, called ``attackers'', to some target set as quickly as possible. The ``defending team'' seeks to delay or prevent the attacking team from doing so by attempting to capture the attackers. Such differential games have been studied extensively \cite{HThesis, Huang2011} and are also powerful theoretical tools for analyzing realistic situations in robotics, aircraft control, security, and other domains \cite{OFTHEAIRFORCEWASHINGTON:2009p37, Erzberger:2006p44, kiva2009}.

% Reach-avoid game difficulty 
% - unintuitive
% - exponential scaling
% - previous solution attempts not general

The multiplayer reach-avoid game is difficult to analyze because the two teams have conflicting and asymmetric goals while complex cooperation within each team may exist. In addition, optimal solutions are impossible to compute using traditional dynamic programming approaches due to the intrinsic high dimensionality of the joint state space. Previously, in \cite{Earl:2007p101}, where a team of defenders assumes that the attackers move towards their target in straight lines, a mixed-integer linear programming approach was used. \cite{Chasparis:2005p102} assumes that the attackers uses a linear feedback control law, and a mixed integer linear program was then relaxed into linear program. In complex pursuit-evasion games where players may change roles over time, a nonlinear model-predictive control \cite{Sprinkle:2004p100} approach has been investigated. Approximate dynamic programming \cite{McGrew:2008p103} has also been used to analyze reach-avoid games.

% Previous most-related work and shortcomings
% - HJI: optimal but intractible
% - open-loop: fast but conservative
Although the above techniques provide some useful insight, they only work well when strong assumptions are made or when accurate models of the opposing team can be obtained. To solve general reach-avoid games, the Hamilton-Jacobi-Isaacs (HJI) approach \cite{b:isaacs-1967} is ideal when the game is low-dimensional. The approach involves solving an HJI partial differential equation (PDE) in the joint state space of the players to compute a reach-avoid set, which partitions the joint state space of the players into a winning region for the defending team and one for the attacking team. The optimal strategies can then be extracted from the gradient of the solution. This approach is particularly useful because of the numerical tools \cite{j:mitchell-TAC-2005, Sethian1996, b:osher-fedkiw-2002} available, and has been able to solve a variety of problems \cite{Huang2011, j:mitchell-TAC-2005, DSST08}. The HJI approach can be applied to a large variety of player dynamics and does not explicitly assume any control strategy or prediction models for the players. However, the approach cannot be directly applied to our multiplayer reach-avoid game because its complexity scales exponentially with the number of players, making the approach only tractable for the two-player game. Thus, complexity-optimality trade-offs must be made.

%\cite{Zhou2012} presented an open-loop approach to approximate the solution to the two-player reach-avoid game, in which the time for the attacker to reach the target was conservatively estimated by assuming that the defender first chooses an open-loop control strategy, after which the attacker chooses a control strategy in response. The open-loop approach utilizes a modified version of the fast marching method (FMM) \cite{Sethian1996,Zhou2012}, and is extremely computationally efficient. Another advantage of this approach is that it is conservative towards the defender and provides guarantees on the defender's performance. However, the degree of conservatism can be very high because the defender is assumed to use an open-loop control strategy. The open-loop approach, therefore, can be considered an extreme trade off between the complexity of solving the HJI PDE and the optimality of the control strategy. 

% Current paper contributions
% - two-player to multiplayer
% - HJI solution: powerful for identical dynamics within each team
% - PD solution: efficient, more general dynamics within each team, better scaling
For the two-player reach-avoid game, we first present the two-player HJI solution \cite{Huang2011}, which computes a four-dimensional (4D) reach-avoid set that determines which player wins the game assuming both players use the closed-loop optimal control strategy. Next, we propose the ``path defense" approximation to the HJI solution, in which the defenders utilize a ``semi-open-loop" control strategy. Here, we approximate two-dimensional (2D) slices of the reach-avoid sets by solving 2D Eikonal equations, and provide guarantees for the defending team's performance.

For the multiplayer reach-avoid game, we propose to merge the $\NA\ND$ pairwise outcomes using the graph theoretic maximum matching, which can be efficiently computed by known algorithms \cite{Schrjiver2004, Karpinski1998}. The maximum matching process incorporates cooperation among defenders without introducing significant additional computation cost. When players on each team have identical dynamics, only a \textit{single} HJI PDE needs to be solved to characterize \textit{all} pairwise outcomes. Furthermore, when applying maximum matching to the two-player path defense solution, the computational complexity scales linearly with the number of players, as opposed to quadratically in the HJI approach.