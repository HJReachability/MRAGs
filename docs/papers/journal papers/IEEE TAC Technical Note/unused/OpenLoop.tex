%\documentclass[journal]{IEEEtran}
\documentclass[12pt,draftcls,onecolumn]{IEEEtran}
%\usepackage{hyperref}
% show labels
%\usepackage[notref,notcite]{showkeys}
%\usepackage[T1]{fontenc}
%\usepackage[latin9]{inputenc}
%\usepackage{amsthm}
\usepackage{amsmath}
\usepackage{amssymb}
\usepackage{graphics}
\usepackage{graphicx}
%\usepackage{enumitem}
\usepackage{color}
%\usepackage{algorithm2e}
%\usepackage{appendix}
%\usepackage{algorithmic}
%\usepackage{program}




%\makeatletter
%%%%%%%%%%%%%%%%%%%%%%%%%%%%%% Textclass specific LaTeX commands.
%\theoremstyle{plain}
%\newtheorem{thm}{Theorem}
%\theoremstyle{plain}
%\newtheorem{prop}[thm]{Proposition}
% \ifx\proof\undefined\
%   \newenvironment{proof}[1][\proofname]{\par
%     \normalfont\topsep6\p@\@plus6\p@\relax
%     \trivlist
%     \itemindent\parindent
%     \item[\hskip\labelsep
%           \scshape
%       #1]\ignorespaces
%   }{%
%     \endtrivlist\@endpefalse
%   }
%   \providecommand{\proofname}{Proof}
% \fi
%\theoremstyle{remark}
\newtheorem{rem}{Remark}
%\theoremstyle{definition}
\newtheorem{defn}{Definition}
\newtheorem{prop}{Proposition}
\newtheorem{thm}{Theorem}
\newtheorem{lem}{Lemma}

%\theoremstyle{plain}
%\newtheorem{lem}[thm]{Lemma}
%\theoremstyle{plain}
%\newtheorem{cor}[thm]{Corollary}
%
%\newtheorem{example}{Example}

%\makeatother

%\usepackage{babel}


\usepackage{my_macros}

\title{An Open-Loop Framework for Reach-Avoid Games}

\author{Zhengyuan Zhou, Haomiao Huang, Jerry Ding, Ryo Takei, and Claire~J.~Tomlin

\thanks{This work has been supported in part by NSF under CPS:ActionWebs
(CNS-931843), by ONR under the HUNT (N0014-08-0696) and
SMARTS (N00014-09-1-1051) MURIs and by grant N00014-12-1-0609, by AFOSR
under the CHASE MURI (FA9550-10-1-0567).}\vspace{-1cm}% <-this % stops a space
\thanks{Z.~Zhou, H.~Huang, J.~Ding, and C.~J.~Tomlin are with the Department of Electrical Engineering and Computer Sciences,
        University of California, Berkeley, CA, USA
        {\tt\small \{zhengyuan, haomiao, jding, tomlin\}@eecs.berkeley.edu}}        
%\thanks{R.~Takei is with ???,
%        ???
%        {\tt\small ??@??}}
}

\begin{document}

\maketitle
\thispagestyle{empty}
\pagestyle{empty}


\begin{abstract}
We consider a game in which one or more attacking players attempt to reach a target while avoiding both obstacles and capture by opposing, defending players. 
Unlike pursuit-evasion games, in this reach-avoid game the players must consider both opposing players and the target.
This complexity makes finding solutions to such games difficult and computationally challenging, especially as the number of players grows.
We propose an approach to solving such games in an open-loop sense, where the players commit to their control actions prior to the beginning of the game.
This reduces the dimensionality of the required computations, and allows us to quickly compute feasible solutions in real time for domains with arbitrary obstacle topologies. 
We describe two such formulations, each of which is conservative towards one side, and derive algorithms based upon modified fast-marching methods (FMM) for efficiently computing their solutions.
The formulations and algorithms are discussed in detail, along with simulation results demonstrating their use in scenarios with complex obstacle geometries and player speed profiles.
\end{abstract}

\section{Introduction}

In a reach-avoid game, one set of players attempt to arrive at a target set in the state-space, while avoiding a set of unsafe states, as well as interceptions by an opposing set of players.  Throughout this paper, we refer to the two opposing sides as attackers and defenders, respectively. 
%The defenders attempt to prevent the attackers from arriving at the target by intercepting them or blocking their motion.
Such games encompass a large number of robotics and control applications.
For example, many safe motion-planning problems may be formulated as reach-avoid games, in which the objective is to control one or more agents into a desired target region, while avoiding a set of obstacles or possibly adversarial agents.
%The game can be defined by a scalar value function representing the arrival time of one or more of the attackers at the target, with the attacking side attempting to minimize this value and the defenders maximizing.
Finding solution strategies for such games can be computationally expensive, even for games involving a single attacker and a single defender, and the computational burden can grow exponentially with the number of players. 
%For convenience, we refer to the two opposing sides of a reach-avoid game as attackers and defenders. 

This paper presents a method to mitigate the computational requirements for finding a solution to reach-avoid games, by addressing them in an open-loop sense, such that the two sides select their plans of action prior to the beginning of the game and do not change them over the course of play. 
In particular, we consider an \emph{upper-value} game, in which the attacking side first selects its control inputs, which are then made known to the defenders, and a \emph{lower-value} game, in which the defending side chooses first.
The upper-value game is conservative from the point of the attackers, and gives an upper bound on the time to accomplish their objectives.  The lower-value game, which is conservative from the point of the defenders, provides a lower bound on this time.
The scenario we consider involves kinematically controlled players moving with spatially varying speed limits in a bounded domain containing obstacles.

The primary contribution of this work is an open-loop framework for efficiently computing solutions to reach-avoid games in the two-player case and also certain multi-agent cases.  More specifically, we discuss formulations of the reach-avoid games in terms of the open-loop upper value and the open-loop lower value, and develop a set of modified fast-marching methods (FMM) that allow us to quickly compute solutions to these open-loop games, in the form of a set of player trajectories with provable properties.  We emphasize here that the two open-loop values, in addition to being interesting for study in their own right, also provide bounds on the closed-loop value of the reach-avoid game.  This closed-loop value is in general the solution to a Hamilton-Jacobi-Isaacs (HJI) equation, which can be very difficult to compute, in particular for games with a large number of players.  Thus, our open-loop framework can be interpreted as a computationally efficient approximation framework for reach-avoid games, through the trade-off of a certain degree of optimality for a reduction in computational complexity.

Some of the theoretical results were presented in two previous publications~\cite{OL_ICRA2012, OL_CDC2012}.
This work unifies the presentation of the open-loop games and their relationship to each other, while providing more detailed proofs of the technical results and extending the solutions to classes of multi-agent games. Additional simulation results are also included for the multi-agent cases. 

We begin by reviewing related work in this domain in Section~\ref{sec:related}.  We then proceed to describe formulations of the problem for a single attacker and a single defender in Section~\ref{sec:formulation}. The classical closed-loop, Hamilton-Jacobi-Isaacs (HJI) formulation is first introduced in this section. The computational complexity of this approach then provides the motivation for open-loop formulations of the game.
The theory of the upper and lower value games are presented in Section~\ref{sec:theory}.
For the upper-value game, we present an algorithm to compute a time-optimal path for the attacker to reach the target while avoiding obstacles and the defender in the low-dimensional state-space of a single player rather than in the joint state-space of both players. 
%This allows us to compute the solution to the motion planning problem from the solution to a Hamilton-Jacobi-Bellman (HJB) equation in the low-dimensional state-space of single attackers rather than in the joint state-space of all players.  
%From the HJB solution, we derive a ``safe-reachable set,'' namely the set of states reachable by the player from a given initial condition, under trajectories which conservatively avoid collisions with the point obstacle.  
For the lower-value game, we present an algorithm which computes a lower bound to the time-to-reach function and derive control inputs for a defending player that achieves this bound.
The properties of the two open-loop values as bounds on the closed-loop value is also discussed.
The algorithms for computing these solutions are discussed in detail in Section~\ref{sec:numerics}, based upon modifications of the FMM.
%The open-loop game formulations and solutions can also be directly applied to scenarios with multiple players on each side.
% as well as sequences of multiple target sets that must be reached in order.
Section~\ref{sec:multiPlayer} extends the discussion of open-loop games to scenarios with multiple players on each side.
%and illustrates the use of the algorithms through simulation results. 
Section~\ref{sec:simulation} presents simulation results demonstrating the application of the computational algorithms.  We conclude with a discussion of the open-loop framework and future work in Section~\ref{sec:conclusion}.
\section{Related Work}
\label{sec:related}
Refer to older papers... maybe combine with introduction
% !TEX root = multiplayer_reach_avoid_games.tex
\section{The Reach-Avoid Problem}
\subsection{The Multiplayer Reach-Avoid Game}
\label{sec:formulation}
Consider $\NA+\ND$ players partitioned into the set of $\NA$ attackers, $\pas = \{\pam{1}, \pam{2}, \ldots, \pam{\NA}\}$ and the set of $\ND$ defenders, $\pbs = \{\pbm{1}, \ldots, \pbm{\ND}\}$, whose states are confined in a bounded, open domain $\amb \subset \R^2$. The domain $\amb$ is partitioned into $\amb$ = $\free \cup \obs$, where $\free$ is a compact set representing the free space in which the players can move, while $\obs = \amb \setminus \free$ corresponds to obstacles in the domain. 

Let $\xam{i}, \xbm{j} \in \R^2$ denote the state of players $\pam{i}$ and $\pbm{j}$, respectively. Then given initial conditions $\xanm{i}\in \free,i=1,2,\ldots,\NA,\xbnm{i}\in \free,i=1,2,\ldots,\ND$, we assume the dynamics of the players to be defined by the following decoupled system for $t \geq 0$:

\bq\label{eq:dynamics}
\begin{aligned}
\dotxam{i}(t) &= \velai{i}\cam{i}(t), & \xam{i}(0) = \xanm{i}, i=1,2,\ldots,\NA \\
\dotxbm{i}(t) &= \velbi{i}\cbm{i}(t), & \xbm{i}(0) = \xbnm{i}, i=1,2,\ldots,\ND
\end{aligned}
\eq
where $\velai{i}, \velbi{i}$ denote maximum speeds for $\pam{i}$ and $\pbm{i}$ respectively, and $\cam{i},\cbm{i}$ denote controls of $\pam{i}$ and $\pbm{i}$ respectively. We assume that $\cam{i},\cbm{i}$ are drawn from the set $\A = \{\sigma \colon [0,\infty)\rightarrow \unitball \mid \sigma \text{ is measurable}\}$, where $\unitball$ denotes the closed unit disk in $\R^2$. We also constrain the players to remain within $\free$ for all time. Denote the joint state of all players by $\xj = (\xja, \xjb)$ where $\xja =(\xam{1},\ldots\xam{\NA})$ is the attacker joint state $\pas$, and $\xjb = (\xbm{1},\ldots,\xbm{\ND})$ is the defender joint state $\pbs$. 

The attacking team wins whenever $\m$ of the $\NA$ attackers reach some target set without being captured by the defenders; $\m$ is pre-specified with $0<M\le \NA$. The target set is denoted $\target\subset\free$ and is compact. The defending team wins if it can prevent the attacking team from winning by capturing or indefinitely delaying $\NA-\m+1$ attackers from reaching $\target$. An illustration of the game setup is shown in Fig. \ref{fig:mp_form}.

We will define the capture sets, $\avoid_{ij}$, to be $\avoid_{ij} = \left\{\xj\in\amb^{\NA+\ND} \mid \|\xam{i}-\xbm{j}\|_2\le\Rc \right\}$. $\pam{i}$ is captured by $\pbm{j}$ if $\pam{i}$'s position is within a distance $\Rc$ of $\pbm{j}$'s position. 

\begin{figure}
\centering
\includegraphics[width=0.35\textwidth]{"fig/formulation"}
\caption{The components of a multiplayer reach-avoid game.}
\label{fig:mp_form}
\end{figure}

In this paper, we address the following problems:
\begin{enumerate}
\item Given $\xjn$, $\target$, and some fixed integer $\m, 0<\m\le\NA$, can the attacking team win?
\item More generally, given $\xjn$ and $\target$, how many attackers can the defending team prevent from reaching the target?
\end{enumerate}

\subsection{The Two-Player Reach-Avoid Game}
\label{sec:2p_ra}
We will answer the above questions about the general $\NA$ vs. $\ND$ reach-avoid game by using the solution to the two-player $1$ vs. $1$ game as a building block. In the two-player game, we denote the attacker $\pa$, the defender $\pb$, their states $\xa,\xb$, and their initial conditions $\xan,\xbn$. Their dynamics are
\bq
\begin{aligned}
\dotxa(t) &= \vela\ca(t), & \xa(0) = \xan,\\
\dotxb(t) &= \velb\cb(t), & \xb(0) = \xbn
\end{aligned}
\eq

The players' joint state becomes $\xj=(\xa,\xb)$, and their joint initial condition becomes $\xjn=(\xan,\xbn)$. The capture set becomes simply $\avoid = \left\{(\xa,\xb)\in\amb^2 \mid \|\xa-\xb\|_2\leq \Rc\right\}$. 

$\pa$ wins if it reaches the target $\target$ without being captured by $\pb$. $\pb$ wins if it can prevent $\pa$ from winning by capturing $\pa$ or indefinitely delaying $\pa$ from reaching $\target$. For the two-player reach-avoid game, we seek to answer the following:
\begin{enumerate}
\item Given $\xjn$ and $\target$, is the defender guaranteed to win? \label{p:tp1}
\item More generally, given $\xa$ and $\target$, what is the set of initial positions such that the defender is guaranteed to win? \label{p:tp2}
\end{enumerate}
\input{theory_modified.tex}
\input{numerics.tex}
%\input{connection.tex}
\section{Generalization to Open-Loop Multi-Player Games}
\label{sec:multiPlayer}

In this section, we generalize the two-player open-loop game to multi-player games. 
As mentioned before, solving HJI equations in high dimensions is in general an intractable problem, and
the state space typically scales exponentially with the number of players. 
However, for certain game formulations with multiple players, the open-loop framework and the modified FMM methods provide a fast way to 
compute the open-loop values and controls that scales linearly with the number of players. 
Due to the inherent difference between the open-loop upper value computation and the open-loop lower value computation, we will provide two open-loop multi-player game formulations corresponding
to the upper and lower value, each leveraging techniques developed previously. 

%We first state the generalised open-loop games.
\subsection{The Open-Loop Upper Value for Multi-Player Games}
Suppose that there are $N$ attackers $P_A^1$,...,$P_A^N$ with 
initial conditions $x_{A1}^0$,...,$x_{AN}^0$, and $M$
defenders $P_D^1$,...,$P_D^M$, with 
initial conditions $x_{D1}^0$,...,$x_{DM}^0$, each having its own 
decoupled dynamics.
We assume that each defender has its own capture set, and use $\avoid \subset  \mathbb{R}^{N+M}$ to encode capture conditions in which at least one defender captures at least one attacker. 
For convenience, we also define the partial avoid set $\avoida^{ij}(x) (1 \le i\le N, 1 \le j \le M)$ to be the set of
all states that $P_A^i$ needs to avoid if $P_D^j$ stays at state $x$.
We use $x_{Ai}$ and $\ca^{i}$ to denote the state and the control of $P_A^i$, respectively.
$x_{Dj}$ and $\cb^{j}$ are similarly defined for defenders.
 $\xjn = (x_{A1}^0,...,x_{AN}^0;x_{D1}^0,...,x_{DM}^0)$ is the joint initial state and 
$\xj$ is the joint state of all agents.

In the problem that we consider, the goal for the attackers is to minimize the time needed for every (and hence the last) attacker to reach the target set $\target$, while the defenders are trying to maximize this time. We note that as soon as an attacker reaches the target, it is removed from the game and hence safe thereafter.
Observe that under this formulation, if the defenders can prevent at least one attacker from reaching the target, then the total arrival time for the attackers is infinity.
In this subsection, we adopt the information pattern that is conservative towards attackers. 
Namely, the controls of every attacker will be revealed to every defender. 
Then all defenders will jointly optimize their controls with this knowledge.

Under this formulation, the payoff of the game is given by
\begin{align}
\pay_{multi}(\xjn; \ca^1 ,..., \ca^N ; \cb^1 ,..., \cb^M) =& \inf\left\{ t\ge 0 \mid \forall 1\le i \le N, \exists t_i \leq t, \x_{Ai}(t_i)\in\target, \right. \nonumber\\
&\left. \xj(s)\not\in\avoid, \forall s\in [0,t] \right\}.
\end{align}
The multi-player open-loop upper value is then
\bq
\begin{aligned}
\vUpper_{multi}(\xjn) = 
%\inf_{\ca^{i}\in\A, 1\le i \le N}  \sup_{\cb^{j}\in\A,1\le j \le M}  \pay_{multi}(\xjn;\{{\ca^{i}}\}_{1\le i \le N};\{{\cb^{j}}\}_{1\le j \le M})
\inf_{\ca^{i}\in\Ua}  \sup_{\cb^{j}\in\Ub}  \pay_{multi}(\xjn;\{{\ca^{i}}\}_{1\le i \le N};\{{\cb^{j}}\}_{1\le j \le M}).
\end{aligned}
\eq

In general, what makes a multi-agent problem difficult is that  simply adopting the optimal control for each agent does not necessarily yield a global optimal solution for the whole system. 
%That is, the global optimality is usually achieved at the expense of some agents' sacrifice. 
However, we shall see here that in this case the global optimal solution in fact decentralizes into each attacker's local optimal control, thus eliminating the need for centralized coordination. 
%We can then leverage this property to solve for each attacker's control individually against all the defenders.

First, observe that we can define for each attacker a function $\valsRi(x;x_{Ai}^0)$ similarly as in Definition~\ref{WRupper} and for each defender a function $\vald_{j}(y;x_{Dj}^0)$ similarly as in Definition~\ref{t*upper}.
Second, given that all defenders cooperate to delay the attackers from reaching the target, we can assign a function $\vald(y;\{x_{Dj}^0\}_{1 \le j \le M}) := \min_{1 \le j \le M}\vald_{j}(y;x_{Dj}^0)$ to the whole defending side.
For each attacker $P^i_A$, we define the corresponding safe reachable set $\safereach_{i}$ to be the maximal set such that
the following is satisfied:
%\bq\label{safereachableSet_multi}
$\valsSi(y;\x_{Ai}^0) < \vald(y;\{x_{Dj}^0\}_{1 \le j \le M}), \ \forall y \in \safereach_i.$
%\eq

The following result generalizes Theorem~\ref{mainThm} to the multi-player case.
\begin{thm}\label{multiUpper}
\mbox{}
\begin{enumerate}
\item\label{first}
$\vUpper_{multi}(\xjn)$ $<$ $\infty$ if and only if
 $\safereach_{i} \cap \target \neq \emptyset$, $\forall i = 1,2,...,N$;
\item
$\vUpper_{multi}(\xjn) = \max_{1 \le i \le N}  \inf \{\valsSi(y;\x_{Ai}^0) \mid {y\in\target}\}$.
\end{enumerate}

\end{thm}

\begin{proof}
The first part can be inferred in a straightforward manner from the formulation of the game.
For the second part, first observe that if $\vUpper_{multi}(\xjn) = \infty$, then by part~\ref{first}) of the theorem, there exists $i \in \{1,2,...,N\}$ such that $\inf \{\valsSi(y;\x_{Ai}^0) \mid {y\in\target}\} = \infty$.
Thus, $\vUpper_{multi}(\xjn) = \max_{1 \le i \le N}  \inf \{\vals_{i}(y;\x_{Ai}^0) \mid {y\in\target}\} =\infty$.

Now consider the case in which $\vUpper_{multi}(\xjn)$ is finite.
Also by part~\ref{first}), we know that $\safereach_{i} \cap \target \neq \emptyset$, $\forall i = 1,2,...,N$.
This implies that for every $P^i_A$, $t_i^* := \inf \{\vals_{i}(y;\x_{Ai}^0) \mid {y\in\target}\} < \infty$.  Let $t^* := \max_{i} t_i^*$.  Then clearly, $\vUpper_{multi}(\xjn) \le t^*$.  Moreover, there exists a collection of controls $\{{\ca^{i}}\}_{1\le i \le N}$ such that $\sup_{\cb^{j}\in\Ub}  \pay_{multi}(\xjn;\{{\ca^{i}}\}_{1\le i \le N};\{{\cb^{j}}\}_{1\le j \le M}) < \infty$.  For any such controls, there exists $t \ge 0$ such that for every $i=1,2,...,N$, there exists $\tau_i \le t$ such that $\x_{Ai}(\tau_i)\in\target$ and 
$\xj(s)\not\in\avoid, \forall s\in [0,t]$.  Let $\tau_i^* := \inf \{t \ge 0 \mid \x_{Ai}(t) \in \target\}$, then it can be checked that $t_i^* \leq \tau_i^*$, $\forall i = 1,2,..,N$ and $\pay_{multi}(\xjn;\{{\ca^{i}}\}_{1\le i \le N};\{{\cb^{j}}\}_{1\le j \le M}) = \max_{i} \tau_i^*$, $\forall \cb^{j}\in\Ub$.  Given that this holds for every $\{{\ca^{i}}\}_{1\le i \le N}$ such that the payoff is finite, we have $t^* \leq \vUpper_{multi}(\xjn)$.  The second part of the theorem then follows.
%Without loss of generality, assume that $\inf \{\vals_{1}(y;\x_{A1}^0) \mid {y\in\target}\}$ achieves the maximum over $i \in \{1,2,...,N\}$.  If the other players on the same team, i.e. $P^i_A$, $i \neq 1$ selects their optimal open-loop controls, then these players will reach the target in no more than $\inf \{\valsSi(y;\x_{Ai}^0) \mid {y\in\target}\}$ time units, with $i \neq 1$.
%Then the opposing side, consisting of players $P^j_D$, $j = 1,2,...,M$ can optimize their controls with respect to the chosen control of $P^1_A$.  The game then becomes equivalent to a one attacker versus multiple defenders game. 
%Thus, we have $\vUpper_{multi}(\xjn) =  \inf \{\vals_{1}(y;\x_{A1}^0) \mid {y\in\target}\}
%= \max_{1 \le i \le N}  \inf \{\vals_{i}(y;\x_{Ai}^0) \mid {y\in\target}\}$.
%No other attacker can choose a control that is suboptimal in an attempt to reduce the time it takes for $P^1_A$ to reach the target under the worst case, as the defenders can be oblivious to all other attackers and still achieve the same final arrival time. 
%Therefore, equality still holds.
\end{proof}

\begin{rem}
We note that under this formulation, it is not necessary for every attacker to use an optimal control to ensure that the upper value is achieved.  In particular, suppose that it takes attacker $P_1^A$ the longest to reach the target set, namely $t_1^* \in \arg \max_{i} t_i$ in the proof above.  Then attackers other than $P_1^A$ can use suboptimal controls while still achieving the same overall payoff, as long as the suboptimal controls do not yield a time-to-reach longer than $t_1^*$.
We note that under this formulation, attackers other than $P_1^A$ can still take suboptimal controls while still
achieving the same overall payoff as long as the suboptimal controls do not yield time longer than 
$\inf \{\vals_{i}(y;\x_{Ai}^0) \mid {y\in\target}\}$.
%A slightly stronger formulation can have the attackers minimizing the sum of all time of the attackers, where the payoff function now
%becomes:
%\bq
%\begin{aligned}
%\pay_{multi}(\xjn; \{\ca^i\}_{1\le i \le N} ;\{\cb^j\}_{1\le j \le M}) =  \sum_{i=1}^{N} \inf\{ t\ge 0 \mid \x_{Ai}(t)\in\target,  &\x_{Ai}(s)\not\in\avoida^{ij}(x_{Dj}(s)), \\
%& \forall 1 \le j \le M, \forall s\in [0,t] \}.
%\end{aligned}
%\eq
%From the proof of the above theorem, we see that the result still holds. Yet under this formulation,
%each attacker must strictly follow the open-loop optimal control since any
%deviation will unnecessarily incur suboptimality.
\end{rem}

\subsection{The Open-Loop Lower Value for Multi-Player Games}
Under the open-loop lower value information pattern, an unfortunate difference from the open-loop upper value is that the 
optimal controls for the defending agents do not decentralize, and the defenders must coordinate their control inputs jointly. 
%This is caused by the fact that as soon as one defending agent chooses a control and commit to it, the other defenders need to take this information into account before choosing their controls. 
This distinction is again created because unlike the attacking agents, whose common goal is to enter a stationary target set, the defenders' goals depend upon where the attackers are at each time instant as well as where other defending agents are. 
This restricts the type of generalization which can be realized.
 
Adopting the notations introduced previously, we will consider a scenario with $N$ attacking agents and a single
defending agent. The attacking agents and the defending agent both have the same objectives as stated in the previous subsection.
The generalized open-loop lower value is then given by
$\vLower_{multi}(\xjn) = \sup_{\cb^1 \in \Ub}\inf_{\ca^{i}\in\Ua, 1\le i \le N} \pay_{multi}(\xjn;\{{\ca^{i}}\}_{1\le i \le N};\cb^1)$.

Each attacking agent $P_i^A$ has its own $\ts^{i}(\x;x_{Ai}^0)$ function, as introduced in Definition~\ref{lowerTStar} and the single defender has the function
 $\wRset^{i}(\x;x_{D1}^0)$, as introduced in Definition~\ref{lowerWR}. Therefore, for each attacking agent $i$ and the defending agent, we can associate a corresponding set
$\Rss_{i}$ similarly defined as in equation~\eqref{RStarStar}, and a function $\vLLower^{i}(x_{Ai}^0,x_{D1}^0)$ similarly defined as in equation~\eqref{lower_bound_maximal}, irrespective of 
all the other attacking agents. 
The following generalization can be then shown using a similar line of reasoning as presented in the proof to Theroem~\ref{bound}.
\begin{thm}
$\vLower_{multi}(\xjn) \ge \max_{1\le i \le N}\{\vLLower^{i}(x_{Ai}^0,x_{D1}^0)\}$
\end{thm}
%\begin{proof}
%The proof follows from the proof for Theorem~\ref{multiUpper}.
%\end{proof}




\section{Numerical Results}
\label{sec:simulation}
In the following subsections, we use a 4 vs. 4 example to illustrate our methods. The game is played on a square domain with obstacles. Defenders have a capture radius of $0.1$ units, and all players have the same maximum speed. We first show simulation results for the HJI and path defense approaches for determining pairwise outcomes, and then compare the two approaches. Computations were done on a Lenovo T420s laptop with a Core i7-2640M processor with 4 gigabytes of memory.

\subsection{HJI Formulation}
Figures \ref{fig:OL_pw_HJIA} to \ref{fig:OL_mm_HJI} show the results of solving the 4D HJI PDE in Equation (\ref{eq:HJ_PDE_reachavoid}) using the terminal set and avoid set in Equations (\ref{eq:4DHJI_terminal_set}) and (\ref{eq:4DHJI_avoid_set}), respectively. Computing the 4D reach-avoid set on a grid with 45 grid points in each dimension took approximately 30 minutes. In general, this makes the computation time required for computing $\N^2$ pair-wise outcomes $C \N^2 = 30 \N^2$ minutes. In our example, however, all players have the same maximum speed, so only a single 4D HJI PDE needed to be solved.

The solution of the 4D HJI PDE characterizes the 4D reach-avoid set $\mathcal{RA}_\infty(R,A)$. To visualize this set in 2D, we take 2D slices of the 4D reach-avoid set sliced at each player's position. Figure \ref{fig:OL_pw_HJIA} shows the 4D reach-avoid set sliced at various \textit{attacker positions}, which gives the all pairwise outcomes. For example, in the left top subplot, the attacker at $(-0.2, 0)$ is guaranteed to win a 1 vs. 1 reach-avoid game against each of the defenders at $(-0.3, 0.5)$ and at $(-0.3, -0.5)$, and guaranteed to lose against each of the defenders at $(0.3, 0.5)$ and at $(0.3, -0.5)$ if they play optimally.

Figure \ref{fig:OL_pw_HJID} shows the 4D reach-avoid set sliced at various \textit{defender positions}, and characterizes the same pairwise outcomes. For example,  in the right top subplot, the defender at $(-0.3, 0.5)$ is guaranteed to win a 1 vs. 1 reach-avoid game against each of the attackers at $(-0.5, 0)$ and at $(-0.2, 0.9)$, and guaranteed to lose against each of the attackers at $(0.7, -0.9)$ and at $(-0.2, 0)$ if they play optimally.

Figure \ref{fig:OL_mm_HJI} shows the bipartite graph and maximum matching resulting from the pairwise outcomes. In this case, the maximum matching is of size 4, a perfect matching. This guarantees that if each defender plays against the attacker matched by the maximum matching using the strategy in Equation (\ref{eq:opt_ctrl2_d}), then \textit{no} attacker will be able to reach the target.

\begin{figure}
	\centering
	\includegraphics[width=0.45\textwidth]{"fig/OLGame_pw_results_HJI_fixA"}
	\caption{The 4D reach-avoid set sliced at various attacker positions. In each subplot, defenders ``inside" the reach-avoid slice boundary win against the plotted attacker in a 1 vs. 1 setting using the control in Equation (\ref{eq:opt_ctrl2_d}); defenders ``outside" lose.}
	\label{fig:OL_pw_HJIA}
\end{figure}

\begin{figure}
	\centering
	\includegraphics[width=0.45\textwidth]{"fig/OLGame_pw_results_HJI_fixD"}
	\caption{The 4D reach-avoid set sliced at various defender positions. In each subplot, attackers ``inside" the reach-avoid slice boundary win against the plotted defender in a 1 vs. 1 setting using the control in Equation (\ref{eq:opt_ctrl2_a}); attackers ``outside" lose.}
	\label{fig:OL_pw_HJID}
\end{figure}

\begin{figure}
	\centering
	\includegraphics[width=0.4\textwidth]{"fig/OLGame_mm_results_HJI"}
	\caption{Tying together pair-wise interactions through maximum matching. Here, a maximum matching of size 4 (a perfect matching) guarantees that \textit{no} attacker will be able to reach the target without being captured if the defenders play optimally according to Equation (\ref{eq:opt_ctrl2_d}) against their matched attackers.}
	\label{fig:OL_mm_HJI}
\end{figure}

\subsection{Path Defense Formulation}
Figures \ref{fig:pd_ex} to \ref{fig:pd_mm} show the results of using the path defense approach to compute conservative approximations of the 4D reach-avoid set sliced at various attacker positions.

Figure \ref{fig:pd_ex} shows the defender winning region for a given attacker position (red cross) and path of defense (blue line). If the defender can get to $\pstar$ (small light green square) before the attacker can get to $\rpa \cup \rpb$, then the defender will be able to strongly defend the path, and prevent the attacker from reaching any target set that is enclosed by the path (large dark green square). The region within which the defender is able to do this is the region $\dr(\apa,\apb)$ (shown in light green). 

As an example, refer to the bottom left subplot in Figure \ref{fig:pd_ex}. Using the semi-open-loop strategy described in Section \ref{subsec:reach_avoid}, the defenders at $(-0.3, 0.5)$ and at $(0.3, 0.5)$ are able to prevent the attacker at $(-0.2, 0.9)$ from reaching the target.

Figure \ref{fig:pd_pw} shows the bipartite graph and maximum matching resulting from the pairwise outcomes. In this case, the maximum matching is of size 3. This guarantees that if each defender plays against the attacker matched by the maximum matching using the semi-open-loop strategy, then \textit{at most} 1 attacker will be able to reach the target.

Computations were done on a $200\times200$ grid, and 937 paths were used to compute the results in Figure \ref{fig:pd_pw}. Computation time varies with the number of paths we chose in steps \ref{step:createPath} and \ref{step:repeatCreatePath} in the algorithm in Section \ref{subsec:reach_avoid}. Taking the union of the defender winning regions from more paths will give a less conservative result, but require more computation time. A summary of the performance of our algorithm is shown in Figure \ref{fig:pd_perf}. 

With 937 paths, the computation of paths took approximately 60 seconds, and the computation of the 2D slice given the set of paths took approximately 30 seconds. However, very few paths are needed to approximate a 2D slice: Even with as few as 30 paths, the computed 2D slice covers more than 95\% of the area of the 2D slice computed using 937 paths. This reduces the computation time of the paths to 2.5 seconds, and the computation time of the 2D slices given the paths to 2.1 seconds. In terms of the complexity constants in Section \ref{subsec:reach_avoid}, we have that the computation time required for computing all pairwise interactions is $C_1 + C_2 \N = 2.5 + 2.1 \N$ seconds.

\begin{figure}
	\centering
	\includegraphics[width=0.35\textwidth]{"fig/PD Example"}
	\caption{Defense of a single path that encloses the target set.}
	\label{fig:pd_ex}
\end{figure}

\begin{figure}
	\centering
	\includegraphics[width=0.45\textwidth]{"fig/OLGame_pw_results_PD"}
	\caption{The reach-avoid set sliced at various attacker positions. In each subplot, defenders ``inside" the reach-avoid slice boundary are guaranteed to win against the plotted attacker in a 1 vs. 1 setting using the semi-open-loop control strategy described in Section \ref{subsec:reach_avoid}.}
	\label{fig:pd_pw}
\end{figure}

\begin{figure}
	\centering
	\includegraphics[width=0.4\textwidth]{"fig/OLGame_mm_results_PD"}
	\caption{Tying together pairwise interactions through maximum matching. Here, a maximum matching of size 3 guarantees that \textit{at most} 1 attacker will be able to reach the target without being captured if the defenders use the semi-open-loop control strategy described in Section \ref{subsec:reach_avoid} against their matched attackers.}
	\label{fig:pd_mm}
\end{figure}

\begin{figure}
	\centering
	\includegraphics[width=0.4\textwidth]{"fig/alg_perf_Rc"}
	\caption{Performance of the Path Defense Solution.}
	\label{fig:pd_perf}
\end{figure}

\subsection{Comparison Between HJI and Path Defense Formulations}
Each pairwise outcome computed with the HJI approach gives the optimal behavior assuming the players utilize a closed-loop strategy. In contrast, each pairwise outcome computed with the path defense approach assumes that the defender is using a semi-open-loop strategy as discussed in Section \ref{sec:comparison}. Figures \ref{fig:comp_ol} and \ref{fig:comp_ml} compare the 2D slices computed from the two different approaches.

Figure \ref{fig:comp_ol} shows the results for the 4 vs. 4 example in the preceding sections, where all players have the same maximum speed. Because the semi-open-loop strategy in the path defense approach is conservative towards the defender, the computed defender winning region is smaller in the path defense approach. The degree of conservatism depends on the domain, obstacles, target set, and attacker position. In this example, the defender winning regions computed using path defense approach on average covers approximately 75\% of the area of the defender winning regions computed using the HJI approach.

The path defense approach becomes more conservative when the attacker is slower than the defender because in this case, the attacker being outside the region $\rpa \cup \rpb$ is not a necessary but only a sufficient condition for the defender to win using the semi-open-loop strategy, as discussed in the Lemmas in Section \ref{sec:path_defense}. The path defense approach also becomes more conservative when there are large obstacles in the domain because we only considered paths that touch the target, which introduces additional conservatism in a non-simply-connected free space, as discussed in \ref{subsec:reach_avoid}. Figure \ref{fig:comp_ml} compares the 2D slices computed by the path defense approach and by the HJI approach in another 4 vs. 4 game where the attackers' maximum speed is 80\% of the defenders' maximum speed (players on each team have the same maximum speed), and where there is a larger obstacle in the domain. In this case,  the defender winning regions computed using path defense approach covers approximately 34\% of the area of the defender winning regions computed using the HJI approach.

\begin{figure}
	\centering
	\includegraphics[width=0.45\textwidth]{"fig/OLGame_compare"}
	\caption{Reach-avoid slices computed using the HJI approach and the path defense approach. Defenders and attackers have the same maximum speed.}
	\label{fig:comp_ol}
\end{figure}

\begin{figure}
	\centering
	\includegraphics[width=0.45\textwidth]{"fig/midLGame_compare"}
	\caption{Reach-avoid slices computed using the HJI approach and the path defense approach. Attackers' maximum speed is 80\% of that of the defenders.}
	\label{fig:comp_ml}
\end{figure}

\subsection{Real-Time Maximum Matching Updates}
After determining all pairwise outcomes either by $\N^2$ HJI PDEs in general or by solving a single 4D HJI PDE when all players on each team have the same maximum speed, pairwise outcomes of \textit{any} joint state of the attacker-defender pair are characterized. Thus, the bipartite graph corresponding to the pairwise outcomes can be updated simply by checking whether the joint state of the attacker-defender pair is inside the appropriate 4D reach-avoid set. This allows for updates of the bipartite graph and its maximum matching as the players play out the game in real time.

Figure \ref{fig:real_time_update} shows the maximum matching at several time snapshots of a 4 vs. 4 game. Each defender that is part of a maximum matching plays optimally against the paired-up attacker according to Equation (\ref{eq:opt_ctrl2_d}), and the remaining defender plays optimally against the closest attacker also according to Equation (\ref{eq:opt_ctrl2_d}). The attackers' strategy is to move towards the target along the shortest path while steering clear of the obstacles by $0.125$ units. The maximum matching is updated every $\Delta=0.005$ seconds. At $t=0$ and $t=0.2$, the maximum matching is of size 3, which guarantees that at most one attacker will be able to reach the target. After $t=0.4$, a perfect matching is found, which guarantees that no attacker will be able to reach the target.

\begin{figure}
	\centering
	\includegraphics[width=0.45\textwidth]{"fig/time varying graph"}
	\caption{An illustration of how the size of maximum matching can increase over time. Throughout the game, the defenders are updating the bipartite graph and maximum matching via the procedure described in Section \ref{subsec:tvarp}. Because the attacking team is not playing optimally, the defending team is able to find a perfect matching after $t=0.4$ (bottom plots) and prevent \textit{all} attackers from reaching the target.}
	\label{fig:real_time_update}
\end{figure}

\section{Conclusions and Future Work} \label{sec:conc}
By applying maximum matching to the solution to the two player reach-avoid game, we were able to approximate the solution to the multiplayer reach-avoid game without significant additional computation overhead over the two player computation. By solving a single 4D HJI PDE, we obtained all possible pairings between the defenders and attackers. Then, a maximum matching algorithm determines the pairing that prevents the maximum number of attackers from reaching the target. This way, we were able to analyze the multiplayer reach-avoid game without directly solving the corresponding high dimensional, numerically intractable HJI PDE. Calculating time-varying defender-attacker pairings allows the defending team to potentially increase the size of the maximum matching over time.

An immediate extension of this work is to investigate defender strategies that optimally promote an increase in the size of maximum matching. This would be a step towards real-time defending team cooperation, in which each defender is not only responsible for preventing a particular assigned attacker from reaching the target, but also enabling other defenders to defend previously seemingly undefendable attackers. Many other extensions of this work naturally arise, including analyzing reach-avoid games with unfair teams with different numbers of players on each team, more complex player dynamics, or partial observability. 

% bibliography
\bibliographystyle{IEEEtran}
\bibliography{refopenloop}

\end{document}
