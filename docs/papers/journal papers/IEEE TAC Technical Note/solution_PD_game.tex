% !TEX root = multiplayer_reach_avoid_games.tex
\section{The Path Defense Solution to the 1 vs. 1 Game}
\label{sec:path_defense}
We approximate 2D slices of the 4D reach-avoid set (or simply ``2D slices") in the path defense approach. Each slice will be taken at an attacker position. Here, we will assume that the defender is not slower than the attacker: $\vela \leq \velb$. 

\subsection{The Path Defense Game}
\label{subsec:pd_game}
The \textit{Path Defense Game} is a two-player reach-avoid game in which the boundary of the target set is the shortest path between two points on $\boundary$, and the target set is on one side of that shortest path (Fig. \ref{fig:pd_form}). We denote the target set as $\target=\sac$ for two given points on the boundary $\apa,\apb$. $\sa$ and $Anal.d$ are defined below. 

% Definitions
\begin{defn} % Shortest path between two points
\textbf{Path of defense}. A path of defense, $Anal.d$, is the shortest path between two boundary points $\apa,\apb\in\boundary$. $\apa$ and $\apb$ are called the \textbf{anchor points} of path $\pathd$. 
\end{defn}

Denote the shortest path between \textit{any} two points $\x,\y\in\free$ to be $\spath(\x,\y)$, with length $\dist(\x,\y)$, and requiring the attacker and defender durations of $\ta(\x, \y),\tb(\x,\y)$ to traverse, respectively. We will also use $\dist(\cdot,\cdot)$ with one or both arguments being sets in $\amb$ to denote the shortest distance between the arguments.

\begin{defn} % Attacker's side of the path (needed?)
\textbf{Attacker's side of the path}. A path of defense $\pathd$ partitions the domain $\amb$ into two regions. Define $\sa$ to be the region that contains the attacker, not including points on the path $\pathd$. The attacker seeks to reach the target set $\target=\sac$.
\end{defn}

\begin{figure}
\centering
\includegraphics[width=0.35\textwidth]{"fig/path_defense_game"}
\caption{The components of a path defense game.}
\label{fig:pd_form}
\end{figure}

\subsection{Solving The Path Defense Game}
A path defense game can be directly solved by computing a 4D reach-avoid set. Since the direct solution is time- and memory- intensive, we propose an efficient approximation of 2D slices that is conservative towards the defender.

% Definitions
\begin{defn} % Defendable path
\textbf{Defendable path}. Given $\xjn=(\xan,\xbn)$, a path $\pathd$ is defendable if regardless of the attacker's actions, the defender has a control function $\cb(\cdot)$ to prevent the attacker from reaching $\pathd$ without being captured.
\end{defn}

\begin{defn} % Strongly defendable path
\textbf{Strongly defendable path}. $\pathd$ is \textit{strongly} defendable if regardless of the attacker's actions, the defender has a control function $\cb(\cdot)$ to reach $\pathd$ after finite time and prevent the attacker from reaching $\pathd$.
\end{defn}

Checking whether a path $\pathd$ is defendable involves a 4D reach-avoid set calculation, so instead we check whether a path $\pathd$ is strongly defendable. The following definitions lead to our first Lemma which describes how to determine strong defendability using 2D distance calculations; the definitions and Lemma are illustrated in Fig. \ref{fig:lemma1}.

\begin{defn} % Level set image of attacker
\textbf{Attacker level set image}. Given attacker position $\xa(t)$, define the attacker level set image with respect to anchor point $\apa$ to be $\xai(t;\apa) = \{\x\in\pathd: \ta(\x,\apa) = \ta(\xa(t),\apa)\}$. $\xai$ is the unique point on $\pathd$ such that $\ta(\xai, \apa)=\ta(\xa,\apa)$. Define $\xai(t;\apb)$ similarly by replacing $\apa$ with $\apb$. For convenience, we sometimes omit the time argument and write $\xai(\apa)$.
\end{defn}

\begin{prop}
\label{rem:image_of_a}
$\dist(\xai(\apb),\apa) \le \dist(\xai(\apa),\apa)$. 
\end{prop}

\begin{IEEEproof}
First note that
\begin{equation*}
\begin{aligned}
\dist(\apa,\apb) &\le \dist(\xa,\apa) + \dist(\xa,\apb)\\
%& \text{(Definition of shortest path between $\apa$ and $\apb$)} \\
& = \dist(\xai(\apa),\apa) + \dist(\xai(\apb),\apb)
\end{aligned}
\end{equation*}

Then, since the left hand side is given by $\dist(\apa,\apb) = \dist(\apa,\xai(\apb))+\dist(\xai(\apb),\apb)$, the result follows.
\end{IEEEproof}


\begin{defn} % Defender capture set
\textbf{Capture set}: Define the capture set to be $\Dc(\y,t)=\{\x \mid \|\x-\y(t)\|_2\leq \Rc\}$. We will drop the second argument of $\Dc$ when $\y$ does not depend on time.
\end{defn}

\begin{rem}
Given $\pathd$, suppose the attacker level set image is within defender's capture set at some time $s$: 
\begin{equation*}
\begin{aligned}
&\xai(s;\apa) \in \Dc(\xb, s) \quad \text{(or } & \xai(s;\apb) \in \Dc(\xb, s) \text{)}
\end{aligned}
\end{equation*}
Then, there exists a control for the defender to keep the attacker level set image within the capture radius of the defender thereafter: 

\begin{equation*}
\begin{aligned}
&\xai(t;\apa) \in \Dc(\xb, t) ~ \forall t\geq s \\
\text{(or } & \xai(t;\apb) \in \Dc(\xb, t) ~ \forall t\geq s\text{)}
\end{aligned}
\end{equation*}

This is because the attacker level set image can move at most as fast as the attacker, who is not faster than the defender.
\end{rem}

\begin{defn} % Winning regions for defender given path and point
\label{def:d_win_region}
\textbf{Regions induced by point $\ppath$ on path}. Given a point $\ppath\in\pathd$, define a region $\rpa\left(\ppath\right)$ associated the point $\ppath$ and anchor point $\apa$ as follows:
\bq
\rpa\left(\ppath\right) = \left\{\x: \dist(\x,\apa) \leq \dist(\Dc(\ppath),\apa) \right\}
\eq

Define $\rpb(\ppath)$ similarly by replacing $\apa$ with $\apb$.
\end{defn}

% Lemma involving winning regions for defender
\begin{lem}
\label{lem:d_winning_region}
Suppose $\xbn = \ppath \in \pathd$, and $\vela=\velb$. Then, $\pathd$ is strongly defendable if and only if $\xan$ is outside the region induced by $\ppath$: $\xan\in\amb\backslash\left(\rpa \cup \rpb\right)$.
\end{lem}

\begin{IEEEproof} % This assumes speed of attacker is same as (or less than) speed of defender 
See Fig. \ref{fig:lemma1}. Assume $\xan \notin \target = \sac$, otherwise the attacker would start inside the target set. 

First, we show that if $\xan\in \rpa \cup \rpb$, then the attacker can reach $\apa$ or $\apb$ and hence $\sac$ without being captured. Without loss of generality (WLOG), suppose $\xan\in\rpa$. To capture the attacker, the defender's capture set must contain $\xai(\apa)$ or $\xai(\apb)$ at some time $t$. By Definition \ref{def:d_win_region}, we have $\dist(\xan,\apa) < \dist(\Dc(\ppath),\apa)$, so $\ta(\xai(\apa),\apa) < \tb(\Dc(\ppath),\apa)$. By Proposition \ref{rem:image_of_a}, $\dist(\xai(\apb),\apa) \le \dist(\xai(\apa),\apa)$, so it suffices to show that the defender's capture set cannot reach $\xai(\apa)$ before the attacker reaches $\apa$. 

If the attacker moves towards $\apa$ along $\spath(\xan,\apa)$ with maximum speed, then $\xai(\apa)$ will move towards $\apa$ along $\spath(\xai(\apa),\apa)$ at the same speed. Since $\ta(\xa,\apa)=\ta(\xai(\apa),\apa)<\tb(\Dc(\ppath),\apa)$, $\xa$ will reach $\apa$ before the defender capture set $\Dc(\xb,t)$ does. 

Next we show, by contradiction, that if $\xa\notin\rpa \cup \rpb$, then the attacker cannot reach $\sac$ without being captured. Suppose $\pa$ will reach some point $\pprime$ before $\Dc(\pb,t)$ does, i.e. $\dist(\xan,\pprime)<\dist(\Dc(\xbn),\pprime)=\dist(\Dc(\ppath),\pprime)$. WLOG, assume $\pprime\in\spath(\ppath,\apb)$, and note that $\dist(\Dc(\ppath),\apb)<\dist(\xan,\apb)$ since the attacker is not in $\rpb$. Starting with the definition of the shortest path, we have
\bq
\begin{aligned}
\dist(\xan,\apb) &\le \dist(\xan,\pprime) + \dist(\pprime,\apb) \\
%& \text{(definition of shortest path / triangle inequality)} \\
&< \dist(\Dc(\ppath),\pprime) + \dist(\pprime,\apb) \\
%& \text{(by assumption, $\dist(\xan,\pprime)<\dist(\Dc(\ppath),\pprime)$)}\\ 
&= \dist(\Dc(\ppath),\apb) \\
\dist(\xan,\apb)&< \dist(\xan, \apb) \quad \text{(since $\xan\notin\rpa$)}
\end{aligned}
\eq
This is a contradiction. Therefore, the attacker cannot cross any point $\pprime$ on $\pathd$ without being captured.
\end{IEEEproof}

\begin{figure}
\centering
\includegraphics[width=0.45\textwidth]{fig/Ra_Rb}
\caption{Left: If the attacker is in $\rpa\cup\rpb$ and moves towards $e_a$, he  will be able to reach $\apa$ without being captured. Right: If the attacker is not in $\rpa\cup\rpb$, there is no point on the path $\pprime\in\pathd$ that the he can reach without being captured.}
\label{fig:lemma1}
\end{figure}
%\begin{figure}
%\centering
%	\begin{subfigure}{0.3\textwidth}
%	\centering
%	\includegraphics[width=\textwidth]{"fig/attacker in Ra"}
%	\caption{\label{subfig:ainra}}
%	\end{subfigure} \quad
%	\begin{subfigure}{0.3\textwidth}
%	\centering
%	\includegraphics[width=\textwidth]{"fig/attacker outside Ra U Rb"}
%	\caption{\label{subfig:aoutra}}
%	\end{subfigure}
%\caption{(\ref{subfig:ainra}) Suppose the attacker is in $\rpa\cup\rpb$. If the attacker moves towards $e_a$, the defender cannot capture the attacker's level set image before the attacker reaches $\apa$. (\ref{subfig:aoutra}) Suppose the attacker is not $\rpa\cup\rpb$, there is no point $\pprime$ on $\pathd$ that the attacker can reach without being captured.}
%\label{fig:lemma1}
%\end{figure}

If $\vela<\velb$, $\pa$ being outside of $\rpa \cup \rpb$ becomes a sufficient condition for the strong defendability of $\pathd$.

In general, $\xbn$ may not be on $\pathd$. In this case, if the defender can arrive at $\ppath$ before the attacker moves into $\rpa(\ppath)\cup\rpb(\ppath)$, then $\pathd$ is strongly defendable. Thus, given $\xjn = (\xan,\xbn)$, we may naively check whether a path $\pathd$ is strongly defendable by checking whether there exists some $\ppath\in\pathd$ such that $\tb(\xbn, \ppath)\le\ta\left(\xan,\rpa(\ppath)\cup\rpb(\ppath)\right)$. If so, then $\pathd$ is strongly defendable. The next lemma shows that it is necessary and sufficient to check whether \textit{one} special point, $\pstar\in\pathd$, can be the first arrival point for strongly defending $\pathd$.

\begin{rem} \label{rem:time_to_region_a}
Given $\ppath\in\pathd$, $\dist\left(\xan,\rpa(\ppath)\right) = \dist(\xan,\apa) - \dist(\Dc(\ppath),\apa)$. Similarly, $\dist\left(\xan,\rpb(\ppath)\right) = \dist(\xan,\apb) - \dist(\Dc(\ppath),\apb)$.
\end{rem}

\begin{lem} \label{lem:pstar}
Define $\pstar\in\pathd$ such that $\ta(\xan,\rpa)=\ta(\xan,\rpb)$. Then, $\pathd$ is strongly defendable if and only if the defender can defend $\pathd$ by first going to $\pstar$.
\end{lem}

\begin{IEEEproof}
One direction is clear by definition.

We now show the other direction's contrapositive: if the defender cannot defend $\pathd$ by first going to $\pstar$, then $\pathd$ is not strongly defendable. Equivalently, we show that if choosing $\pstar$ as the first entry point does not allow the defender to defend $\pathd$, then no other entry point does.

Suppose that the defender cannot defend $\pathd$ by choosing $\pstar$ as the first entry point, but can defend $\pathd$ by choosing another entry point $\pprime$. WLOG, assume $\dist(\Dc(\pstar),\apa)>\dist(\Dc(\pprime),\apa)$. This assumption moves $\pprime$ further away from $\apa$ than $\pstar$, causing $\rpa$ to move closer to $\xan$. Starting with Remark \ref{rem:time_to_region_a}, we have
\bq
\begin{aligned}
\dist\left(\xan,\rpa(\pstar)\right) & = \dist(\xan,\apa) - \dist(\Dc(\pstar),\apa) \\
\ta\left(\xan,\rpa(\pstar)\right) & = \ta(\xan,\apa) - \ta(\Dc(\pstar),\apa) 
\end{aligned}
\eq

Similarly, for the point $\pprime$, we have
\bq
\begin{aligned}
\dist\left(\xan,\rpa(\pprime)\right) & = \dist(\xan,\apa) - \dist(\Dc(\pprime),\apa) \\
\ta\left(\xan,\rpa(\pprime)\right) & = \ta(\xan,\apa) - \ta(\Dc(\pprime),\apa) 
\end{aligned}
\eq

Then, subtracting the above two equations, we see that the attacker can get to $\rpa$ sooner by the following amount:
\bq
\begin{aligned}
& \ta\left(\xan,\rpa(\pstar)\right) - \ta\left(\xan,\rpa(\pprime)\right) \\
&= \ta(\Dc(\pprime),\apa) - \ta(\Dc(\pstar),\apa) \\
&=\ta(\pprime,\pstar) \ge \tb(\pprime,\pstar)
\end{aligned}
\eq

We now show that the defender can get to $\pprime$ sooner than to $\pstar$ by less than the amount $\tb(\pprime,\pstar)$, and in effect ``gains less time" than the attacker does by going to $\pprime$. We assume that $\pprime$ is closer to the defender than $\pstar$ is (otherwise the defender ``loses time" by going to $\pprime$). By the triangle inequality,
\bq
\begin{aligned}
\dist(\xbn,\pstar) & \leq \dist(\xbn,\pprime) + \dist(\pprime,\pstar) \\
\dist(\xbn,\pstar) - \dist(\xbn,\pprime) & \leq  \dist(\pprime,\pstar) \\
\tb(\xbn,\pstar) - \tb(\xbn,\pprime) & \leq  \tb(\pprime,\pstar)
\end{aligned}
\eq
\end{IEEEproof}
Lemmas \ref{lem:d_winning_region} and \ref{lem:pstar} give a simple algorithm to compute, given $\xan$, the region that the defender must be in for a path of defense $\pathd$ to be strongly defendable:
\begin{enumerate}
\item Given $\apa, \apb,\xan$, compute $\pstar$ and $\rpa(\pstar),\rpb(\pstar)$.
\item If $\vela=\velb$, then $\pathd$ is strongly defendable if and only if $\xbn\in\dr(\apa,\apb;\xan)=\{x:\tb(\x,\pstar) \le \ta(\xan,\rpa \cup \rpb)\}$. %If $\vela\le\velb$, then $\xbn\in\dr(\apa,\apb;\xan)$ becomes a sufficient condition for $\pathd$ being strongly defendable.
\end{enumerate}

The computations in this algorithm can be efficiently done by solving a series of 2D Eikonal equations by using FMM \cite{Sethian1996}, reducing our 4D problem to 2D. Fig. \ref{fig:lemma2} illustrates the proof of Lemma \ref{lem:pstar} and the defender winning region $\dr$.

\begin{figure}
\centering
\includegraphics[width=0.45\textwidth]{"fig/winning_regions"}
\caption{Left: If the defender cannot defend $\pathd$ by first going to $\pstar$, then he cannot defend $\pathd$ by going to any other point $\ppath$. Right: Defender winning region $\dr$.}
\label{fig:lemma2}
\end{figure}
%
%\begin{figure}
%\centering
%	\centering
%	\begin{subfigure}{0.3\textwidth}
%	\includegraphics[width=\textwidth]{"fig/best point pstar"}
%	\caption{\label{subfig:pstar}}
%	\end{subfigure}
%	\begin{subfigure}{0.3\textwidth}
%	\centering
%	\includegraphics[width=\textwidth]{"fig/defender winning pd"}
%	\caption{\label{subfig:dregion}}
%	\end{subfigure}
%	\caption{(\ref{subfig:pstar}) If the defender cannot defend $\pathd$ by first going to $\pstar$, then the defender cannot defend $\pathd$ by going to any other point $\ppath$. (\ref{subfig:dregion}) Lemma \ref{lem:pstar} allows an easy computation of $\dr$.}
%	\label{fig:lemma2}
%\end{figure}

% \subsection{Conservatism of strong path defense \label{subsec:pd_cons}}
% To quantify how conservative strong path defense is, we consider the following question: what paths of defense are defendable, but not strongly defendable? One example of such a path is the Voronoi line between an attacker and a defender with equal speeds in a convex domain. This is shown in figure \ref{fig:voronoi_pod}.

% \begin{figure}[h]
% \centering

% \begin{subfigure}{0.45\textwidth}
% \includegraphics[width=\textwidth]{"fig/defend voronoi"}
% \caption{The defender can defend the initial Voronoi line by mirroring the attacker's movement.}
% \label{subfig:vor_pod1}
% \end{subfigure}

% \begin{subfigure}{0.45\textwidth}
% \includegraphics[width=\textwidth]{"fig/voronoi strong defense"}
% \caption{If the defender tries to arrive at some point $\ppath$, and then strongly defend it, then the attacker will arrive at one of the anchor points before the defender.}
% \label{subfig:vor_pod2}
% \end{subfigure}

% \caption{The Voronoi line is a path of defense that is defendable but not strongly defendable.}
% \label{fig:voronoi_pod}
% \end{figure}

% To quantify the conservatism of strong path defense compared to path defense, we compute the defendable region and the strongly defendable region, defined below.

% First, observe that if a path is strongly defendable, any path ``behind" it on the defender's side is also strongly defendable. This is illustrated in figure \ref{fig:path_behind}. Suppose $\pathds{i}$ is strongly defendable, then it is defendable. This necessarily implies that $\pathds{j}$ is defendable, because if the defender defends $\pathds{i}$, then the attacker can never cross $\pathds{i}$ and thus never cross $\pathds{j}$. Since $\pathds{i}$ is strongly defendable, the defender can move to some point on $\pathds{i}$, and then defend $\pathds{i}$. However, on the way to $\pathds{i}$, the defender would have crossed $\pathds{j}$. Now, we have that the defender arrives at some point on $\pathds{j}$ and can defend it afterwards; therefore $\pathds{j}$ is strongly defendable.

% \begin{figure}[h]
% \includegraphics[width=0.45\textwidth]{"fig/path behind"}
% \caption{$\pathds{i}$ is strongly defendable if $\pathds{j}$ is strongly defendable.}
% \label{fig:path_behind}
% \end{figure}

% \begin{defn}
% Given joint initial condition $\xn$ and a (strongly) defendable path $\pathd$, call the region $\sac$ the \textbf{(strongly) defendable region given $\pathd$}. Define the union of all (strongly) defendable regions given paths to be the \textbf{(strongly) defendable region}, denoted ($\sdr$) $\dr$:
% \bq
% \begin{aligned}
% \dr &= \bigcup_{\pathd \text{ defendable}} \sac\\
% \sdr &= \bigcup_{\pathd \text{ strongly defendable}} \sac
% \end{aligned}
% \eq
% \end{defn}

% For a given joint initial condition $\xn$, we express the conservatism of strong path defense as the ratio $\text{area}(\dr) / \text{area}(\sdr)$. 

