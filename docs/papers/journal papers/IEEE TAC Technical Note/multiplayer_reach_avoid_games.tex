\documentclass[technote]{IEEEtran}
%\documentclass[12pt,draftcls,onecolumn]{IEEEtran}
%\usepackage{hyperref}
% show labels
%\usepackage[notref,notcite]{showkeys}
%\usepackage[T1]{fontenc}
%\usepackage[latin9]{inputenc}
%\usepackage{amsthm}
\usepackage{amsmath}
\usepackage{amssymb}
\usepackage{graphics}
\usepackage{graphicx}
%\usepackage{enumitem}
\usepackage{color}
%\usepackage{algorithm2e}
%\usepackage{appendix}
%\usepackage{algorithmic}
%\usepackage{program}
\usepackage{subcaption}

%\makeatletter
%%%%%%%%%%%%%%%%%%%%%%%%%%%%%% Textclass specific LaTeX commands.
%\theoremstyle{plain}
%\newtheorem{thm}{Theorem}
%\theoremstyle{plain}
%\newtheorem{prop}[thm]{Proposition}
% \ifx\proof\undefined\
%   \newenvironment{proof}[1][\proofname]{\par
%     \normalfont\topsep6\p@\@plus6\p@\relax
%     \trivlist
%     \itemindent\parindent
%     \item[\hskip\labelsep
%           \scshape
%       #1]\ignorespaces
%   }{%
%     \endtrivlist\@endpefalse
%   }
%   \providecommand{\proofname}{Proof}
% \fi
%\theoremstyle{remark}
\newtheorem{rem}{Remark}
%\theoremstyle{definition}
\newtheorem{defn}{Definition}
\newtheorem{prop}{Proposition}
\newtheorem{thm}{Theorem}
\newtheorem{lem}{Lemma}
\newtheorem{alg}{Algorithm}

%\theoremstyle{plain}
%\newtheorem{lem}[thm]{Lemma}
%\theoremstyle{plain}
%\newtheorem{cor}[thm]{Corollary}
%
%\newtheorem{example}{Example}

%\makeatother

%\usepackage{babel}

\newcommand{\NA}{N_A}
\newcommand{\ND}{N_D}

\usepackage{"../../my_macros"}

\title{Multiplayer Reach-Avoid Games via Pairwise Outcomes}

\author{Mo Chen, Zhengyuan Zhou and Claire J. Tomlin%
\thanks{This work has been supported in part by NSF under CPS:ActionWebs (CNS-931843), by ONR under the HUNT (N0014-08-0696) and SMARTS (N00014-09-1-1051) MURIs and by grant N00014-12-1-0609, by AFOSR under the CHASE MURI (FA9550-10-1-0567).}% <-this % stops a space
\thanks{M.~Chen, and C.~J.~Tomlin are with the Department of Electrical Engineering and Computer Sciences,
        University of California, Berkeley, CA 94720, USA
        {\tt\small \{mochen72,tomlin\}@eecs.berkeley.edu}}
\thanks{Z.~Zhou is with the Department of Electrical Engineering,
        Stanford University, Stanford, CA 94305, USA
        {\tt\small zyzhou@stanford.edu}}   
\thanks{We thank Haomiao Huang for sharing MatLab code for 4D HJI calculations.}
}

\begin{document}

\maketitle
\thispagestyle{empty}
\pagestyle{empty}


\begin{abstract}
A multiplayer reach-avoid game is a differential game between an attacking team with $\NA$ attackers and a defending team with $\ND$ defenders playing on a compact domain with obstacles. The attacking team aims to send $\m$ of the $\NA$ attackers to some target location, while the defending team aims to prevent that by capturing attackers or indefinitely delaying attackers from reaching the target. The analysis of this game plays an important role in many applications. The optimal solution to this game is computationally intractable when $\NA>1$ or $\ND>1$. In this paper, we present two approaches for the $\NA=\ND=1$ case to determine pairwise outcomes, and a graph theoretic maximum matching approach to merge these pairwise outcomes for an $\NA,\ND>1$ solution that provides guarantees on the performance of the defending team. We will show that the four-dimensional Hamilton-Jacobi-Isaacs approach allows for real-time updates to the maximum matching, and that the two-dimensional ``path defense" approach is considerably more scalable with the number of players while maintaining defender performance guarantees.
\end{abstract}

% A multiplayer reach-avoid game is a differential game between an attacking team with NA attackers and a defending team with ND defenders playing on a compact domain with obstacles. The attacking team aims to send M of the NA attackers to some target location, while the defending team aims to prevent that by capturing attackers or indefinitely delaying attackers from reaching the target. The analysis of this game plays an important role in many applications. The optimal solution to this game is computationally intractable when NA>1 or ND>1. In this paper, we present two approaches for the NA=ND=1 case to determine pairwise outcomes, and a graph theoretic maximum matching approach to merge these pairwise outcomes for an NA,ND>1 solution that provides guarantees on the performance of the defending team. We will show that the four-dimensional Hamilton-Jacobi-Isaacs approach allows for real-time updates to the maximum matching, and that the two-dimensional "path defense" approach is considerably more scalable with the number of players while maintaining defender performance guarantees.

\section{Introduction}
% Applications of differential games/reach-avoid games
% - CTF
% - warehouse robots
% - air traffic control
% - etc.

Multiplayer reach-avoid games are differential games between two adversarial teams of cooperative players playing on a compact domain with obstacles. One team, called the attacking team, aims to send as many team members, called attackers, to some target set as quickly as possible. The other team, called the defending team, seeks to delay or prevent the attacking team from doing so by attempting to capture the attackers. Such differential games have been studied extensively, and are not only useful in analyzing games such as Capture-the-Flag (CTF), which was explored most notably in the Cornell Roboflag competition \cite{HThesis, Huang2011, Earl:2007p101, Campbell:2003p5, Waydo:2003p97, Parasuraman:2005p99}. The multiplayer reach-avoid game that we present in this paper is also a powerful theoretical tool for analyzing realistic situations such as autonomous robots maneuvering in a warehouse trying to reach preset destinations while avoiding other robots, and aircrafts trying to reach a target airport while avoiding other aircrafts in the vicinity, among other applications in robotics, aircraft control, security, and other domains \cite{OFTHEAIRFORCEWASHINGTON:2009p37, Erzberger:2006p44, kiva2009}.

% Reach-avoid game difficulty 
% - unintuitive
% - exponential scaling
% - previous solution attempts not general

The multiplayer reach-avoid game is difficult to analyze for several reasons. First, the two teams have conflicting and asymmetric goals while at the same time, complex cooperation among the players exists within each team. The optimal control for each player is difficult to intuit and visualize even in the case of a 1 vs. 1 game, in which human players sometimes lose in situations in which an optimal winning strategy exists \cite{HThesis, Huang2011}. Also, in the general multiplayer reach-avoid game, optimal solutions are impossible to compute using traditional dynamic programming approaches due to the intrinsic high dimensionality of the joint state space. 

Multiplayer differential games have been previously addressed using various techniques. In \cite{Earl:2007p101}, where a team of defenders assumes that the attackers move towards their target in straight lines, a mixed-integer linear programming approach was used. In \cite{Chasparis:2005p102}, optimal defender strategies can also be determined using a mixed integer linear program, with the assumption that the attackers use a linear feedback control law; the mixed integer linear program here is then relaxed and a linear program is solved instead. In complex pursuit-evasion games where players may change roles over time, a nonlinear model-predictive control \cite{Sprinkle:2004p100} approach has been investigated. In this case, a cost function of the opponent is assumed, and opponent strategies are estimated based on an explicit prediction model. Approximate dynamic programming (ADP) \cite{McGrew:2008p103} has also been used to analyze reach-avoid games. While in some cases ADP provides good estimates of the value function that would be obtained from dynamic programming, guarantees on the direction of conservatism typically cannot be made.

% Previous most-related work and shortcomings
% - HJI: optimal but intractible
% - open-loop: fast but conservative
Although the above techniques provide some useful insight into multiplayer differential games, they only work well when strong assumptions are made or when accurate models of the opposing team can be obtained. Furthermore, the techniques cannot be easily adapted to solve a general multiplayer reach-avoid game when no prior information of the control strategies of each team is known. To solve general reach-avoid games, the Hamilton-Jacobi-Isaacs (HJI) approach \cite{b:isaacs-1967} is ideal when the game is low-dimensional. The HJI approach involves solving an HJI partial differential equation (PDE) on a grid to compute a reach-avoid set in the joint state space of the players. The reach-avoid set partitions the joint state space of the players into a winning region for the defending team and a winning region for the attacking team, assuming both sides play optimally. The optimal strategies can then be extracted from the gradient of the solution to the HJI PDE. The HJI approach is particularly useful because of the many numerical tools \cite{j:mitchell-TAC-2005, Sethian1996, b:osher-fedkiw-2002} available to numerically solve the HJI PDE. The approach has been able to solve a variety of optimal control problems such as aircraft collision avoidance \cite{j:mitchell-TAC-2005}, automated in-flight refueling \cite{DSST08}, and two-player reach-avoid games \cite{Huang2011}. The advantage of the HJI approach is that it can be applied to a large variety of player dynamics and does not explicitly assume any control strategy or prediction models for the players. However, the HJI approach cannot be directly applied to our multiplayer reach-avoid game because the complexity of solving the HJI PDE on a grid scales exponentially with the number of players, making the HJI approach only applicable to the 1 vs. 1 game. Even in the 1 vs. 1 case, the memory requirement is significant even for a relatively coarse grid. Therefore, when analyzing the multiplayer reach-avoid game, complexity-optimality trade-offs must be made.

\cite{Zhou2012} presented an open-loop approach to approximate the solution to the 1 vs. 1 reach-avoid game, in which the time for attacker to reach the target was conservatively estimated by assuming that the defender first chooses an open-loop control strategy, after which the attacker chooses a control strategy in response. The open-loop approach utilizes a modified version of the fast marching method (FMM) \cite{Sethian1996,Zhou2012}, and is extremely computationally efficient. Another advantage of this approach is that it is conservative towards the defender and provides guarantees on the defender's performance. However, the degree of conservatism can be very high because the defender is assumed to use an open-loop control strategy. The open-loop approach, therefore, can be considered an extreme trade off between the complexity of solving the HJI PDE and the optimality of the control strategy. 

% Current paper contributions
% - two-player to multiplayer
% - HJI solution: powerful for identical dynamics within each team
% - PD solution: efficient, more general dynamics within each team, better scaling
Our contributions in this paper are as follows: First of all, we propose a novel way of approximating the solution to the multiplayer reach-avoid game by merging together the $\N^2$ pairwise outcomes between players of opposite teams in the $\N$ vs. $\N$ game using the graph theoretic maximum matching. Pairwise outcomes can be obtained by solving the 1 vs. 1 (two-player) reach-avoid game. Secondly, we show that the HJI solution to the 1 vs. 1 game can be used to provide real-time updates to the optimal pairwise outcomes at virtually no additional computation cost, and that in the case where the players on each team have identical dynamics, only a \textit{single} HJI PDE needs to be solved to characterize \textit{all} pairwise outcomes. Finally, we present the new ``path defense" approach, an efficient and conservative approximation to the HJI solution of the 1 vs. 1 game in which the defenders utilize a ``semi-open-loop" control strategy. This path defense approach conservatively approximates two-dimensional (2D) slices of the reach-avoid sets given by the HJI solution by solving a series of 2D Eikonal equations using FMM, while maintaining guarantees on the defending team's performance. The computation complexity in the path defense approach is thus vastly reduced compared to the HJI approach, and the degree of conservatism is much lower than the degree of conservatism in the open-loop approach. Furthermore, we show that in general, when applying maximum matching to the path defense solution to the 1 vs. 1 game, the computational complexity scales linearly with the number of players, as opposed to quadratically in the HJI approach. 
% Motivation and significance of result

%\section{Related Work}
\label{sec:related}
Refer to older papers... maybe combine with introduction 
% CTF paper, Ryo OL paper, OL paper, etc.

% !TEX root = multiplayer_reach_avoid_games.tex
\section{The Reach-Avoid Problem}
\subsection{The Multiplayer Reach-Avoid Game}
\label{sec:formulation}
Consider $\NA+\ND$ players partitioned into the set of $\NA$ attackers, $\pas = \{\pam{1}, \pam{2}, \ldots, \pam{\NA}\}$ and the set of $\ND$ defenders, $\pbs = \{\pbm{1}, \ldots, \pbm{\ND}\}$, whose states are confined in a bounded, open domain $\amb \subset \R^2$. The domain $\amb$ is partitioned into $\amb$ = $\free \cup \obs$, where $\free$ is a compact set representing the free space in which the players can move, while $\obs = \amb \setminus \free$ corresponds to obstacles in the domain. 

Let $\xam{i}, \xbm{j} \in \R^2$ denote the state of players $\pam{i}$ and $\pbm{j}$, respectively. Then given initial conditions $\xanm{i}\in \free,i=1,2,\ldots,\NA,\xbnm{i}\in \free,i=1,2,\ldots,\ND$, we assume the dynamics of the players to be defined by the following decoupled system for $t \geq 0$:

\bq\label{eq:dynamics}
\begin{aligned}
\dotxam{i}(t) &= \velai{i}\cam{i}(t), & \xam{i}(0) = \xanm{i}, i=1,2,\ldots,\NA \\
\dotxbm{i}(t) &= \velbi{i}\cbm{i}(t), & \xbm{i}(0) = \xbnm{i}, i=1,2,\ldots,\ND
\end{aligned}
\eq
where $\velai{i}, \velbi{i}$ denote maximum speeds for $\pam{i}$ and $\pbm{i}$ respectively, and $\cam{i},\cbm{i}$ denote controls of $\pam{i}$ and $\pbm{i}$ respectively. We assume that $\cam{i},\cbm{i}$ are drawn from the set $\A = \{\sigma \colon [0,\infty)\rightarrow \unitball \mid \sigma \text{ is measurable}\}$, where $\unitball$ denotes the closed unit disk in $\R^2$. We also constrain the players to remain within $\free$ for all time. Denote the joint state of all players by $\xj = (\xja, \xjb)$ where $\xja =(\xam{1},\ldots\xam{\NA})$ is the attacker joint state $\pas$, and $\xjb = (\xbm{1},\ldots,\xbm{\ND})$ is the defender joint state $\pbs$. 

The attacking team wins whenever $\m$ of the $\NA$ attackers reach some target set without being captured by the defenders; $\m$ is pre-specified with $0<M\le \NA$. The target set is denoted $\target\subset\free$ and is compact. The defending team wins if it can prevent the attacking team from winning by capturing or indefinitely delaying $\NA-\m+1$ attackers from reaching $\target$. An illustration of the game setup is shown in Fig. \ref{fig:mp_form}.

We will define the capture sets, $\avoid_{ij}$, to be $\avoid_{ij} = \left\{\xj\in\amb^{\NA+\ND} \mid \|\xam{i}-\xbm{j}\|_2\le\Rc \right\}$. $\pam{i}$ is captured by $\pbm{j}$ if $\pam{i}$'s position is within a distance $\Rc$ of $\pbm{j}$'s position. 

\begin{figure}
\centering
\includegraphics[width=0.35\textwidth]{"fig/formulation"}
\caption{The components of a multiplayer reach-avoid game.}
\label{fig:mp_form}
\end{figure}

In this paper, we address the following problems:
\begin{enumerate}
\item Given $\xjn$, $\target$, and some fixed integer $\m, 0<\m\le\NA$, can the attacking team win?
\item More generally, given $\xjn$ and $\target$, how many attackers can the defending team prevent from reaching the target?
\end{enumerate}

\subsection{The Two-Player Reach-Avoid Game}
\label{sec:2p_ra}
We will answer the above questions about the general $\NA$ vs. $\ND$ reach-avoid game by using the solution to the two-player $1$ vs. $1$ game as a building block. In the two-player game, we denote the attacker $\pa$, the defender $\pb$, their states $\xa,\xb$, and their initial conditions $\xan,\xbn$. Their dynamics are
\bq
\begin{aligned}
\dotxa(t) &= \vela\ca(t), & \xa(0) = \xan,\\
\dotxb(t) &= \velb\cb(t), & \xb(0) = \xbn
\end{aligned}
\eq

The players' joint state becomes $\xj=(\xa,\xb)$, and their joint initial condition becomes $\xjn=(\xan,\xbn)$. The capture set becomes simply $\avoid = \left\{(\xa,\xb)\in\amb^2 \mid \|\xa-\xb\|_2\leq \Rc\right\}$. 

$\pa$ wins if it reaches the target $\target$ without being captured by $\pb$. $\pb$ wins if it can prevent $\pa$ from winning by capturing $\pa$ or indefinitely delaying $\pa$ from reaching $\target$. For the two-player reach-avoid game, we seek to answer the following:
\begin{enumerate}
\item Given $\xjn$ and $\target$, is the defender guaranteed to win? \label{p:tp1}
\item More generally, given $\xa$ and $\target$, what is the set of initial positions such that the defender is guaranteed to win? \label{p:tp2}
\end{enumerate}
% General domain, # of players, goal of attackers
% Specialization to 2 player case

% HJI solution of 2 player game (CTF paper)
\section{The Hamilton-Jacobi-Isaacs Solution of the Two-Player Game} \label{sec:solution_hji}
In this section, we describe the HJI approach for solving differential games with arbitrary terrain, domain, obstacles, target set, and player velocities based on \cite{Huang2011,j:mitchell-TAC-2005, LSToolbox}. In particular, we will show how to compute the optimal joint control strategies for the attacker and the defender in a two-player reach-avoid game by solving a 4D HJI PDE. This solution allows us to determine whether the defender will win against the attacker in a 1 vs. 1 setting. 

In general, to approximate the solution to the $\N$ vs. $\N$ game, we solve $\N^2$ HJI PDEs corresponding to the $\N^2$ attacker-defender pairs, and then piece together the pairwise solutions using the maximum matching approach described in Section \ref{subsec:max_match}. In Section \ref{subusec:MMHJI}, we will show that in the special case where players on each team have the same dynamics,  only a \textit{single} 4D HJI PDE needs to be solved. This is because the solution to the 4D HJI PDE completely characterizes the outcome of the game given \textit{any} joint initial condition.

\subsection{Hamilton-Jacobi-Isaacs Reachability} \label{subsec:hj_background}
The multiplayer reach-avoid game is a differential game in which two teams have competing objectives \cite{b:basar-olsder-1999}. The results of the HJI computations assume a closed-loop strategy for both players given previous information of the other players. The setup for using the HJI approach to solve differential games can be found in \cite{Huang2011, j:mitchell-TAC-2005, LSToolbox}. In summary, we are given the continuous dynamics of the system state:

\bq
\dxj = f(\xj,u,d), \xj(0)=\xjn
\eq

\noindent where $\xj\in\R^n$ is the system state, $u\in\mathbb{U}$ is the joint control input of the attacking team, and $d\in\mathbb{D}$ is the joint control input of the defending team. In our reach-avoid game, $n=4\N$ where $\N$ is the number of players on each team. The sets $\mathbb{U}$ and $\mathbb{D}$ represent the sets of the joint admissible control inputs of the attacking team and the defending team, respectively. The attacking team selects a control input based on the past and the current joint states of all the players. The defending team then selects a control input based on the past and the present control inputs of the attacking team, in addition to the past and the current joint states. \textit{A priori}, this information pattern is conservative towards the attackers, as defenders have more information available. However, in the case that the system (described by the function $f$) is decoupled, as in Equation (\ref{eq:dynamics}), the Isaacs condition \cite{b:isaacs-1967} holds and this information pattern yields the same optimal solutions for both the attackers and the defenders compared to the analagous information pattern that is conservative towards the defenders.

In the HJI approach, we specify the terminal set $R$ as the attackers' winning condition, and propagate backwards this set for some time horizon, subject to the constraint imposing that the attackers be outside the capture regions and the obstacles. This constraint is described by the avoid set $A$. The result of the backwards propagation is a reach-avoid set that partitions the state space into two regions. All points inside the reach-avoid set represent the joint initial conditions from which the attacking team is guaranteed to win within a certain time horizon, and all points outside represent the joint initial conditions from which the attacking team is guaranteed to not win within a certain time horizon.

More precisely, the HJI reachability calculation is as follows. First, given a set $G$, the level set representation of $G$ is a function $\valsR_\set:\R^n \rightarrow \R$ such that $\set = \left\{\xj\in\R^n \mid \valsR_\set\le0\right\}$. In particular, the terminal set $R$ and the avoid set $A$ are represented by the functions $\valsR_R(\xj)$ and $\valsR_A(\xj)$ respectively.
 
Let $\Phi:\R^n\times[-\T,0]\rightarrow\R$ be the viscosity solution \cite{j:Crandall-TAMS-1983} to the constrained terminal value HJI PDE in the form of an HJI variational inequality:
\begin{equation} \label{eq:HJ_PDE_reachavoid}
\begin{aligned}
	\max\Big\{\frac{\partial \Phi}{\partial t} + \min \left[0, H\left(\xj,\nabla_{\xj}\Phi(\xj)\right)\right], \qquad\\
-\valsR_A(\xj)-\Phi(\xj,t)\Big\} = 0\\
\Phi(\xj,0) = \valsR_R(\xj)	 
\end{aligned}
\end{equation}
where the optimal Hamiltonian is given by
$$H\left(\xj,p\right) = \min_{u \in \mathbb{U}} \max_{d \in \mathbb{D}} p \cdot f(\xj,u,d)$$
By the argument presented in \cite{j:mitchell-TAC-2005,mitchell-thesis,bokanowski10}, the set of initial conditions from which the attackers are guaranteed to win within time $T$ is given by 
\bq
\mathcal{RA}_T(R,A) := \left\{\xj\in \R^n \mid \Phi (\xj,-\T) \leq 0\right\}
\eq
Hence, $\Phi (\xj,-\T)$ is the level set representation of $\mathcal{RA}_T(R,A)$.

The optimal control input for the attacking team is given by \cite{Huang2011,j:Lygeros-automatica-1999, j:Tomlin-ProcIEEE-2000}:
\bq \label{eq:opt_ctrl_u}
  u^*(\xj,t) = \arg \min_{u \in \mathbb{U}} \max_{d \in \mathbb{D}} \nabla_{\xj}\Phi(\xj,t) \cdot f(\xj,u,d), \!\ t \in [-T,0] 
\eq

Similarly, an initial player configuration outside $\mathcal{RA}_T(R,A)$ guarantees that the defenders will be able to delay the attackers from winning for time $\T$ by using the optimal control input
\bq \label{eq:opt_ctrl_d}
  d^*(\xj,t) = \arg \max_{d \in \mathbb{D}} \nabla_{\xj}\Phi(\xj,t)\cdot f(\xj,u^*,d), \ t \in [-T, 0]
\eq

Taking $T\rightarrow\infty$, we obtain the set of initial conditions from which the attackers are guaranteed to win. We denote this set $\mathcal{RA}_\infty(R,A)$. The set of initial conditions from which the defenders are guaranteed to win is given by all points \textit{not} in $\mathcal{RA}_\infty(R,A)$. For a two-player game on a 2D domain $\amb\subset\R^2$, the reach-avoid set $\mathcal{RA}_\infty(R,A)$ is 4D.

A highly accurate numerical solution to Equation (\ref{eq:HJ_PDE_reachavoid}) can be computed using the Level Set Toolbox for MATLAB \cite{LSToolbox}.

Next, we will describe the terminal set and the avoid set for the two player reach-avoid game.

\subsection{Hamilton-Jacobi-Isaacs Reachability for the Two-Player Game}
\label{subsec:hj_two}
In the two-player reach-avoid game, the goal of the attacker is to reach the target set $\target$ while avoiding capture by the defender. This goal is represented by the attacker being inside $\target$. En route to $\target$, the attacker must avoid capture by the defender. This is represented by the set $\avoid$. 

In addition, both players need to avoid the obstacles $\obs$, which can be considered to be the locations in $\amb$ where the players have zero velocity. In particular, the defender wins if the attacker is in $\obs$, and vice versa. Therefore, we define the terminal set as 

\begin{equation} \label{eq:4DHJI_terminal_set}
R = \left\{\xj\in\amb^2 \mid \xa\in\target \right\} \cup \left\{\xj\in\amb^2\mid \xb\in\obs \right\}
\end{equation}

\noindent Similarly, we define the avoid set as
\bq \begin{aligned}
A &= \left\{\xj\in\amb^2 \mid \|\xa-\xb\|_2\le \Rc \right\} \cup \left\{\xj\in\amb^2\mid \xa\in\obs \right\} \\
&= \avoid \cup \left\{\xj\in\amb^2\mid \xa\in\obs \right\}
\end{aligned} 
\label{eq:4DHJI_avoid_set}
\eq

Given these sets, we can define the corresponding level set representations $\valsR_R,\valsR_A$, and solve (\ref{eq:HJ_PDE_reachavoid}). Assuming $\amb\subset\R^2$, the result is $\mathcal{RA}_\infty(R,A)\in\R^4$, a 4D reach-avoid set with the level set representation $\Phi(\xj,-\infty)$. The attacker wins if and only if the joint initial condition is such that $(\xan,\xbn)=\xjn\in\mathcal{RA}_\infty(R,A)$.

If $\xjn\in\mathcal{RA}_\infty(R,A)$, then the attacker is guaranteed to win the game by using the optimal control input given in (\ref{eq:opt_ctrl_u}). Applying Equation (\ref{eq:opt_ctrl_u}) to the two-player game, we have that the attacker winning strategy satisfies 

\bq \label{eq:opt_ctrl2_u_gen}
  \ca^*(\xa,\xb,t) = \arg \min_{\ca \in \mathbb{U}} \max_{\cb \in \mathbb{D}} p(\xa,\xb,t) \cdot f(\xa,\xb,\ca,\cb) 
\eq

\noindent for $t\le 0$. The explicit winning strategy satisfying (\ref{eq:opt_ctrl2_u_gen}) is given in \cite{Huang2011} as

\bq \label{eq:opt_ctrl2_a}
\ca^*(\xa,\xb,t) = -\vela \frac{p_u(\xa,\xb,t)}{\|p_u(\xa,\xb,t)\|_2}
\eq

\noindent where $p = (p_u,p_d) = \nabla_{\xj} \Phi(\xa,\xb)$.  

Similarly, if $\xjn\notin\mathcal{RA}_\infty(R,A)$, then the defender is guaranteed to win the game by using the optimal control input given in (\ref{eq:opt_ctrl_d}). Applying Equation (\ref{eq:opt_ctrl_d}) to the two-player game, we have that the defender winning strategy satisfies 

\bq \label{eq:opt_ctrl2_d_gen}
  \cb^*(\xa,\xb,t) = \arg \max_{\cb \in \mathbb{D}} p(\xa,\xb,t) \cdot f(\xa,\xb,\ca^*,\cb)
\eq

\noindent for $t\le 0$. The explicit winning strategy satisfying (\ref{eq:opt_ctrl2_d_gen}) is given in \cite{Huang2011} as
\bq \label{eq:opt_ctrl2_d}
\cb^*(\xa,\xb,t) = \velb \frac{p_d(\xa,\xb,t)}{\|p_d(\xa,\xb,t)\|_2}
\eq

% Solving the 4D HJI PDE (\ref{eq:HJ_PDE_reachavoid}) with $T\rightarrow \infty$ gives us $\mathcal{RA}_\infty(R,A)$, which characterizes the pair-wise outcome between an attacker-defender pair. In general, when the speed of every player on each team differs from those of all her teammates, solving $\N^2$ 4D HJI PDEs, each corresponding to a different pair of attacker speed and defender speed, gives us the pairwise outcomes between every attacker-defender pair. The computation time required is thus $C \N^2$, where $C$ is the time required to solve a single 4D HJI PDE. The pairwise outcomes can then be tied together to approximate the $\N$ vs. $\N$ game as described in Section \ref{sec:two_to_multi}. In the special case where each team has a single maximum speed, i.e. $\velai{i} = \vela, \velbi{i} = \velb, \forall i$, solving a \textit{single} 4D HJI PDE will characterize all pairwise outcomes.

% The solution to the 4D HJI PDE not only characterizes the outcome between an attacker-defender pair given their initial positions, but also characterizes the outcome between any joint-positions of that pair. When using maximum matching to determine how many attackers are guaranteed to not reach the target set, the HJI approach allows for real-time updates of the maximum matching. As the players move on the game domain $\amb$ to new positions, the pairwise outcome between $\pam{i},\pbm{j}$ can be determined by simply checking whether $(\xam{i}, \xbm{j})$ is in $\mathcal{RA}_\infty(R,A)$.

% After finding a maximum matching, we can determine whether the defending team can win as follows. After constructing the bipartite graph, if the maximum matching is of size $\mm$, then the defending team would be able to prevent \textit{at least} $\mm$ attackers from reaching the target. Alternatively, $\N-\mm$ is an upper bound on the number of attackers that can reach the target.

% For intuition, consider the following specific cases of $\mm$. If the size of the maximum matching $\mm=\N$, then no attacker will be able to reach the target. If $\mm=0$, then there is no initial pairing that will prevent any attacker from reaching the target; however, the attackers are not guaranteed to all reach the target, as $\N-\mm=\N$ is only an upper bound on the number of attackers who can reach the target. Finally, $\mm=\N-\m+1$, then the attacking team would only be able to send at most $\N-\mm=\m-1$ attackers to the target. 

% The optimal strategy for the defenders can be obtained from (\ref{eq:opt_ctrl2_d}). If the \ith defender $\pbm{i}$ is assigned to defend against the \jth attacker $\pam{j}$ by the maximum matching, then the strategy that guarantees that $\pam{j}$ never reaches the target satisfies (\ref{eq:opt_ctrl2_d_gen}): 

% \bq \label{eq:opt_ctrl3_d_gen}
  % \cbm{i}^*(\xam{j},\xbm{i},t) = \arg \max_{\cbm{i} \in \mathbb{D}} p(\xam{j},\xbm{i},-t)^T f(\xam{j},\xbm{i},\cam{j}^*,\cbm{i})
% \eq

% \noindent for $t\in [0,T]$, where $\cam{j}^*$ is given in (\ref{eq:opt_ctrl2_a}) as

% \bq \label{eq:opt_ctrl2_a}
% \cam{j}^*(\xam{j},\xbm{i},t) = -\vela \frac{p_u(\xam{j},\xbm{i},-t)}{\|p_u(\xam{j},\xbm{i},-t)\|_2}.
% \eq

% \noindent where $p = (p_u, p_d) = \frac{\partial \Phi}{\partial (\xam{j},\xbm{i})}$. The explicit strategy is then similar to (\ref{eq:opt_ctrl2_d}):
% \bq \label{eq:opt_ctrl3_d}
% \cb^*(\xam{j},\xbm{i},t) = \velb \frac{p_d(\xam{j},\xbm{i},-t)}{\|p_d(\xam{j},\xbm{i},-t)\|_2}
% \eq

% The entire procedure of applying maximum matching to the 4D HJ reachability calculation is illustrated in Figure \ref{fig:general_procedure}.

% Our solution to the multiplayer reach-avoid game is an approximation to the optimal solution that would be obtained by directly solving the $4\N$ dimensional HJI PDE obtained in Section \ref{subsec:hj_multi}; it is conservative for the defending team because by creating defender-attacker pairs, each defender restricts its attention to only one opposing player. For example, if no suitable matching is found, the defending team is not guaranteed to allow all attackers to reach the target, as the defending team could potentially capture some attackers without using a strategy that creates defender-attacker pairs. Nevertheless, our solution is able to overcome the numerical intractibility to approximate a reachability calculation, and is useful in many game configurations.

% \begin{figure}[h]
% \centering
% \includegraphics[width=0.5\textwidth]{"fig/general procedure"}
% \caption{An illustration of using 4D HJ reachability and maximum matching to solve the multiplayer reach-avoid game. A bipartite graph is created based on results of the 4D HJ reachability calculation. Then, a maximum matching of the bipartite graph is found to optimally assign defender-attacker pairs. A maximum matching of size $\mm$ indicates that at most $\N-\mm$ attackers will be able to rach the target.}
% \label{fig:general_procedure}
% \end{figure}

% \subsection{Time-Varying Defender-Attacker Pairings}
% \label{subsec:tvarp}
% The procedure outlined in Section \ref{subsec:max_match} assigns an attacker to each defender that is part of a maximum pairing in an open-loop manner: the assignment is done in the beginning of the game, and does not change during the course of the game. However, the bipartite graph and its corresponding maximum matching can be updated as the players change positions during the game. Because $\mathcal{RA}_\infty(R,A)$ captures the winning conditions for every joint defender-attacker configuration given $\amb,\obs,\target$, this update can be performed in real time by the following procedure:

% \begin{enumerate}
% \item Given the position of each defender $\xbm{i}$ and each attacker $\xam{j}$, determine whether $\xam{j}$ can win for all $j$. 
% \item Construct the bipartite graph and find its maximum matching to assign an attacker to each defender that is part of the maximum matching.
% \item For a short, chosen duration $\Delta$, compute the optimal control input and trajectory for each defender that is part of the maximum matching via Equation (\ref{eq:opt_ctrl2_d}). For the rest of the defenders and for all attackers, compute the trajectories assuming some control function.
% \item Repeat the procedure with the new player positions.
% \end{enumerate}

% As $\Delta\rightarrow 0$, the above procedure computes a bipartite graph and its maximum matching as a function of time. Whenever the maximum matching is not unique, the defenders can choose a different maximum matching and still be guaranteed to prevent the same number of attackers from reaching the target. As long as each defender uses the optimal control input given in Equation (\ref{eq:opt_ctrl2_d}), the size of the maximum matching can never decrease as a function of time. 

% On the other hand, it is possible for the size of the maximum matching to increase as a function of time. This occurs if the joint configuration of the players becomes such that the resulting bipartite graph has a bigger maximum matching than before, which may happen since the size of the maximum matching only gives an upper bound on the number of attackers that are able to reach the target. Furthermore, there is no numerically tractable way to compute the joint optimal control input for the attacking team, so a suboptimal strategy from the attacking team can be expected, making an increase of maximum matching size likely. Determining defender control strategies that optimally promote an increase in the size of the maximum matching would be an important step towards the investigation of cooperation, and will be part of our future work.

%% THEORY
\input{"solution PD game.tex"}
% Path defense game as special case of general 2-player reach-avoid
% Alternate way of solving path defense game through strong defense
	% lemma 1: winning regions for defender once defender is on the path
	% lemma 2: optimal point on the path for defender to go to
% Conservatism of strong path defense (insight beyond HJI)

\input{"solution PD reach-avoid.tex"} 
% Properties of strong path defense
	% Rotating paths backwards
% 2 player reach-avoid using path defense
	% Convex (solved)
	% Simply connected (strong path defense method)
		% only need to check paths touching target
	% Obstacles (strong path defense method)
		% only checking paths touching target may be even more conservative
% Conservatism of reach-avoid via strong path defense (defender winning regions)
	% Compare to 4D HJI solution

%%%%% MULTIPLAYER GAME
% Assume we have the solution to the two-player game
% Maximum matching ties together two-player game solutions
% !TEX root = multiplayer_reach_avoid_games.tex
\section{From Two-Player to Multiplayer} \label{sec:two_to_multi}
%Given the outcome of each attacker-defender pair obtained from either the HJI approach or the path defense approach, we construct an approximation to the $N$ vs. $N$ game using the graph-theoretic maximum matching. The approximation \textit{guarantees} an upper bound on the number of attackers that are able to reach the target. 

\subsection{Maximum Matching}
\label{subsec:max_match}
We piece together the outcomes of all attacker-defender pairs using maximum matching as follows:

\begin{alg}~
\begin{enumerate}
\item Construct a bipartite graph with two sets of nodes $\pas,\pbs$. Each node represents a player.
\item For all $i,j$, draw an edge between $\pbm{i}$ and $\pam{j}$ if $\pbm{i}$ wins against $\pam{j}$ in a two-player reach-avoid game.
\item Run any matching algorithm (eg. \cite{Schrjiver2004, Karpinski1998}) to find a maximum matching in the graph. %This can be done using, for example, a linear program \cite{Schrjiver2004}.
\end{enumerate}
\end{alg}

After finding a maximum matching, we can guarantee an upper bound on the number of attackers that is be able to reach the target. If the maximum matching is of size $\mm$, then the defending team would be able to prevent \textit{at least} $\mm$ attackers from reaching the target, and thus $\NA-\mm$ is an upper bound on the number of attackers that can reach the target. The maximum matching approach is illustrated in Fig. \ref{fig:general_procedure}.

\begin{figure}
\centering
\includegraphics[width=0.4\textwidth]{"fig/general_procedure"}
\caption{An illustration of using maximum matching to conservatively approximate the multiplayer reach-avoid game.}
\label{fig:general_procedure}
\end{figure}

\subsection{Time-Varying Defender-Attacker Pairings}
\label{subsec:tvarp}
With the next algorithm, the bipartite graph and its corresponding maximum matching can be updated, potentially in real time, as the players change positions during the game:

\begin{alg}~
\begin{enumerate}
\item Given each $\xbm{i}$ and each $\xam{j}$, determine whether $\pam{j}$ can win against $\pbm{i}$ for all $i,j$. \label{step:pairwise}
\item Assign a defender to each attacker that is part of a maximum matching.
\item For a short duration $\Delta$, apply a winning control input and compute the resulting trajectory for each defender that is part of the maximum matching. For the rest of the defenders and for all attackers, compute the trajectories assuming some (any) control function. \label{step:traj}
\item Update the player positions after the duration $\Delta$ and repeat steps \ref{step:pairwise} to \ref{step:traj}  with the new player positions.
\end{enumerate}
\end{alg}

As $\Delta\rightarrow 0$, the above procedure continuously computes a bipartite graph and its maximum matching. As long as each defender uses a winning control input against the paired-up attacker, the size of maximum matching can never decrease. 

%Since the size of the maximum matching only gives an upper bound on the number of attackers that are able to reach the target, it is possible for the size of the maximum matching to increase as a function of time. Also, because there is currently no numerically tractable way to compute the joint optimal control input for the attacking team, a suboptimal strategy from the attacking team can be expected, making an increase of maximum matching size likely. %Determining defender control strategies that optimally promote an increase in the size of the maximum matching would be an important step towards the investigation of cooperation. Since the maximum matching of a bipartate is in general not unique, such strategies may involve intelligently choosing among several maximum matchings or adding weights to edges and performing weighted maximum matching. These considerations will be part of our future work.

\subsection{Application to the Two-Player HJI Solution}
\label{subusec:MMHJI}
In general, solving $\NA\ND$ 4D HJI PDEs gives us the pairwise outcomes between every attacker-defender pair. The computation time required is thus $C \NA\ND$, where $C$ is the time required to solve a single 4D HJI PDE. The pairwise outcomes can then be merged together to approximate the $\NA$ vs. $\ND$ game. In the case where each team has a single maximum speed, solving \textit{one} 4D HJI PDE would characterize all pairwise outcomes.

Since the solution to the 4D HJI PDE characterizes pairwise outcomes based on any attacker-defender joint-state, it allows for real-time updates of the maximum matching. As players move to new positions, the pairwise outcome can be updated by simply checking whether $(\xam{i}, \xbm{j})$ is in $\mathcal{RA}_\infty(R,A)$.

\subsection{Application to the Two-Player Path Defense Solution}
\label{subsec:MMPD}
To use the pairwise outcomes determined by the path defense approach for approximating the solution to the multiplayer game, we add the following step to Algorithm \ref{alg:PD_RA}: 
\begin{enumerate}
\setcounter{enumi}{5}
\item Repeat steps \ref{step:dWinRegion} to \ref{step:union} for every attacker position.
\end{enumerate}

For a given domain, set of obstacles, and target set, steps \ref{step:createPath} and \ref{step:repeatCreatePath} in Algorithm \ref{alg:PD_RA} only need to be performed once, regardless of the number of players. In step \ref{step:dWinRegion}, the speeds of defenders come in only through a single distance calculation from $\pstar$, which only needs to be done once per attacker position. Therefore, the total computation time required is on the order of $C_1 + C_2 \NA$, where $C_1$ is the time required for steps \ref{step:createPath} and \ref{step:repeatCreatePath}, $C_2$ is the time required for steps \ref{step:dWinRegion} to \ref{step:union}. 

\subsection{Defender Cooperation}
One of the strengths of the maximum matching approach is its simplicity in the way cooperation among the defenders is incorporated from pairwise outcomes. More specifically, cooperation is incorporated using the knowledge of the strategy of each teammate, and the knowledge of which attackers each teammate can win against in a 1 vs. 1 setting. 

The knowledge of the strategy of each teammate is incorporated in the following way: When the pairwise outcomes for each defender is computed, a particular defender strategy used. The strategy of each defender is then used to compute pairwise outcomes, which are used in the maximum matching process. Each defender may use the optimal closed-loop strategy given by the two-player HJI solution, the semi-open-loop strategy given by the two-player path defense solution, or even another strategy that is not described in this paper. In fact, different defenders may use a different strategy.

As already mentioned, all of the information about the strategy of each defender is used to compute the pairwise outcomes. Since each pairwise outcome specifies a winning region for the corresponding defender, each defender can be guaranteed to win against a set of attackers in a one vs. one setting. The set of attackers against which each defender can win is then used to construct the bipartite graph on which maximum matching is performed. While executing the joint defense strategy as a team, each defender simply needs to execute its \textit{pairwise} defense strategy against the attacker to which the defender is assigned. 

The maximum matching process optimally combines the information about teammates' strategies and competence to derive a joint strategy to prevent as many attackers from reaching the target as possible. The size of the maximum matching then guarantees an upper bound on the number of attackers that can reach the target. To our knowledge, no other method can synthesize a joint defender control strategy that can provide such a guarantee in a multiplayer game. 

%% IMPLEMENTATION (Put algorithms in the previous sections)
% Finding the shortest path touching the target
% Choosing a path to defend
% Computing attacker winning regions given point
% Determining optimal point on a path to defend
% Conservatism (binary search on paths)

%\input{"comparison.tex"}

\section{Numerical Results}
\label{sec:simulation}
In the following subsections, we use a 4 vs. 4 example to illustrate our methods. The game is played on a square domain with obstacles. Defenders have a capture radius of $0.1$ units, and all players have the same maximum speed. We first show simulation results for the HJI and path defense approaches for determining pairwise outcomes, and then compare the two approaches. Computations were done on a Lenovo T420s laptop with a Core i7-2640M processor with 4 gigabytes of memory.

\subsection{HJI Formulation}
Figures \ref{fig:OL_pw_HJIA} to \ref{fig:OL_mm_HJI} show the results of solving the 4D HJI PDE in Equation (\ref{eq:HJ_PDE_reachavoid}) using the terminal set and avoid set in Equations (\ref{eq:4DHJI_terminal_set}) and (\ref{eq:4DHJI_avoid_set}), respectively. Computing the 4D reach-avoid set on a grid with 45 grid points in each dimension took approximately 30 minutes. In general, this makes the computation time required for computing $\N^2$ pair-wise outcomes $C \N^2 = 30 \N^2$ minutes. In our example, however, all players have the same maximum speed, so only a single 4D HJI PDE needed to be solved.

The solution of the 4D HJI PDE characterizes the 4D reach-avoid set $\mathcal{RA}_\infty(R,A)$. To visualize this set in 2D, we take 2D slices of the 4D reach-avoid set sliced at each player's position. Figure \ref{fig:OL_pw_HJIA} shows the 4D reach-avoid set sliced at various \textit{attacker positions}, which gives the all pairwise outcomes. For example, in the left top subplot, the attacker at $(-0.2, 0)$ is guaranteed to win a 1 vs. 1 reach-avoid game against each of the defenders at $(-0.3, 0.5)$ and at $(-0.3, -0.5)$, and guaranteed to lose against each of the defenders at $(0.3, 0.5)$ and at $(0.3, -0.5)$ if they play optimally.

Figure \ref{fig:OL_pw_HJID} shows the 4D reach-avoid set sliced at various \textit{defender positions}, and characterizes the same pairwise outcomes. For example,  in the right top subplot, the defender at $(-0.3, 0.5)$ is guaranteed to win a 1 vs. 1 reach-avoid game against each of the attackers at $(-0.5, 0)$ and at $(-0.2, 0.9)$, and guaranteed to lose against each of the attackers at $(0.7, -0.9)$ and at $(-0.2, 0)$ if they play optimally.

Figure \ref{fig:OL_mm_HJI} shows the bipartite graph and maximum matching resulting from the pairwise outcomes. In this case, the maximum matching is of size 4, a perfect matching. This guarantees that if each defender plays against the attacker matched by the maximum matching using the strategy in Equation (\ref{eq:opt_ctrl2_d}), then \textit{no} attacker will be able to reach the target.

\begin{figure}
	\centering
	\includegraphics[width=0.45\textwidth]{"fig/OLGame_pw_results_HJI_fixA"}
	\caption{The 4D reach-avoid set sliced at various attacker positions. In each subplot, defenders ``inside" the reach-avoid slice boundary win against the plotted attacker in a 1 vs. 1 setting using the control in Equation (\ref{eq:opt_ctrl2_d}); defenders ``outside" lose.}
	\label{fig:OL_pw_HJIA}
\end{figure}

\begin{figure}
	\centering
	\includegraphics[width=0.45\textwidth]{"fig/OLGame_pw_results_HJI_fixD"}
	\caption{The 4D reach-avoid set sliced at various defender positions. In each subplot, attackers ``inside" the reach-avoid slice boundary win against the plotted defender in a 1 vs. 1 setting using the control in Equation (\ref{eq:opt_ctrl2_a}); attackers ``outside" lose.}
	\label{fig:OL_pw_HJID}
\end{figure}

\begin{figure}
	\centering
	\includegraphics[width=0.4\textwidth]{"fig/OLGame_mm_results_HJI"}
	\caption{Tying together pair-wise interactions through maximum matching. Here, a maximum matching of size 4 (a perfect matching) guarantees that \textit{no} attacker will be able to reach the target without being captured if the defenders play optimally according to Equation (\ref{eq:opt_ctrl2_d}) against their matched attackers.}
	\label{fig:OL_mm_HJI}
\end{figure}

\subsection{Path Defense Formulation}
Figures \ref{fig:pd_ex} to \ref{fig:pd_mm} show the results of using the path defense approach to compute conservative approximations of the 4D reach-avoid set sliced at various attacker positions.

Figure \ref{fig:pd_ex} shows the defender winning region for a given attacker position (red cross) and path of defense (blue line). If the defender can get to $\pstar$ (small light green square) before the attacker can get to $\rpa \cup \rpb$, then the defender will be able to strongly defend the path, and prevent the attacker from reaching any target set that is enclosed by the path (large dark green square). The region within which the defender is able to do this is the region $\dr(\apa,\apb)$ (shown in light green). 

As an example, refer to the bottom left subplot in Figure \ref{fig:pd_ex}. Using the semi-open-loop strategy described in Section \ref{subsec:reach_avoid}, the defenders at $(-0.3, 0.5)$ and at $(0.3, 0.5)$ are able to prevent the attacker at $(-0.2, 0.9)$ from reaching the target.

Figure \ref{fig:pd_pw} shows the bipartite graph and maximum matching resulting from the pairwise outcomes. In this case, the maximum matching is of size 3. This guarantees that if each defender plays against the attacker matched by the maximum matching using the semi-open-loop strategy, then \textit{at most} 1 attacker will be able to reach the target.

Computations were done on a $200\times200$ grid, and 937 paths were used to compute the results in Figure \ref{fig:pd_pw}. Computation time varies with the number of paths we chose in steps \ref{step:createPath} and \ref{step:repeatCreatePath} in the algorithm in Section \ref{subsec:reach_avoid}. Taking the union of the defender winning regions from more paths will give a less conservative result, but require more computation time. A summary of the performance of our algorithm is shown in Figure \ref{fig:pd_perf}. 

With 937 paths, the computation of paths took approximately 60 seconds, and the computation of the 2D slice given the set of paths took approximately 30 seconds. However, very few paths are needed to approximate a 2D slice: Even with as few as 30 paths, the computed 2D slice covers more than 95\% of the area of the 2D slice computed using 937 paths. This reduces the computation time of the paths to 2.5 seconds, and the computation time of the 2D slices given the paths to 2.1 seconds. In terms of the complexity constants in Section \ref{subsec:reach_avoid}, we have that the computation time required for computing all pairwise interactions is $C_1 + C_2 \N = 2.5 + 2.1 \N$ seconds.

\begin{figure}
	\centering
	\includegraphics[width=0.35\textwidth]{"fig/PD Example"}
	\caption{Defense of a single path that encloses the target set.}
	\label{fig:pd_ex}
\end{figure}

\begin{figure}
	\centering
	\includegraphics[width=0.45\textwidth]{"fig/OLGame_pw_results_PD"}
	\caption{The reach-avoid set sliced at various attacker positions. In each subplot, defenders ``inside" the reach-avoid slice boundary are guaranteed to win against the plotted attacker in a 1 vs. 1 setting using the semi-open-loop control strategy described in Section \ref{subsec:reach_avoid}.}
	\label{fig:pd_pw}
\end{figure}

\begin{figure}
	\centering
	\includegraphics[width=0.4\textwidth]{"fig/OLGame_mm_results_PD"}
	\caption{Tying together pairwise interactions through maximum matching. Here, a maximum matching of size 3 guarantees that \textit{at most} 1 attacker will be able to reach the target without being captured if the defenders use the semi-open-loop control strategy described in Section \ref{subsec:reach_avoid} against their matched attackers.}
	\label{fig:pd_mm}
\end{figure}

\begin{figure}
	\centering
	\includegraphics[width=0.4\textwidth]{"fig/alg_perf_Rc"}
	\caption{Performance of the Path Defense Solution.}
	\label{fig:pd_perf}
\end{figure}

\subsection{Comparison Between HJI and Path Defense Formulations}
Each pairwise outcome computed with the HJI approach gives the optimal behavior assuming the players utilize a closed-loop strategy. In contrast, each pairwise outcome computed with the path defense approach assumes that the defender is using a semi-open-loop strategy as discussed in Section \ref{sec:comparison}. Figures \ref{fig:comp_ol} and \ref{fig:comp_ml} compare the 2D slices computed from the two different approaches.

Figure \ref{fig:comp_ol} shows the results for the 4 vs. 4 example in the preceding sections, where all players have the same maximum speed. Because the semi-open-loop strategy in the path defense approach is conservative towards the defender, the computed defender winning region is smaller in the path defense approach. The degree of conservatism depends on the domain, obstacles, target set, and attacker position. In this example, the defender winning regions computed using path defense approach on average covers approximately 75\% of the area of the defender winning regions computed using the HJI approach.

The path defense approach becomes more conservative when the attacker is slower than the defender because in this case, the attacker being outside the region $\rpa \cup \rpb$ is not a necessary but only a sufficient condition for the defender to win using the semi-open-loop strategy, as discussed in the Lemmas in Section \ref{sec:path_defense}. The path defense approach also becomes more conservative when there are large obstacles in the domain because we only considered paths that touch the target, which introduces additional conservatism in a non-simply-connected free space, as discussed in \ref{subsec:reach_avoid}. Figure \ref{fig:comp_ml} compares the 2D slices computed by the path defense approach and by the HJI approach in another 4 vs. 4 game where the attackers' maximum speed is 80\% of the defenders' maximum speed (players on each team have the same maximum speed), and where there is a larger obstacle in the domain. In this case,  the defender winning regions computed using path defense approach covers approximately 34\% of the area of the defender winning regions computed using the HJI approach.

\begin{figure}
	\centering
	\includegraphics[width=0.45\textwidth]{"fig/OLGame_compare"}
	\caption{Reach-avoid slices computed using the HJI approach and the path defense approach. Defenders and attackers have the same maximum speed.}
	\label{fig:comp_ol}
\end{figure}

\begin{figure}
	\centering
	\includegraphics[width=0.45\textwidth]{"fig/midLGame_compare"}
	\caption{Reach-avoid slices computed using the HJI approach and the path defense approach. Attackers' maximum speed is 80\% of that of the defenders.}
	\label{fig:comp_ml}
\end{figure}

\subsection{Real-Time Maximum Matching Updates}
After determining all pairwise outcomes either by $\N^2$ HJI PDEs in general or by solving a single 4D HJI PDE when all players on each team have the same maximum speed, pairwise outcomes of \textit{any} joint state of the attacker-defender pair are characterized. Thus, the bipartite graph corresponding to the pairwise outcomes can be updated simply by checking whether the joint state of the attacker-defender pair is inside the appropriate 4D reach-avoid set. This allows for updates of the bipartite graph and its maximum matching as the players play out the game in real time.

Figure \ref{fig:real_time_update} shows the maximum matching at several time snapshots of a 4 vs. 4 game. Each defender that is part of a maximum matching plays optimally against the paired-up attacker according to Equation (\ref{eq:opt_ctrl2_d}), and the remaining defender plays optimally against the closest attacker also according to Equation (\ref{eq:opt_ctrl2_d}). The attackers' strategy is to move towards the target along the shortest path while steering clear of the obstacles by $0.125$ units. The maximum matching is updated every $\Delta=0.005$ seconds. At $t=0$ and $t=0.2$, the maximum matching is of size 3, which guarantees that at most one attacker will be able to reach the target. After $t=0.4$, a perfect matching is found, which guarantees that no attacker will be able to reach the target.

\begin{figure}
	\centering
	\includegraphics[width=0.45\textwidth]{"fig/time varying graph"}
	\caption{An illustration of how the size of maximum matching can increase over time. Throughout the game, the defenders are updating the bipartite graph and maximum matching via the procedure described in Section \ref{subsec:tvarp}. Because the attacking team is not playing optimally, the defending team is able to find a perfect matching after $t=0.4$ (bottom plots) and prevent \textit{all} attackers from reaching the target.}
	\label{fig:real_time_update}
\end{figure}

% Convex (illustration of conservatism of strong path defense)
% Non-convex
	% Comparison with 4D HJI

% Conservatism of strong path defense
% Conservatism of reach-avoid via strong path defense (defender winning regions)

\section{Conclusions and Future Work} \label{sec:conc}
By applying maximum matching to the solution to the two player reach-avoid game, we were able to approximate the solution to the multiplayer reach-avoid game without significant additional computation overhead over the two player computation. By solving a single 4D HJI PDE, we obtained all possible pairings between the defenders and attackers. Then, a maximum matching algorithm determines the pairing that prevents the maximum number of attackers from reaching the target. This way, we were able to analyze the multiplayer reach-avoid game without directly solving the corresponding high dimensional, numerically intractable HJI PDE. Calculating time-varying defender-attacker pairings allows the defending team to potentially increase the size of the maximum matching over time.

An immediate extension of this work is to investigate defender strategies that optimally promote an increase in the size of maximum matching. This would be a step towards real-time defending team cooperation, in which each defender is not only responsible for preventing a particular assigned attacker from reaching the target, but also enabling other defenders to defend previously seemingly undefendable attackers. Many other extensions of this work naturally arise, including analyzing reach-avoid games with unfair teams with different numbers of players on each team, more complex player dynamics, or partial observability. 


% bibliography
\bibliographystyle{IEEEtran}
\bibliography{references}

\end{document}
