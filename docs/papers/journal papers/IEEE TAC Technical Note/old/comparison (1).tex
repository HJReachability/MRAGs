\section{Comparison Between the HJI and Path Defense Approaches}
\label{sec:comparison}
In Sections \ref{sec:solution_hji} and \ref{sec:path_defense}, we presented two approaches for determining pairwise outcomes in our multiplayer reach-avoid game. The pairwise outcomes determined by either method can then be tied together to approximate the solution to the full $\N$ vs. $\N$ game using maximum matching according to Section \ref{sec:two_to_multi}. We now briefly compare the two approaches.

The HJI approach holds an advantage over the path defense approach in terms of optimality. The HJI solution to the two-player reach-avoid game computes the entire 4D reach-avoid set and gives the optimal closed-loop control strategies for each attacker-defender pair; on the other hand, the path defense solution computes 2D slices of the 4D reach-avoid set sliced at the attacker position, and provides a more conservative semi-open-loop control strategy to the defenders. As a result, the defender winning regions computed in the path defense approach tend to be smaller than those computed in the HJI approach, as seen in Figures \ref{fig:comp_ol} and \ref{fig:comp_ml}. The degree of conservatism is more pronounced when the attacker is slower than the defender and when the domain contains large obstacles, as seen in Figure \ref{fig:comp_ol}. Furthermore, because the HJI approach computes entire 4D reach-avoid sets, very little additional computation overhead is required for updating pairwise outcomes as the players move during the course of the game.

The path defense approach trades off optimality for computation complexity. Because the HJI approach involves solving an HJI PDE in a 4D space, the computation time tends to be long and the memory requirements are high. Typically, a single 4D HJI PDE takes approximately half an hour to solve on a relative coarse grid in each dimension. Computation of a 2D slice using the path defense approach can be done on much finer grids with the same amount of memory, and takes less than a minute. Furthermore, when computing all pair-wise outcomes, the number of 2D slice computations required in path defense approach only scales with $\N$, the number of attackers, whereas the number of 4D HJI PDEs that needs to be solved is $\N^2$.
