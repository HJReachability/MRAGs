% !TEX root = multiplayer_reach_avoid_games.tex
\subsection{The Path Defense Solution to the Reach-Avoid Game}
\label{subsec:reach_avoid}
The central idea in using path defense is that if the target set is enclosed by some strongly defendable path for some $\apa,\apb$, then the defender can win the game using the semi-open-loop strategy outlined in this section, \textit{even if} the attacker uses the optimal control. Checking for strongly defendable paths adds more conservatism towards the defender, but makes computation much more efficient.

Naively, one could fix $\apa$, then search all other anchor points $\apb\in\boundary$ to find a defendable path. However, we can reduce the number of paths that needs to be checked by only checking paths of defense $\pathd$ that touch the target set. In a simply connected domain, this reduction in the number of paths checked does not introduce any additional conservatism. 

%To see this, observe that if a path is strongly defendable, any path ``behind" it on the defender's side is also strongly defendable. This is illustrated in Figure \ref{fig:path_behind}. Suppose $\pathds{i}$ is strongly defendable, then it is defendable. This necessarily implies that $\pathds{j}$ is defendable, because if the defender defends $\pathds{i}$, then the attacker can never cross $\pathds{i}$ and thus never cross $\pathds{j}$. Since $\pathds{i}$ is strongly defendable, the defender must be able to move to some point on $\pathds{i}$, and then defend $\pathds{i}$. However, on the way to $\pathds{i}$, the defender would have crossed $\pathds{j}$. So, the defender is able to arrive at some point on $\pathds{j}$ and then defend it afterwards; therefore $\pathds{j}$ is strongly defendable. Thus, if we restrict the paths considered to be only those that touch the target set, then picking $\apa$ determines $\apb$. 

%\begin{figure}
%\centering
%\includegraphics[width=0.4\textwidth]{"fig/path behind"}
%\caption{$\pathds{i}$ is strongly defendable if $\pathds{j}$ is strongly defendable.}
%\label{fig:path_behind}
%\end{figure}

If some strongly defendable path $\pathd$ encloses the target set, then the defender's strategy would be to first go to $\pstar\in\pathd$ (an open-loop strategy), then move towards $\xai(\apa)$ or $\xai(\apb)$ until the level set image is captured (a closed-loop strategy). Finally, the defender can simply track the captured level set image (a closed-loop strategy). This is a ``semi-open-loop" strategy. The following algorithm approximates a 2D slice conservatively towards the defender:

\begin{alg}~ \label{alg:PD_RA}
Given attacker position,
\begin{enumerate}
\item Choose some point $\apa\in\boundary$, which defines $\apb$ to create a path of defense $\pathd$ that touches the target $\target$. \label{step:createPath}
\item Repeat step \ref{step:createPath} for a desired set of points $\apa\in\boundary$. \label{step:repeatCreatePath}
\item For some particular $\pathd$, determine the defender winning region $\dr(\apa,\apb;\xan)$.\label{step:dWinRegion}
\item Repeat step \ref{step:dWinRegion} for all the paths created in steps \ref{step:createPath} and \ref{step:repeatCreatePath}.
\item The union $\bigcup_{\apa} \dr(\apa,\apb;\xan)$ gives the approximate 2D slice, representing the conservative winning region for the defender in the two-player reach-avoid game. \label{step:union}
\end{enumerate}
\end{alg}

