% !TEX root = multiplayer_reach_avoid_games.tex
\section{Numerical Results}
\label{sec:simulation}
We use a 4 vs. 4 example to illustrate our methods. The game is played on a square domain with obstacles. Defenders have a capture radius of $0.1$ units, and all players have the same maximum speed. Computations were done on a laptop with a Core i7-2640M processor with 4 gigabytes of memory.

\subsection{HJI Formulation}
Fig.  \ref{fig:comp_ol} shows the results of solving the corresponding 4D HJI PDE in blue. Computing the 4D reach-avoid set on a grid with 45 points in each dimension took approximately 30 minutes. All players have the same maximum speed, so only a single 4D HJI PDE needed to be solved. To visualize the 4D reach-avoid set, we take 2D slices of the 4D reach-avoid set sliced at each attacker's position. %For example, in the left top subplot, the attacker at $(-0.2, 0)$ is guaranteed to win a 1 vs. 1 reach-avoid game against each of the defenders at $(-0.3, 0.5)$ and at $(-0.3, -0.5)$, and guaranteed to lose against each of the defenders at $(0.3, 0.5)$ and at $(0.3, -0.5)$ if they play optimally.

% Figure \ref{fig:OL_pw_HJID} shows the 4D reach-avoid set sliced at various \textit{defender positions}, and characterizes the same pairwise outcomes. For example,  in the right top subplot, the defender at $(-0.3, 0.5)$ is guaranteed to win a 1 vs. 1 reach-avoid game against each of the attackers at $(-0.5, 0)$ and at $(-0.2, 0.9)$, and guaranteed to lose against each of the attackers at $(0.7, -0.9)$ and at $(-0.2, 0)$ if they play optimally.

Fig. \ref{fig:mm} shows the bipartite graph and maximum matching obtained from the pairwise outcomes. The maximum matching is of size 4. This guarantees that if each defender plays optimally against the attacker matched by the maximum matching, then \textit{no} attacker will be able to reach the target.

%\begin{figure}
%	\centering
%	\includegraphics[width=0.45\textwidth]{"fig/OLGame_pw_results_HJI_fixA"}
%	\caption{The 4D reach-avoid set sliced at various attacker positions. In each subplot, defenders ``inside" the reach-avoid slice boundary win against the plotted attacker in a 1 vs. 1 setting using the control in Equation (\ref{eq:opt_ctrl2_d}); defenders ``outside" lose.}
%	\label{fig:OL_pw_HJIA}
%\end{figure}
%
%\begin{figure}
%	\centering
%	\includegraphics[width=0.45\textwidth]{"fig/OLGame_pw_results_HJI_fixD"}
%	\caption{The 4D reach-avoid set sliced at various defender positions. In each subplot, attackers ``inside" the reach-avoid slice boundary win against the plotted defender in a 1 vs. 1 setting using the control in Equation (\ref{eq:opt_ctrl2_a}); attackers ``outside" lose.}
%	\label{fig:OL_pw_HJID}
%\end{figure}

%\begin{figure}
%	\centering
%	\includegraphics[width=0.4\textwidth]{"fig/OLGame_mm_results_HJI"}
%	\caption{Tying together pair-wise interactions through maximum matching. Here, a maximum matching of size 4 (a perfect matching) guarantees that \textit{no} attacker will be able to reach the target without being captured if the defenders play optimally according to Equation (\ref{eq:opt_ctrl2_d}) against their matched attackers.}
%	\label{fig:OL_mm_HJI}
%\end{figure}

\begin{figure}
	\centering
	\includegraphics[width=0.45\textwidth]{"fig/OLGame_compare"}
	\caption{Reach-avoid slices computed using the HJI approach and the path defense approach. }
	\label{fig:comp_ol}
\end{figure}

\subsection{Path Defense Formulation}
%Figure \ref{fig:pd_ex} shows the defender winning region for a given attacker position and path of defense. If the defender can get to $\pstar$ before the attacker can get to $\rpa \cup \rpb$, then the defender will be able to prevent the attacker from reaching any target set that is enclosed by the path. The region within which the defender is able to do this is the region $\dr(\apa,\apb)$. 

Fig. \ref{fig:comp_ol} shows, in red, the results of using the path defense approach to compute conservative approximations of the 4D reach-avoid set sliced at various attacker positions.

%As an example, refer to the bottom left subplot in Figure \ref{fig:pd_ex}. Using the semi-open-loop strategy described in Section \ref{subsec:reach_avoid}, the defenders at $(-0.3, 0.5)$ and at $(0.3, 0.5)$ are able to prevent the attacker at $(-0.2, 0.9)$ from reaching the target.

Fig. \ref{fig:mm} shows the bipartite graph and maximum matching resulting from the pairwise outcomes. In this case, the maximum matching is of size 3. This guarantees that if each defender plays against the attacker matched by the maximum matching using the semi-open-loop strategy, then \textit{at most} 1 attacker will be able to reach the target.

Computations were done on a $200\times200$ grid, and 937 paths were used to compute the results in Fig. \ref{fig:comp_ol}. Computation time varies with the number of paths we chose in steps \ref{step:createPath} and \ref{step:repeatCreatePath} in Algorithm \ref{alg:PD_RA}. Taking the union of the defender winning regions from more paths will give a less conservative result, but requires more computation time. A summary of the performance is shown in Fig. \ref{fig:pd_perf}. With 937 paths, the computation of paths took approximately 60 seconds, and the computation of the 2D slice given the set of paths took approximately 30 seconds. However, very few paths are needed to approximate a 2D slice: Even with as few as 30 paths, the computed 2D slice covers more than 95\% of the area of the 2D slice computed using 937 paths. This reduces the computation time of the paths to 2.5 seconds, and the computation time of the 2D slices given the paths to 2.1 seconds.

%\begin{figure}
%	\centering
%	\includegraphics[width=0.25\textwidth]{"fig/PD Example"}
%	\caption{Defense of a single path that encloses the target set.}
%	\label{fig:pd_ex}
%\end{figure}

%\begin{figure}
%	\centering
%	\includegraphics[width=0.45\textwidth]{"fig/OLGame_pw_results_PD"}
%	\caption{The reach-avoid set sliced at various attacker positions. In each subplot, defenders ``inside" the reach-avoid slice boundary are guaranteed to win against the plotted attacker in a 1 vs. 1 setting using the semi-open-loop control strategy described in Section \ref{subsec:reach_avoid}.}
%	\label{fig:pd_pw}
%\end{figure}

\begin{figure}
	\centering
	\includegraphics[width=0.5\textwidth]{"fig/OLGame_mm_results"}
	\caption{Merging pairwise outcomes with maximum matching.}
	\label{fig:mm}
\end{figure}

\begin{figure}
	\centering
	\includegraphics[width=0.35\textwidth]{"fig/alg_perf_Rc"}
	\caption{Performance of the Path Defense Solution.}
	\label{fig:pd_perf}
\end{figure}

%\subsection{Comparison Between HJI and Path Defense Formulations}
%Each pairwise outcome computed with the HJI approach gives the optimal behavior assuming the players utilize a closed-loop strategy. In contrast, each pairwise outcome computed with the path defense approach assumes that the defender is using a semi-open-loop strategy as discussed in Section \ref{sec:comparison}. Figures \ref{fig:comp_ol} and \ref{fig:comp_ml} compare the 2D slices computed from the two different approaches.
%
%Figure \ref{fig:comp_ol} shows the results for the 4 vs. 4 example in the preceding sections, where all players have the same maximum speed. Because the semi-open-loop strategy in the path defense approach is conservative towards the defender, the computed defender winning region is smaller in the path defense approach. The degree of conservatism depends on the domain, obstacles, target set, and attacker position. In this example, the defender winning regions computed using path defense approach on average covers approximately 75\% of the area of the defender winning regions computed using the HJI approach.

%The path defense approach becomes more conservative when the attacker is slower than the defender because in this case, the attacker being outside the region $\rpa \cup \rpb$ is not a necessary but only a sufficient condition for the defender to win using the semi-open-loop strategy, as discussed in the Lemmas in Section \ref{sec:path_defense}. The path defense approach also becomes more conservative when there are large obstacles in the domain because we only considered paths that touch the target, which introduces additional conservatism in a non-simply-connected free space, as discussed in \ref{subsec:reach_avoid}. Figure \ref{fig:comp_ml} compares the 2D slices computed by the path defense approach and by the HJI approach in another 4 vs. 4 game where the attackers' maximum speed is 80\% of the defenders' maximum speed (players on each team have the same maximum speed), and where there is a larger obstacle in the domain. In this case,  the defender winning regions computed using path defense approach covers approximately 34\% of the area of the defender winning regions computed using the HJI approach.



%\begin{figure}
%	\centering
%	\includegraphics[width=0.45\textwidth]{"fig/midLGame_compare"}
%	\caption{Reach-avoid slices computed using the HJI approach and the path defense approach. Attackers' maximum speed is 80\% of that of the defenders.}
%	\label{fig:comp_ml}
%\end{figure}

\subsection{Defender Cooperation}
To highlight the element of cooperation among the defenders, we examine Fig. \ref{fig:mm} more closely. For the maximum matching results from the two-player HJI solutions (left sub-figure), we see that the maximum matching creates the pairs $(\pbm{1}, \pam{1})$, $(\pbm{2}, \pam{4})$, $(\pbm{3}, \pam{2})$, and $(\pbm{4}, \pam{3})$. In general, such a pairing is not intuitively obvious. For example, it is not obvious, without knowledge of the pairwise outcomes, that $\pbm{1}$ can win against $\pam{1}$ in a one vs. one setting. If one were not sure whether $\pbm{1}$ can win against $\pam{1}$, then $(\pbm{2}, \pam{1})$ may seem like a reasonable pair. However, if $\pbm{2}$ defends against $\pam{1}$, then$\pbm{1}$ would defend against $\pam{2}$, leaving $\pbm{3}$ unable to find an attacker that $\pbm{3}$ can be guaranteed to win against to pair up with. The same observations can be made in many of the other pairings in both sub-figures.

For the maximum matching results from the two-player path defense solutions (right sub-figure), a semi-open-loop strategy is used. In this case, given the same initial conditions of the two teams, each defender may only be guaranteed to successfully defend against fewer attackers in a one vs. one setting compared to when using the optimal closed-loop strategy from the two-player HJI solution. Regardless of the strategy that each defender uses, the maximum matching process can be applied to obtain optimal defender-attacker pairs given the strategy used and the set of attackers each defender can be guaranteed to win against in a 1 vs. 1 setting.

\subsection{Real-Time Maximum Matching Updates}
After determining all pairwise outcomes, pairwise outcomes of \textit{any} joint state of the attacker-defender pair are characterized by the HJI approach. This allows for updates of the bipartite graph and its maximum matching as the players play out the game in real time. Fig. \ref{fig:real_time_update} shows the maximum matching at several time snapshots of a 4 vs. 4 game. Each defender that is part of a maximum matching plays optimally against the paired-up attacker, and the remaining defender plays optimally against the closest attacker. The attackers' strategy is to move towards the target along the shortest path while steering clear of the obstacles by $0.125$ units. The maximum matching is updated every $\Delta=0.005$ seconds. At $t=0$ and $t=0.2$, the maximum matching is of size 3, which guarantees that at most one attacker will be able to reach the target. After $t=0.4$, the maximum matching size increases to 4, which guarantees that no attacker will be able to reach the target.

\begin{figure}
	\centering
	\includegraphics[width=0.4\textwidth]{"fig/time varying graph"}
	\caption{Increasing maximum matching size over time.}
	\label{fig:real_time_update}
\end{figure}
