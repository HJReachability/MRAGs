\section{Conclusion and Future Work}
\label{sec:conclusion}
A general multiplayer reach-avoid game is numerically intractable to analyze by directly solving the corresponding high dimensional HJI PDE. To address this, we presented a way to tie together pairwise outcomes using maximum matching to approximate the solution to the full multiplayer game, guaranteeing an upper bound on the number of attackers that can reach the target. We also presented two approaches for determining the pairwise outcomes. The HJI approach is computationally more expensive, produces the optimal closed-loop control strategy for each attacker-defender pair, and efficiently allows for real time maximum updates. The path defense approach is conservative towards the defender, performs computation on the state space of a single player as opposed to the joint state space, and scales only linearly in the number of players in the general case where each player has a different maximum speed.

The methods presented in this paper illustrate the power of using special properties of shortest paths and tying together low dimensional information to approximate a high dimensional problem. Our future work will focus on applying these themes to both similar problems such as a reach-avoid game where players have more complex dynamics, and other related problems such as cooperative multi-agent collision avoidance.