\section{Numerical Results}
\label{sec:results}

\begin{figure*}[t]
	\centering
	\begin{tabular}{ccc}
%		\includegraphics[width=0.275\textwidth]{map} &
%		\includegraphics[width=0.275\textwidth]{stage1}  &
%		\includegraphics[width=0.275\textwidth]{stage2}  \\ 
	         \includegraphics[width=0.3\textwidth]{figures/ol_example_init.pdf} &
		\includegraphics[width=0.3\textwidth]{figures/ol_example_rstar.pdf}  &
		\includegraphics[width=0.3\textwidth]{figures/ol_example_paths.pdf}  \\ 
		(a) & (b) & (c)
	\end{tabular} 
	\caption{\textbf{Example scenario showing (a) the initial conditions of the players, the target set $\target$, and game domain, (b) the set $\Rs$ with contours plotted for $\ts$ within, and (c) the trajectories taken for each player, with equal time-to-reach contours for player 1 plotted. }}
	\label{fig:results_ol}
\end{figure*} 

We now show numerical results for an example scenario to illustrate the concepts presented in this paper and to demonstrate the algorithm presented in Section~\ref{sec:algorithm}.
Figure~\ref{fig:results_ol}(a) shows the initial condition for the scenario considered.
The gray, rectangular regions in the middle are the obstacles, and the green, semi-circular region to the right is the target set.
The initial positions of both players are plotted, along with the capture radius around $\pb$.
In this scenario, $\pb$ has a maximum speed of $0.25$ and $\pa$ has a maximum speed of $1$.
Figure~\ref{fig:results_ol}(b) shows $\Rs$ and contour plots for $\ts$ within $\Rs$. 
Obviously $\ts$ increases as $\pb$ moves toward the opening in the obstacles, forcing $\pa$ to move up around the obstacles in order to reach the target.
The final paths of the two players are shown in Figure~\ref{fig:results_ol}(c): as expected $\pb$ moves to block the opening, as that results in the maximum $\ts$.

The same scenario computed using the upper value solution discussed in~\cite{OL_ICRA2012} is shown in Figure~\ref{fig:results_vu}.
In this case, $\pa$ is playing conservatively.
$\mathcal{S}_1$ denotes the set of points that $\pa$ can safely reach before $\pb$, given that $\pb$ has knowledge of $\pa$'s inputs.
Again, $\pa$ is forced to move up over the obstacles as opposed to passing through the central opening.
The control inputs in the two cases are the identical, implying that the solutions found are equivalent to the optimal, feedback strategy results.

The computations are performed on a 100x100 grid using compiled C++ code in Matlab, on a Macbook Pro laptop with a 2.4 GHz Intel Core i7 processor and 8 GB RAM.
Total elapsed time for computing $\Rs$ and associated $\ts$ was 1.25 seconds.

\begin{figure}[t]
	\centering
		\includegraphics[width=0.3\textwidth]{figures/ol_example_vupath.pdf}  
	\caption{\textbf{Solution to the upper value problem, showing the safe reachable set $\mathcal{S}_1$ for $\pa$ and the resultant path to the target set. }}
	\label{fig:results_vu}
\end{figure} 
