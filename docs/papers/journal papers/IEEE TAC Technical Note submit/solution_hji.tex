% !TEX root = multiplayer_reach_avoid_games.tex
\section{The HJI Solution of the 1 vs. 1 Game} \label{sec:solution_hji}
The HJI approach for solving differential games is outlined in \cite{Huang2011,j:mitchell-TAC-2005, LSToolbox}. The optimal joint closed-loop control strategies for the attacker and the defender in a two-player reach-avoid game can be obtained by solving a 4D HJI PDE. This solution allows us to determine whether the defender will win against the attacker in a 1 vs. 1 setting. 

In the two-player game, the attacker aims to reach $\target$ while avoiding $\avoid$. Both players also avoid $\obs$. In particular, the defender wins if the attacker is in $\obs$, and vice versa. Therefore, we define the terminal set and avoid set to be

\begin{equation} \label{eq:4DHJI_sets}
\begin{aligned}
R &= \left\{\xj\in\amb^2 \mid \xa\in\target \right\} \cup \left\{\xj\in\amb^2\mid \xb\in\obs \right\} \\
A &= \avoid \cup \left\{\xj\in\amb^2\mid \xa\in\obs \right\}
\end{aligned} 
\end{equation}

Given \eqref{eq:4DHJI_sets}, we can define the corresponding implicit surface functions $\valsR_R,\valsR_A$ required for solving the HJI PDE. Since $\amb\subset\R^2$, the result is $\mathcal{RA}_\infty(R,A)\in\R^4$, a 4D reach-avoid set. If $\xjn\in\mathcal{RA}_\infty(R,A)$, then the attacker is guaranteed to win the game by using the optimal control \textit{even if} the defender is also using the optimal control; if $\xjn\notin\mathcal{RA}_\infty(R,A)$, then the defender is guaranteed to win the game by using the optimal control \textit{even if} the attacker is also using the optimal control.