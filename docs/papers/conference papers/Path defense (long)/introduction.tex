\section{Introduction}
\label{sec:intro}
Differential games have been studied extensively, and are powerful theoretical tools in robotics, aircraft control, security, and other domains \cite{OFTHEAIRFORCEWASHINGTON:2009p37, Erzberger:2006p44, kiva2009}. The multiplayer reach-avoid game (to be defined precisely in Section \ref{sec:formulation}) is a differential game between two adversarial teams of cooperating players, where one team attempts to reach a certain target quickly while the other team aims to delay, or if possible, prevent the opposing team from reaching the target. One example of a reach-avoid game is the popular game Capture-the-Flag (CTF) \cite{HThesis, Huang2011}. In robotics and automation, CTF has been explored most notably in the Cornell Roboflag competition, where two opposing teams of mobile robots are directed by human players to play the game \cite{DAndrea:2003p95}. A number of results related to motion planning and human-robot interactions have been reported from the competition \cite{Earl:2007p101, Campbell:2003p5, Waydo:2003p97, Parasuraman:2005p99}. 

A multiplayer reach-avoid game is a complex game due to both the conflicting goals of the two teams and the cooperation among the players within each team, rendering the optimal solution for each team nontrivial to obtain and visualize. Previous work \cite{HThesis, Huang2011} has shown that even in a 1 vs. 1 scenario, human agents are sometimes unable to find the optimal way to play, losing in situations in which an optimal winning strategy exists. For general multiplayer reach-avoid games, optimal solutions are extremely difficult to compute due to the intrinsic high dimensionality of the joint state space. Multiplayer differential games have been previously addressed using various techniques. In \cite{Earl:2007p101}, where a team of defenders assumes that the attackers move towards their target in straight lines, a mixed-integer linear programming approach was used. In \cite{Chasparis:2005p102}, optimal defender strategies are determined using a linear program, with the assumption that the attackers use a linear feedback control law. In complex pursuit-evasion games where players may change roles over time, nonlinear model-predictive control \cite{Sprinkle:2004p100} and approximate dynamic programming \cite{McGrew:2008p103} approaches have been investigated. In both cases, opponent strategies are estimated based on explicit prediction models.

The above-mentioned techniques tend to work well in the situations in which accurate models of the opponent team can be obtained. While those techniques have proven to be effective in their corresponding scenarios, they cannot be easily adapted to solve a general multiplayer reach-avoid game when no prior information on each side is known. The ideal framework to use for such a general multiplayer reach-avoid game is the classical Hamilton-Jacobi-Isaacs (HJI) approach \cite{b:isaacs-1967}, in which an HJI partial differential equation (PDE) is solved to obtain optimal strategies for both teams. In this case, if both teams play optimally, the result of the game is completely determined by the joint initial condition of the players. In addition, modern numerical tools \cite{j:mitchell-TAC-2005, Sethian1996, b:osher-fedkiw-2002} have been developed to carry out the computations and leverage the power of the HJI framework when the dimensionality of the problem is low. These numerical tools have been employed to successfully solve a variety of differential games, path planning problems, and optimal control problems, including aircraft collision avoidance \cite{j:mitchell-TAC-2005}, automated in-flight refueling \cite{DSST08}, and two-player reach-avoid games \cite{Huang2011}. These tools offer tremendous flexibility in terms of the player dynamics and terrain, and do not explicitly assume any specific control strategy or prediction models for the players. A notable success \cite{j:mitchell-TAC-2005} has been witnessed in applying such tools to solve low dimensionality problems.

Despite the power that the HJI framework and the numerical tools have to offer, solving a general multiplayer reach-avoid game is computationally intractable due to the curse of dimensionality: the joint state space of two $\N$-player teams in a two-dimensional domain is $4N$-dimensional. Numerically, when the state space is discretized, the number of nodes scales exponentially with the number of dimensions. Therefore, computing optimal solutions for a general multiplayer reach-avoid game in the HJI framework is out of reach. As a result, the inherent trade-off between optimality of the solutions and computational complexity must be considered and made. Our solution in this paper is no exception to this rule. 

In \cite{Chen2014}, we considered each defender-attacker pair and compute the optimal solutions for both players using the HJI framework. We then invoke the graph-theoretical maximum matching algorithm  \cite{Schrjiver2004, Karpinski1998} to determine optimal pairings. This procedure approximates the HJI solution to the multiplayer game by combining the solutions from the $\N^2$ two-player games between each attacker-defender pair. This procedure reduces the computation complexity from solving a $4N$-dimensional HJI PDE to solving $\N^2$ 4-dimensional HJI PDEs. This is a substantial complexity reduction; however, without solving each 4-dimensional HJI PDE is still very time- and memory-intensive. This is the motivation of this paper.

Our contributions can mainly be stated as follows. First, it provides a computationally efficient approximation to the solution proposed in \cite{Huang2011} and \cite{Chen2014} on the two-player reach-avoid game. Having computed this approximiation, we the same extension of the two-player game to the multiplayer game is easily implemented at almost no additional computational cost compared to the two-player games. Hence, our approach provides an appealing solution, especially when the number of players becomes large. In addition, some cooperation is taken into account by the maximum matching process. Second, we quantify the degree of conservatism in our solution using the 4-dimensional HJI calculations in \cite{Chen2014} as a bench mark. 

