\section{Introduction}
\label{sec:intro}
Differential games have been studied extensively, and are powerful theoretical tools in robotics, aircraft control, security, and other domains \cite{OFTHEAIRFORCEWASHINGTON:2009p37, Erzberger:2006p44, kiva2009}. The multiplayer reach-avoid game (to be defined precisely in Section \ref{sec:formulation}) is a differential game between two adversarial teams of cooperating players, where one team attempts to reach a certain target quickly while the other team aims to prevent the opposing team from reaching the target. One example of a reach-avoid game is the popular game Capture-the-Flag (CTF) \cite{HThesis, Huang2011}. In robotics and automation, CTF has been explored most notably in the Cornell Roboflag competition, where two opposing teams of mobile robots are directed by human players to play the game \cite{DAndrea:2003p95}. Several results related to motion planning and human-robot interactions have been reported from the competition \cite{Earl:2007p101, Campbell:2003p5, Waydo:2003p97, Parasuraman:2005p99}. 

A multiplayer reach-avoid game is a complex game due to both the conflicting goals of the two teams and the cooperation among the players within each team, rendering the optimal solution for each team nontrivial to obtain and visualize. In \cite{HThesis, Huang2011}, the authors show that even in a 1 vs. 1 scenario capture the flag game, human players sometimes lose in situations in which an optimal winning strategy exists. General multiplayer reach-avoid games are even more difficult to intuit, and computing optimal solutions becomes computationally intractable due to the intrinsic high dimensionality of the joint state space. Multiplayer differential games have been previously addressed using various techniques. In \cite{Earl:2007p101}, where a team of defenders assumes that the attackers move towards their target in straight lines, a mixed-integer linear programming approach is used. In \cite{Chasparis:2005p102}, optimal defender strategies are determined using a linear program, with the assumption that the attackers use a linear feedback control law. In complex pursuit-evasion games where players may change roles over time, nonlinear model-predictive control \cite{Sprinkle:2004p100} and approximate dynamic programming \cite{McGrew:2008p103} approaches have been investigated. In both cases, opponent strategies are estimated based on explicit prediction models.

One framework for a general multiplayer reach-avoid game is the classical Hamilton-Jacobi-Isaacs (HJI) approach \cite{b:isaacs-1967}, in which an HJI partial differential equation (PDE) is solved to obtain optimal strategies for both teams. Modern numerical tools such as \cite{j:mitchell-TAC-2005, Sethian1996, b:osher-fedkiw-2002} are able to solve the HJI PDE when the dimensionality of the problem is low. Despite the power that the HJI framework and the numerical tools have to offer, solving a general multiplayer reach-avoid game is computationally intractable due to the high dimensionality of the game. Numerically, when the state space is discretized, the number of grid points scales exponentially with the number of dimensions. As a result, the inherent trade-off between optimality of the solutions and computational complexity must be considered. 

\cite{Chen2014} considers defender-attacker pairs and computed the optimal solutions for each pair using the HJI framework. The pairwise interactions are then tied together by using the graph-theoretical maximum matching algorithm \cite{Schrjiver2004, Karpinski1998} to determine optimal pairings. This procedure approximates the HJI solution to the multiplayer game by combining the solutions from the $\N^2$ two-player games, and reduces the computation complexity from solving a $4N$D HJI PDE to solving $\N^2$ 4-dimensional (4D) HJI PDEs. This is a substantial complexity reduction; however, solving 4D HJI PDEs is still very time- and memory-intensive, especially when the number of PDEs that need to be solved is $\N^2$. To address this issue, we present the path defense solution, which approximates the solution to the 4D HJI PDEs using a series of two-dimensional (2D) calculations. Our new approach is conservative towards the defender, and provide sufficient conditions for the defender to win.

Our contributions can be stated as follows. First, our path defense approach provides a computationally efficient approximation to the solution proposed in \cite{Huang2011} and \cite{Chen2014} for the two-player reach-avoid game. Having computed this approximation, the same extension of the two-player game to the multiplayer game as in \cite{Chen2014} can be easily implemented at almost no additional computational cost. Furthermore, we will demonstrate that the computation complexity of the path defense approach will scale with $\N$ as opposed to $\N^2$. Hence, our approach provides an appealing solution, especially when the number of players becomes large. In addition, some cooperation is taken into account by the maximum matching process. 

