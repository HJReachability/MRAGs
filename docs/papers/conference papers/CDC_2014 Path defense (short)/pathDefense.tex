%%%%%%%%%%%%%%%%%%%%%%%%%%%%%%%%%%%%%%%%%%%%%%%%%%%%%%%%%%%%%%%%%%%%%%%%%%%%%%%%
%2345678901234567890123456789012345678901234567890123456789012345678901234567890
%        1         2         3         4         5         6         7         8

\documentclass[letterpaper, 10 pt, conference]{ieeeconf}  % Comment this line out
                                                          % if you need a4paper
%\documentclass[a4paper, 10pt, conference]{ieeeconf}      % Use this line for a4
                                                          % paper

\IEEEoverridecommandlockouts                              % This command is only
                                                          % needed if you want to
                                                          % use the \thanks command
\overrideIEEEmargins
% See the \addtolength command later in the file to balance the column lengths
% on the last page of the document

\usepackage{graphicx}
\usepackage{amsmath} % assumes amsmath package installed
\usepackage{amssymb}  % assumes amsmath package installed
\usepackage{amsfonts}
\usepackage{graphicx}
\usepackage{algorithm, algorithmic}
\usepackage{subcaption}
\numberwithin{algorithm}{section}

\newtheorem{defn}{Definition}
\newtheorem{thm}{Proposition}[section]
\newtheorem{cor}{Corollary}[section]
\newtheorem{rem}{Remark}
\newtheorem{lem}{Lemma}
\newtheorem{IEEEproof}{Proof of Lemma}
\usepackage{"my_macros"}

\title{\LARGE \bf
A Path Defense Approach to the Multiplayer Reach-Avoid Game
}

\author{Mo~Chen, Zhengyuan~Zhou, and Claire~J.~Tomlin% <-this % stops a space
\thanks{This work has been supported in part by NSF under CPS:ActionWebs (CNS-931843), by ONR under the HUNT (N0014-08-0696) and SMARTS (N00014-09-1-1051) MURIs and by grant N00014-12-1-0609, by AFOSR under the CHASE MURI (FA9550-10-1-0567).}% <-this % stops a space
\thanks{M.~Chen, and C.~J.~Tomlin are with the Department of Electrical Engineering and Computer Sciences,
        University of California, Berkeley, CA 94720, USA
        {\tt\small \{mochen72,tomlin\}@eecs.berkeley.edu}}
\thanks{Z.~Zhou is with the Department of Electrical Engineering,
        Stanford University, Stanford, CA 94305, USA
        {\tt\small zyzhou@stanford.edu}}   
}

\begin{document}

\maketitle

\thispagestyle{empty}
\pagestyle{empty}

\begin{abstract}
We consider a multiplayer reach-avoid game played between $\N$ attackers and $\N$ defenders moving with simple dynamics on a general two-dimensional domain. The attackers attempt to win the game by sending at least $\m$ of them ($1\le\m\le\N$) to a target location while the defenders try to prevent the attackers from doing so by capturing them. The analysis of this game plays an important role in collision avoidance, motion planning, and aircraft control, among other applications involving cooperative agents. The high dimensionality of the game makes computing an optimal solution for either side intractable when $\N>1$. The solution is difficult even when $\N=1$. To address this issue, we present an efficient, approximate solution to the 1 vs. 1 problem. We call the approximate solution the ``path defense solution", which is conservative towards the defenders. This serves as a building block for an approximate solution of the multiplayer game. Compared to the classical Hamilton-Jacobi-Isaacs approach for solving the 1 vs. 1 game, our new method is orders of magnitude faster, and scales much better with the number of players.
\end{abstract}

\section{Introduction}
\label{sec:intro}
Differential games have been studied extensively, and are powerful theoretical tools in robotics, aircraft control, security, and other domains \cite{OFTHEAIRFORCEWASHINGTON:2009p37, Erzberger:2006p44, kiva2009}. The multiplayer reach-avoid game (to be defined precisely in Section \ref{sec:formulation}) is a differential game between two adversarial teams of cooperating players, where one team attempts to reach a certain target quickly while the other team aims to prevent the opposing team from reaching the target. One example of a reach-avoid game is the popular game Capture-the-Flag (CTF) \cite{HThesis, Huang2011}. In robotics and automation, CTF has been explored most notably in the Cornell Roboflag competition, where two opposing teams of mobile robots are directed by human players to play the game \cite{DAndrea:2003p95}. Several results related to motion planning and human-robot interactions have been reported from the competition \cite{Earl:2007p101, Campbell:2003p5, Waydo:2003p97, Parasuraman:2005p99}. 

A multiplayer reach-avoid game is a complex game due to both the conflicting goals of the two teams and the cooperation among the players within each team, rendering the optimal solution for each team nontrivial to obtain and visualize. In \cite{HThesis, Huang2011}, the authors show that even in a 1 vs. 1 scenario capture the flag game, human players sometimes lose in situations in which an optimal winning strategy exists. General multiplayer reach-avoid games are even more difficult to intuit, and computing optimal solutions becomes computationally intractable due to the intrinsic high dimensionality of the joint state space. Multiplayer differential games have been previously addressed using various techniques. In \cite{Earl:2007p101}, where a team of defenders assumes that the attackers move towards their target in straight lines, a mixed-integer linear programming approach is used. In \cite{Chasparis:2005p102}, optimal defender strategies are determined using a linear program, with the assumption that the attackers use a linear feedback control law. In complex pursuit-evasion games where players may change roles over time, nonlinear model-predictive control \cite{Sprinkle:2004p100} and approximate dynamic programming \cite{McGrew:2008p103} approaches have been investigated. In both cases, opponent strategies are estimated based on explicit prediction models.

One framework for a general multiplayer reach-avoid game is the classical Hamilton-Jacobi-Isaacs (HJI) approach \cite{b:isaacs-1967}, in which an HJI partial differential equation (PDE) is solved to obtain optimal strategies for both teams. Modern numerical tools such as \cite{j:mitchell-TAC-2005, Sethian1996, b:osher-fedkiw-2002} are able to solve the HJI PDE when the dimensionality of the problem is low. Despite the power that the HJI framework and the numerical tools have to offer, solving a general multiplayer reach-avoid game is computationally intractable due to the high dimensionality of the game. Numerically, when the state space is discretized, the number of grid points scales exponentially with the number of dimensions. As a result, the inherent trade-off between optimality of the solutions and computational complexity must be considered. 

\cite{Chen2014} considers defender-attacker pairs and computed the optimal solutions for each pair using the HJI framework. The pairwise interactions are then tied together by using the graph-theoretical maximum matching algorithm \cite{Schrjiver2004, Karpinski1998} to determine optimal pairings. This procedure approximates the HJI solution to the multiplayer game by combining the solutions from the $\N^2$ two-player games, and reduces the computation complexity from solving a $4N$D HJI PDE to solving $\N^2$ 4-dimensional (4D) HJI PDEs. This is a substantial complexity reduction; however, solving 4D HJI PDEs is still very time- and memory-intensive, especially when the number of PDEs that need to be solved is $\N^2$. To address this issue, we present the path defense solution, which approximates the solution to the 4D HJI PDEs using a series of two-dimensional (2D) calculations. Our new approach is conservative towards the defender, and provide sufficient conditions for the defender to win.

Our contributions can be stated as follows. First, our path defense approach provides a computationally efficient approximation to the solution proposed in \cite{Huang2011} and \cite{Chen2014} for the two-player reach-avoid game. Having computed this approximation, the same extension of the two-player game to the multiplayer game as in \cite{Chen2014} can be easily implemented at almost no additional computational cost. Furthermore, we will demonstrate that the computation complexity of the path defense approach will scale with $\N$ as opposed to $\N^2$. Hence, our approach provides an appealing solution, especially when the number of players becomes large. In addition, some cooperation is taken into account by the maximum matching process. 


% !TEX root = multiplayer_reach_avoid_games.tex
\section{The Reach-Avoid Problem}
\subsection{The Multiplayer Reach-Avoid Game}
\label{sec:formulation}
Consider $\NA+\ND$ players partitioned into the set of $\NA$ attackers, $\pas = \{\pam{1}, \pam{2}, \ldots, \pam{\NA}\}$ and the set of $\ND$ defenders, $\pbs = \{\pbm{1}, \ldots, \pbm{\ND}\}$, whose states are confined in a bounded, open domain $\amb \subset \R^2$. The domain $\amb$ is partitioned into $\amb$ = $\free \cup \obs$, where $\free$ is a compact set representing the free space in which the players can move, while $\obs = \amb \setminus \free$ corresponds to obstacles in the domain. 

Let $\xam{i}, \xbm{j} \in \R^2$ denote the state of players $\pam{i}$ and $\pbm{j}$, respectively. Then given initial conditions $\xanm{i}\in \free,i=1,2,\ldots,\NA,\xbnm{i}\in \free,i=1,2,\ldots,\ND$, we assume the dynamics of the players to be defined by the following decoupled system for $t \geq 0$:

\bq\label{eq:dynamics}
\begin{aligned}
\dotxam{i}(t) &= \velai{i}\cam{i}(t), & \xam{i}(0) = \xanm{i}, i=1,2,\ldots,\NA \\
\dotxbm{i}(t) &= \velbi{i}\cbm{i}(t), & \xbm{i}(0) = \xbnm{i}, i=1,2,\ldots,\ND
\end{aligned}
\eq
where $\velai{i}, \velbi{i}$ denote maximum speeds for $\pam{i}$ and $\pbm{i}$ respectively, and $\cam{i},\cbm{i}$ denote controls of $\pam{i}$ and $\pbm{i}$ respectively. We assume that $\cam{i},\cbm{i}$ are drawn from the set $\A = \{\sigma \colon [0,\infty)\rightarrow \unitball \mid \sigma \text{ is measurable}\}$, where $\unitball$ denotes the closed unit disk in $\R^2$. We also constrain the players to remain within $\free$ for all time. Denote the joint state of all players by $\xj = (\xja, \xjb)$ where $\xja =(\xam{1},\ldots\xam{\NA})$ is the attacker joint state $\pas$, and $\xjb = (\xbm{1},\ldots,\xbm{\ND})$ is the defender joint state $\pbs$. 

The attacking team wins whenever $\m$ of the $\NA$ attackers reach some target set without being captured by the defenders; $\m$ is pre-specified with $0<M\le \NA$. The target set is denoted $\target\subset\free$ and is compact. The defending team wins if it can prevent the attacking team from winning by capturing or indefinitely delaying $\NA-\m+1$ attackers from reaching $\target$. An illustration of the game setup is shown in Fig. \ref{fig:mp_form}.

We will define the capture sets, $\avoid_{ij}$, to be $\avoid_{ij} = \left\{\xj\in\amb^{\NA+\ND} \mid \|\xam{i}-\xbm{j}\|_2\le\Rc \right\}$. $\pam{i}$ is captured by $\pbm{j}$ if $\pam{i}$'s position is within a distance $\Rc$ of $\pbm{j}$'s position. 

\begin{figure}
\centering
\includegraphics[width=0.35\textwidth]{"fig/formulation"}
\caption{The components of a multiplayer reach-avoid game.}
\label{fig:mp_form}
\end{figure}

In this paper, we address the following problems:
\begin{enumerate}
\item Given $\xjn$, $\target$, and some fixed integer $\m, 0<\m\le\NA$, can the attacking team win?
\item More generally, given $\xjn$ and $\target$, how many attackers can the defending team prevent from reaching the target?
\end{enumerate}

\subsection{The Two-Player Reach-Avoid Game}
\label{sec:2p_ra}
We will answer the above questions about the general $\NA$ vs. $\ND$ reach-avoid game by using the solution to the two-player $1$ vs. $1$ game as a building block. In the two-player game, we denote the attacker $\pa$, the defender $\pb$, their states $\xa,\xb$, and their initial conditions $\xan,\xbn$. Their dynamics are
\bq
\begin{aligned}
\dotxa(t) &= \vela\ca(t), & \xa(0) = \xan,\\
\dotxb(t) &= \velb\cb(t), & \xb(0) = \xbn
\end{aligned}
\eq

The players' joint state becomes $\xj=(\xa,\xb)$, and their joint initial condition becomes $\xjn=(\xan,\xbn)$. The capture set becomes simply $\avoid = \left\{(\xa,\xb)\in\amb^2 \mid \|\xa-\xb\|_2\leq \Rc\right\}$. 

$\pa$ wins if it reaches the target $\target$ without being captured by $\pb$. $\pb$ wins if it can prevent $\pa$ from winning by capturing $\pa$ or indefinitely delaying $\pa$ from reaching $\target$. For the two-player reach-avoid game, we seek to answer the following:
\begin{enumerate}
\item Given $\xjn$ and $\target$, is the defender guaranteed to win? \label{p:tp1}
\item More generally, given $\xa$ and $\target$, what is the set of initial positions such that the defender is guaranteed to win? \label{p:tp2}
\end{enumerate}
\input{"solution HJI.tex"}
\input{"solution PD game.tex"}
\input{"solution PD reach-avoid.tex"}
\input{"solution matching.tex"}
%\section{Solution} \label{sec:solution}

\section{Computational Results}
\label{sec:results}
We illustrate our path defense and maximum matching approach in the example below. The 2D slices of the 4D reach-avoid sets are calculated using our path defense approach. The example is shown in Figures \ref{fig:pd_ex}, \ref{fig:fixed_a_1}, and \ref{fig:max_matching_1}. In this example, there are four attackers and four defenders playing on a square domain with obstacles. All players have equal speeds. 

Figure \ref{fig:pd_ex} illustrates the defense of a single path $\pathd$. The regions $\rpa,\rpb$ induced by $\pstar$ are shown in red. $\pstar$ is calculated based on the position of the attacker, shown as a red cross. Any defender within the region $\dr(\apa,\apb)$, shown in green, will be able to defend $\pathd$, and therefore defend the target set. Taking the union of all the defender winning regions for many paths, we obtain an approximation of the 2D slices shown in Figure \ref{fig:fixed_a_1}.

Each subplot in Figure \ref{fig:fixed_a_1} shows the boundary of 2D slices for each fixed attacker position. Defenders, shown as blue stars, that are inside the boundary win against the particular attacker in each subplot. For example, in the top left subplot of Figure \ref{fig:fixed_a_1}, the defender at $(0.3, 0.5)$ wins against the attacker at $(-0.2, 0)$, while the other three defenders lose against the attacker at $(-0.2, 0)$.

Figure \ref{fig:max_matching_1} shows the resulting bipartite graphs (thin solid blue lines) and the maximum matching (thick dashed blue lines) after applying the algorithm described in Section \ref{sec:max_match}. The maximum matching is of size 3, which means that at most 1 attacker is able to reach the target. Therefore, if $\m>1$ attackers getting to the target is required for the attackers to win, then the defenders are guaranteed to win. 

%As the players play out the game and reach a new joint-configuration, the bipartite graph and the maximum matching can be recomputed to obtain new optimal pairings for the defending team.

\begin{figure}[H]
	\centering
	\includegraphics[width=0.375\textwidth]{"fig/PD example"}
	\caption{Defense of a single path that encloses the target set.}
	\label{fig:pd_ex}
\end{figure}

\begin{figure}[H]
	\centering
	\includegraphics[width=0.4\textwidth]{"fig/fixed attacker PD 1"}
	\caption{2D slices of reach-avoid sets at the positions of the four attackers. Defenders  inside the 2D slice boundary are guaranteed to win against the attacker in each subplot.}
	\label{fig:fixed_a_1}
\end{figure}

\begin{figure}[H]
	\centering
	\includegraphics[width=0.375\textwidth]{"fig/max matching PD 1"}
	\caption{Bipartite graph and maximum matching. Each edge (thin solid blue line) connects a defender to an attacker against whom the defender is guaranteed to win, creating a bipartite graph. Dashed thick blue lines show a maximum matching. Here, a maximum matching of size 3 indicates that at most one attacker can reach the target.}
	\label{fig:max_matching_1}
\end{figure}

Computations were done on a $200\times200$ grid, a grid resolution that was not possible when solving 4D HJI PDEs \cite{Chen2014}. 937 paths were used to compute the results in Figure \ref{fig:fixed_a_1}. Computation time varies with the number of paths we chose in steps \ref{step:createPath} and \ref{step:repeatCreatePath} in the algorithm in Section \ref{sec:reach_avoid}. Computations were done on a Lenovo T420s laptop with a Core i7-2640M processor.  %A summary of the performance of our algorithm is shown in Figure \ref{fig:alg_perf_1}. 

Figure \ref{fig:alg_perf_1}(top) shows the area of the 2D slices as a function of the number of paths used for the computation. The areas are expressed as an average (over the four attacker positions) fraction of the 2D slices computed using 937 paths. Even with as few as 59 paths, the computed 2D slice covers more than 95\% of the area of the 2D slice computed using 937 paths.

Figures \ref{fig:alg_perf_1}(middle) and \ref{fig:alg_perf_1}(bottom) show the paths computation and 2D slice computation times, denoted $C_1$ and $C_2$ respectively in Section \ref{sec:reach_avoid}. If one uses 59 paths to compute the 2D slices, we would have $C_1=3.2$ seconds, and $C_2=2.1$ seconds, with a resulting overall complexity of $3.2 + 2.1 \N$ \textit{seconds}. This is vastly superior to solving 4D HJI PDEs, whose complexity is $C_3\N^2$ with $C_3\approx30$ \textit{minutes} \cite{Chen2014}.

\begin{figure}[H]
	\centering
	\includegraphics[width=0.325\textwidth]{"fig/alg_perf_1"}
	\caption{Performance of the Path Defense Solution}
	\label{fig:alg_perf_1}
\end{figure}

\section{Conclusions and Future Work} \label{sec:conc}
We have presented an efficient method, based on the idea of path defense, for approximating 2D slices of 4D reach-avoid sets in the two player reach-avoid game. By leveraging the properties of shortest paths between points on the domain boundary, we were able to compute these slices using a series of 2D distance calculations, which can be done very quickly using FMM. Our method is conservative towards the defender, but two to three orders of magnitude faster than the previous HJI reachability method \cite{Chen2014}.

By using maximum matching tie together two-player game solutions, we were also able to approximate the solution to the multiplayer reach-avoid game without significant additional computation overhead. A maximum matching algorithm determines the defender-attacker pairing that prevents the maximum number of attackers from reaching the target. This way, we were able to guarantee an upper bound on the number of attackers that are able to get to the target without directly solving the corresponding high dimensional, numerically intractable HJI PDE. 

%An immediate extension of this work is to investigate defender strategies that optimally promote an increase in maximum matching size. Many other extensions of this work naturally arise, including analyzing reach-avoid games with unfair teams with different numbers of players on each team, more complex player dynamics, or partial observability. 

\bibliographystyle{IEEEtran}
\bibliography{references}

\end{document}
