\section{Hamilton-Jacobi-Isaacs Reachability} \label{sec:hj_background}
The multiplayer reach-avoid game is a differential game in which two teams have competing objectives \cite{b:basar-olsder-1999}. By specifying the winning conditions, we can determine the winning regions of the attacker and the defender by numerically computing the HJI reach-avoid sets. These computations allow us to consider games with arbitrary terrain, domain, obstacles, target set, and player velocities. Furthermore, the results of HJI computation assume a closed-loop strategy for both players given previous information of the other players. The setup for using HJI reachability to solve differential games can be found in \cite{Huang2011, j:mitchell-TAC-2005, Chen2014, LSToolbox}. %In summary, we are given the continuous dynamics of the system state:
%
%\bq
%\dxj = f(\xj,u,d), \xj(0)=\xjn,
%\eq
%
%\noindent where $\xj\in\R^n$ is the system state, $u\in\mathbb{U}$ and $d\in\mathbb{D}$ are the joint control inputs of the attackers and defenders, respectively. The sets $\mathbb{U}$ and $\mathbb{D}$ represent the sets of the joint admissible control inputs of the attackers and the defenders, respectively. The attacking team selects a control input based on the past and the current joint states of all the players. The defending team then selects a control input based on the past and the present control inputs of the attacking team, in addition to
%the past and the current joint states. \textit{A priori}, this information pattern is conservative towards the attackers, as defenders have more information available. However, in the case that the system (described by the function $f$) is decoupled, which is true in our reach-avoid game defined in Equation~\eqref{eq:dynamics}, the Isaacs condition \cite{b:isaacs-1967} holds and the two information patterns yield the same optimal solutions for both the attackers and the defenders.

To use the HJI reachability framework, we specify the terminal set as the attackers' winning condition, and propagate backwards this set subject to the constraint imposing that the attackers be outside the capture regions and the obstacles. This constraint is described by the avoid set. The result is a reach-avoid set that partitions the joint state space into a region that represents the joint initial conditions from which the attacking team is guaranteed to win, and a region that represents the joint initial conditions from which the defending team is guaranteed to win.

The HJI reachability framework involves solving the HJI PDE on a grid in the joint state space of all players. The full $\N$ vs. $\N$ game would involve gridding up a $4\N$D state space, making the HJI PDE infeasible to solve numerically. \cite{Chen2014} showed that maximum matching can piece together pairwise interactions between players, so that instead of solving the 4$\N$D HJI PDE, one could instead solve $\N^2$ 4D HJI PDEs that describe all the pairwise interactions between the $\N$ players on each team. %In the following sections, we will present the path defense solution to the two-player game. Our new method approximates 2D slices of the 4D reach-avoid set obtained from solving the HJI PDE, and involves a series of 2D distance calculations which can be done very efficiently by solving the 2D Eikonal equation using the fast marching method (FMM) \cite{Sethian1996}.
