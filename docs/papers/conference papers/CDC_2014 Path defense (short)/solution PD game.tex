\section{The Path Defense Solution to the Two-Player Game}
\label{sec:path_defense}
Solving a 4D HJI PDE involves discretizing a 4D space, and is thus time- and memory-intensive. Instead of computing the entire 4D reach-avoid set given by the HJI solution, we will approximate 2D slices of the reach-avoid set (or simply ``2D slices"). These slices will be the 4D reach-avoid set sliced at desired attacker positions.  

In this section, we first introduce the \textit{Path Defense Game} (a specific type of reach-avoid game) and then describe an efficient way to approximately solve it. The solution to the path defense game will be a building block for approximating 2D slices of the general two-player reach-avoid game, which will be presented in the Section \ref{sec:reach_avoid}.The computation involved in the methods presented in this section and in Section \ref{sec:reach_avoid} can be done very efficiently by solving the 2D Eikonal equation using the fast marching method (FMM) \cite{Sethian1996}.

\subsection{The Path Defense Game}
The \textit{Path Defense Game} is a two-player reach-avoid game in which the boundary of the target set is the shortest path between two particular points on $\boundary$, and the target set is on one side of the shortest path. We denote the target set as $\target=\sac$ for two given points on the boundary $\apa,\apb$. $\sa,\pathd,\apa,\apb$ are defined below. 

% Definitions
\begin{defn} % Shortest path between two points
\textbf{Path of defense}. Denote the shortest path between points $\x,\y$ to be $\path(\x,\y)$. The length of $\path(\x,\y)$ is denoted $\dist(\x,\y)$, and the time it takes for the attacker and defender to traverse $\path(\x,\y)$ is denoted $\ta(\x, \y),\tb(\x,\y)$, respectively. A path of defense, $\pathd$, is the shortest path between points $\apa$ and $\apb$ located on $\boundary$. $\apa$ and $\apb$ are referred to as the \textbf{anchor points} of path $\pathd$. 
\end{defn}

\begin{defn} % Attacker's side of the path (needed?)
\textbf{Attacker's side of the path}. A path of defense $\pathd$ partitions the domain $\amb$ into two regions. Define $\sa$ to be the region that contains the attacker, not including points on the path $\pathd$. The attacker seeks to reach the target set $\target=\sac$.
\end{defn}

The basic setup of the path defense game is illustrated in Figure \ref{fig:pd_form}.
\begin{figure}[H]
\centering
\includegraphics[width=0.25\textwidth]{"fig/path defense game"}
\caption{An illustration of the components of a path defense game between two players.}
\label{fig:pd_form}
\end{figure}

\subsection{Solving the path defense game}
A path defense game can be directly solved by computing a 4D reach-avoid set with $\target=\sac$, which, as noted earlier, is time- and memory- intensive. Hence, we propose an efficient way to approximate a 2D slice, which is conservative towards the defender.

% Definitions
%\begin{defn} % Defendable path
%\textbf{Defendable path}. Given initial conditions $\xjn=(\xan,\xbn)$, a path $\pathd$ is defendable if regardless of the attacker's actions, the defender has a control function $\cb(\cdot)$ to prevent the attacker from reaching $\pathd$ without being captured.
%\end{defn}

\begin{defn} % Strongly defendable path
\textbf{Strongly defendable path}. Given initial conditions $\xjn=(\xan,\xbn)$, a path $\pathd$ is strongly defendable if regardless of the attacker's actions, the defender has a control function $\cb(\cdot)$ to reach $\pathd$ after finite time and prevent the attacker from reaching $\pathd$ without being captured. %Note that the defender is not explicitly required to stay on $\pathd$ after reaching it.
\end{defn}

%\begin{rem}
%A path $\pathd$ is defendable if it is strongly defendable.
%\end{rem}

%Checking whether a path $\pathd$ is defendable is exactly the path defense problem. 
Since solving the path defense problem involves a 4D reach-avoid set calculation, we instead consider the problem of checking whether a path $\pathd$ is strongly defendable. The following definitions lead to our first Lemma which describes how to determine strong defendability using 2D distance calculations; the definitions and Lemma are illustrated in Figure \ref{fig:lemma1}.

\begin{defn} % Level set image of attacker
\textbf{Level set image of attacker}. Given attacker position $\xa(t)$, define the level set image of the attacker with respect to anchor point $\apa$ to be $\xai(t;\apa) = \{\x\in\pathd: \ta(\x,\apa) = \ta(\xa(t),\apa)\}$. $\xai$ is unique. Define $\xai(t;\apb)$ similarly by replacing $\apa$ with $\apb$. For convenience, we will sometimes omit the time argument and write $\xai(\apa)$.
\end{defn}

\begin{rem}
\label{rem:image_of_a}
$\dist(\xai(\apb),\apa) \le \dist(\xai(\apa),\apa)$. 

\textit{Proof}:
First note that
\begin{equation*}
\begin{aligned}
\dist(\apa,\apb) &\le \dist(\xa,\apa) + \dist(\xa,\apb)\\
& \text{(Definition of shortest path between $\apa$ and $\apb$)} \\
& = \dist(\xai(\apa),\apa) + \dist(\xai(\apb),\apb) \\
& \text{(Definition of level set image)}
\end{aligned}
\end{equation*}

Then, since the left hand side is given by $\dist(\apa,\apb) = \dist(\apa,\xai(\apb))+\dist(\xai(\apb),\apb)$, the result follows.
\end{rem}

\begin{rem}
Given some path of defense $\pathd$, if the defender's position coincides with the level set image of the attacker, i.e. $\xb(s) = \xai(s;\apa)$ (or $\xai(s;\apb)$) at some time $s$, then there exists a control for the defender to always remain on the level set image of the attacker thereafter, i.e. $\xb(t) = \xai(t;\apa)$ (or $\xai(t;\apb)$) $\forall t\ge s$. This is because the attacker's image can move at most as fast as the attacker, whose speed is at most the defender's speed by assumption.
\end{rem}

\begin{defn} % Winning regions for defender given path and point
\label{def:d_win_region}
\textbf{Regions induced by point $\ppath$ on path}. Given a point $\ppath\in\pathd$, define a region $\rpa\left(\ppath\right)$ associated with the point $\ppath$ and anchor point $\apa$ as follows:
\bq
\rpa\left(\ppath\right) = \left\{\x: \dist(\x,\apa) \leq \dist(\ppath,\apa) \right\}
\eq
Define $\rpb(\ppath)$ similarly by replacing $\apa$ with $\apb$.
\end{defn}

% Lemma involving winning regions for defender
Now we are ready for our first Lemma, which gives sufficient conditions for strong path defense.
\begin{lem}
\label{lem:d_winning_region}
Suppose that the defender is on some point $\ppath$ on the path $\pathd$, i.e. $\xbn = \ppath$. Furthermore, assume that $\vela=\velb$. Then, $\pathd$ is strongly defendable if and only if the attacker's initial position $\xan$ is outside the regions induced by $\ppath$: $\xan\in\amb\backslash\left(\rpa \cup \rpb\right)$.
\end{lem}

\begin{IEEEproof} % This assumes speed of attacker is same as (or less than) speed of defender 
We assume $\xan \notin \target = \sac$, otherwise the attacker would start inside the target set. The proof is illustrated in Figure \ref{fig:lemma1}.

First, we show that if $\xan\in \rpa \cup \rpb$, then the attacker can reach $\apa$ or $\apb$ and hence $\sac$ without being captured. 

Without loss of generality, suppose $\xan\in\rpa$. To capture the attacker, the defender must necessarily be on $\xai(\apa)$ or $\xai(\apb)$ at some time $t$. By definition \ref{def:d_win_region}, we have $\dist(\xan,\apa) < \dist(\ppath,\apa)$, so $\ta(\xai(\apa),\apa) < \tb(\ppath,\apa)$. By remark \ref{rem:image_of_a}, we also have $\dist(\xai(\apb),\apa) \le \dist(\xai(\apa),\apa)$, so it suffices to show that the defender never reaches $\xai(\apa)$ before the attacker reaches $\apa$. 

If the attacker moves towards $\apa$ along $\path(\xan,\apa)$ with maximum speed, then $\xai(\apa)$ will move towards $\apa$ along $\path(\xai(\apa),\apa)$ at the same speed. Since $\ta(\xa,\apa)=\ta(\xai(\apa),\apa)<\tb(\ppath,\apa)$, $\xai(\apa)$ will reach $\apa$ before $\xb$ does. When $\xai(\apa)=\apa$, we also have $\xa=\apa\in\sac$. Thus, the defender never captures the attacker's level set image. Therefore, no matter what the defender does, the attacker can reach $\sac$ by moving towards $\apa$ at maximum speed along $\path(\xan,\apa)$.

Next, we show, by contradiction, that if $\xa\notin\rpa \cup \rpb$, then the attacker cannot reach $\sac$ without being captured. 

Suppose $\pa$ will reach some point $\pprime$ before $\pb$ does, i.e. $\dist(\xan,\pprime)<\dist(\xbn,\pprime)=\dist(\ppath,\pprime)$. Without loss of generality, assume $\pprime\in\path(\ppath,\apb)$, and note that $\dist(\ppath,\apb)<\dist(\xan,\apb)$ since the attacker is not in $\rpb$. Starting with the definition of the shortest path, we have
\bq
\begin{aligned}
\dist(\xan,\apb) &\le \dist(\xan,\pprime) + \dist(\pprime,\apb) \\
& \text{(definition of shortest path / triangle inequality)} \\
&< \dist(\ppath,\pprime) + \dist(\pprime,\apb) \\
& \text{(by assumption, $\dist(\xan,\pprime)<\dist(\ppath,\pprime)$)}\\ 
&= \dist(\ppath,\apb) \\
&< \dist(\xan, \apb) \text{ (since $\xan\notin\rpa$)}
\end{aligned}
\eq

This is a contradiction. Therefore, the attacker cannot cross any point $\pprime$ on $\pathd$ without being captured. This proves Lemma \ref{lem:d_winning_region}. 
\end{IEEEproof}

\begin{figure}[h]
\centering
	\begin{subfigure}{0.24\textwidth}
	\centering
	\includegraphics[width=\textwidth]{"fig/attacker in Ra"}
	\caption{\label{subfig:ainra}}
	\end{subfigure}~
	\begin{subfigure}{0.24\textwidth}
	\centering
	\includegraphics[width=\textwidth]{"fig/attacker outside Ra U Rb"}
	\caption{\label{subfig:aoutra}}
	\end{subfigure}
\caption{(\ref{subfig:ainra}) Suppose the attacker is in $\rpa\cup\rpb$. If the attacker moves towards $e_a$, the defender cannot capture the attacker's level set image before the attacker reaches $\apa$. (\ref{subfig:aoutra}) Suppose the attacker is not $\rpa\cup\rpb$, there is no point $\pprime$ on $\pathd$ that the attacker can reach without being captured.}
\label{fig:lemma1}
\end{figure}

Given $\xbn=\ppath\in\pathd$, Lemma \ref{lem:d_winning_region} partitions $\amb$ into two regions, assuming $\vela=\velb$: if the attacker is initially in $\rpa \cup \rpb$, then $\pathd$ is not strongly defendable; otherwise, it is strongly defendable. In the case that $\vela<\velb$, the attacker being outside of $\rpa \cup \rpb$ becomes a sufficient condition (not necessary) for the strong defendability of $\pathd$.

In general, $\xbn$ may not be on $\pathd$. In this case, to strongly defend $\pathd$, the defender needs to first arrive at some point $\ppath\in\pathd$. If there exists some $\ppath\in\pathd$ such that $\tb(\xbn, \ppath)\le\ta\left(\xan,\rpa(\ppath)\cup\rpb(\ppath)\right)$, then $\pathd$ is strongly defendable. %If the defender can arrive at $\ppath$ before the attacker moves into $\rpa(\ppath)\cup\rpb(\ppath)$, then $\ppath$ is strongly defendable. 

%Thus, given $\xjn = (\xan,\xbn)$, we can check whether a path $\pathd$ is strongly defendable as follows:
%
%For all points on the path $\ppath\in\pathd$,
%\begin{enumerate}
%\item Compute $\tb(\xbn, \ppath)$.
%\item Compute $\ta(\xan),\rpa(\ppath)\cup\rpb(\ppath)$.
%\end{enumerate}


The above procedure requires two distance computations for every point on the path $\pathd$. The following lemma shows that it is necessary and sufficient to perform the computations for only one special point denoted $\pstar$.

\begin{rem} \label{rem:time_to_region_a}
Given $\ppath\in\pathd$, $\dist\left(\xan,\rpa(\ppath)\right) = \dist(\xan,\apa) - \dist(\ppath,\apa)$. Similarly, $\dist\left(\xan,\rpb(\ppath)\right) = \dist(\xan,\apb) - \dist(\ppath,\apb)$.
\end{rem}

\begin{lem} \label{lem:pstar}
Let a point $\pstar$ on the path $\pathd$ be such that $\ta(\xan,\rpa)=\ta(\xan,\rpb)$. Then, $\pathd$ is strongly defendable if and only if the defender can defend $\pathd$ by first going to $\pstar$.
\end{lem}

\begin{IEEEproof}
One direction is clear: if the defender can defend $\pathd$ by first going to $\pstar$, then $\pathd$ is strongly defendable by definition.

We will show the other direction by showing its contrapositive: if the defender cannot defend $\pathd$ by first going to $\pstar$, then $\pathd$ is not strongly defendable. Equivalently, we will show that if choosing $\pstar$ as the first point of entry does not allow the defender to defend $\pathd$, then no other choice of $\ppath$ as the first point of entry does.

Suppose that the defender cannot defend $\pathd$ by choosing $\pstar$ as the first point of entry, but can defend $\pathd$ by choosing another point $\pprime$ as the first point of entry. Without loss of generality, assume $\dist(\pstar,\apa)-\dist(\pprime,\apa)>0$. This assumption moves $\pprime$ further away from $\apa$ than $\pstar$, causing $\rpa$ to move closer to $\xan$. Starting with Remark \ref{rem:time_to_region_a}, we have

\bq
\begin{aligned}
\dist\left(\xan,\rpa(\pstar)\right) & = \dist(\xan,\apa) - \dist(\pstar,\apa) \\
\ta\left(\xan,\rpa(\pstar)\right) & = \ta(\xan,\apa) - \ta(\pstar,\apa) 
\end{aligned}
\eq

Similarly, for the point $\pprime$, we have
\bq
\begin{aligned}
\dist\left(\xan,\rpa(\pprime)\right) & = \dist(\xan,\apa) - \dist(\pprime,\apa) \\
\ta\left(\xan,\rpa(\pprime)\right) & = \ta(\xan,\apa) - \ta(\pprime,\apa) 
\end{aligned}
\eq

Then, subtracting the above two equations, we derive that the attacker will be able to get to $\rpa$ sooner by the following amount of time:
\bq
\begin{aligned}
\ta\left(\xan,\rpa(\pstar)\right) - \ta\left(\xan,\rpa(\pprime)\right) &= \ta(\pprime,\apa) - \ta(\pstar,\apa) \\
&= \ta(\pprime,\pstar) \\
& \ge \tb(\pprime,\pstar)
\end{aligned}
\eq

We now show that the defender can get to $\pprime$ sooner than to $\pstar$ by less than the amount $\tb(\pprime,\pstar)$, and therefore the defender in effect ``gains less time" than the attacker does by going to $\pprime$. We assume that $\pprime$ is closer to the defender than $\pstar$ is. Then, by the triangle inequality, we have

\bq
\begin{aligned}
\dist(\xbn,\pstar) & \leq \dist(\xbn,\pprime) + \dist(\pprime,\pstar) \\
\dist(\xbn,\pstar) - \dist(\xbn,\pprime) & \leq  \dist(\pprime,\pstar) \\
\tb(\xbn,\pstar) - \tb(\xbn,\pprime) & \leq  \tb(\pprime,\pstar)
\end{aligned}
\eq
This completes the proof.
\end{IEEEproof}

Lemmas \ref{lem:d_winning_region} and \ref{lem:pstar} give a simple algorithm to compute, given $\xan$, the region that the defender must be in for a path of defense $\pathd$ to be strongly defendable:
\begin{enumerate}
\item Given $\apa, \apb,\xan$, compute $\pstar$ and $\rpa(\pstar),\rpb(\pstar)$.
\item $\pathd$ is strongly defendable if $\tb(\xbn, \pstar)\le\ta\left(\xan,\rpa(\pstar)\cup\rpb(\pstar)\right)$. Therefore, if $\xbn\in\dr(\apa,\apb;\xan)=\{x:\tb(\x,\pstar) \le \ta(\xan,\rpa \cup \rpb)\}$, then $\pathd$ is strongly defendable.
\end{enumerate}

The computations in this algorithm can be efficiently computed by using FMM \cite{Sethian1996} on a 2D grid. Thus, we have conservatively solved the path defense problem, orginally a 4D problem, using a series of 2D fast marching calculations. 

Figure \ref{fig:lemma2} illustrates the proof of Lemma \ref{lem:pstar} and $\dr$.
\begin{figure}[h]
\centering
	\begin{subfigure}{0.24\textwidth}
	\includegraphics[width=\textwidth]{"fig/best point pstar"}
	\caption{\label{subfig:pstar}}
	\end{subfigure}~
	\begin{subfigure}{0.24\textwidth}
	\includegraphics[width=\textwidth]{"fig/defender winning pd"}
	\caption{\label{subfig:dregion}}
	\end{subfigure}
	\caption{(\ref{subfig:pstar}) If the defender cannot defend $\pathd$ by first going to $\pstar$, then the defender cannot defend $\pathd$ by going to any other point $\ppath$. (\ref{subfig:dregion}) Lemma \ref{lem:pstar} allows an easy computation of $\rpd$.}
	\label{fig:lemma2}
\end{figure}


% \subsection{Conservatism of strong path defense \label{subsec:pd_cons}}
% To quantify how conservative strong path defense is, we consider the following question: what paths of defense are defendable, but not strongly defendable? One example of such a path is the Voronoi line between an attacker and a defender with equal speeds in a convex domain. This is shown in figure \ref{fig:voronoi_pod}.

% \begin{figure}[h]
% \centering

% \begin{subfigure}{0.45\textwidth}
% \includegraphics[width=\textwidth]{"fig/defend voronoi"}
% \caption{The defender can defend the initial Voronoi line by mirroring the attacker's movement.}
% \label{subfig:vor_pod1}
% \end{subfigure}

% \begin{subfigure}{0.45\textwidth}
% \includegraphics[width=\textwidth]{"fig/voronoi strong defense"}
% \caption{If the defender tries to arrive at some point $\ppath$, and then strongly defend it, then the attacker will arrive at one of the anchor points before the defender.}
% \label{subfig:vor_pod2}
% \end{subfigure}

% \caption{The Voronoi line is a path of defense that is defendable but not strongly defendable.}
% \label{fig:voronoi_pod}
% \end{figure}

% To quantify the conservatism of strong path defense compared to path defense, we compute the defendable region and the strongly defendable region, defined below.

% First, observe that if a path is strongly defendable, any path ``behind" it on the defender's side is also strongly defendable. This is illustrated in figure \ref{fig:path_behind}. Suppose $\pathds{i}$ is strongly defendable, then it is defendable. This necessarily implies that $\pathds{j}$ is defendable, because if the defender defends $\pathds{i}$, then the attacker can never cross $\pathds{i}$ and thus never cross $\pathds{j}$. Since $\pathds{i}$ is strongly defendable, the defender can move to some point on $\pathds{i}$, and then defend $\pathds{i}$. However, on the way to $\pathds{i}$, the defender would have crossed $\pathds{j}$. Now, we have that the defender arrives at some point on $\pathds{j}$ and can defend it afterwards; therefore $\pathds{j}$ is strongly defendable.

% \begin{figure}[h]
% \includegraphics[width=0.45\textwidth]{"fig/path behind"}
% \caption{$\pathds{i}$ is strongly defendable if $\pathds{j}$ is strongly defendable.}
% \label{fig:path_behind}
% \end{figure}

% \begin{defn}
% Given joint initial condition $\xn$ and a (strongly) defendable path $\pathd$, call the region $\sac$ the \textbf{(strongly) defendable region given $\pathd$}. Define the union of all (strongly) defendable regions given paths to be the \textbf{(strongly) defendable region}, denoted ($\sdr$) $\dr$:
% \bq
% \begin{aligned}
% \dr &= \bigcup_{\pathd \text{ defendable}} \sac\\
% \sdr &= \bigcup_{\pathd \text{ strongly defendable}} \sac
% \end{aligned}
% \eq
% \end{defn}

% For a given joint initial condition $\xn$, we express the conservatism of strong path defense as the ratio $\text{area}(\dr) / \text{area}(\sdr)$. 

