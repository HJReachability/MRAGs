\section{Solution} \label{sec:solution}
In this section, we first describe the HJ reachability framework for solving differential games with arbitrary terrain, domain, obstacles, target set, and player velocities based on \cite{j:mitchell-TAC-2005, LSToolbox, Huang2011}. In particular, we will show how to compute the optimal joint control strategies for the attackers and the defenders by solving a 4$\N$ dimensional HJI PDE. Solving this high dimensional PDE turns out to be numerically intractible, so we solve a 4 dimensional HJI PDE instead, and construct an approximation to the 4$\N$ dimensional solution using maximum matching. The approximation provides an upper bound on the number of attackers who are able to reach the target. For clarity in this section, we say that the attacking team wins if $\m$ or more attackers reach the target, and the defending team wins otherwise.

\subsection{Hamilton-Jacobi Reachability} \label{subsec:hj_background}
The multiplayer reach-avoid game is a differential game in which two teams have competing objectives \cite{b:basar-olsder-1999}. The results of HJ computation assume a closed-loop strategy for both players given previous information of the other players. The setup for using HJ reachability to solve differential games can be found in \cite{j:mitchell-TAC-2005, LSToolbox, Huang2011}. In summary, we are given the continuous dynamics of the system state:

\bq
\dxj = f(\xj,u,d), \xj(0)=\xjn,
\eq

\noindent where $\xj\in\R^n$ is the system state, $u\in\mathbb{U}$ is the joint control input of the attacking team, and $d\in\mathbb{D}$ is the joint control input of the defending team. The sets $\mathbb{U}$ and $\mathbb{D}$ represent the sets of the joint admissible control inputs of the attacking team and the defending team, respectively. The attacking team selects a control input based on the past and the current joint states of all the players. The defending team then selects a control input based on the past and the present control inputs of the attacking team, in addition to
the past and the current joint states. \textit{A priori}, this information pattern is conservative towards the attackers, as defenders have more information available. However, in the case that the system (described by the function $f$) is decoupled as in Equation~\eqref{eq:dynamics}, the Isaacs condition \cite{b:isaacs-1967} holds and the two information patterns yield the same optimal solutions for both the attackers and the defenders.

To use the HJ reachability framework, we specify the terminal set $R$ (described in detail in the next subsections) as the attackers' winning condition, and propagate backwards this set subject to the constraint imposing that the attackers be outside the capture regions and the obstacles. This constraint is described by the avoid set $A$. The result is a reach-avoid set that partitions the state space into two regions. All points inside the set represent the joint initial conditions from which the attacking team is guaranteed to win, and all points outside represent the joint initial conditions from which the defending team is guaranteed to win.

More precisely, the HJ reachability calculation is as follows. 
First, given a set $G$, the level set representation of $G$ is a function $\valsR_\set:\R^n \rightarrow \R$ such that $\set = \left\{\xj\in\R^n \mid \valsR_\set\le0\right\}$.
In particular, the terminal set $R$ and the avoid set $A$ will be represented by the functions $\valsR_R$ and $\valsR_A$ respectively.
 
Let $\Phi:\R^n\times[-\T,0]\rightarrow\R$ be the viscosity solution \cite{j:Crandall-TAMS-1983} to the constrained terminal value HJI PDE:
\begin{equation}
	\label{eq:HJ_PDE_reachavoid}
	\frac{\partial \Phi}{\partial t} + \min \left[0, H\left(\xj,\frac{\partial \Phi}{\partial \xj}\right)\right] = 0,\;\Phi(\xj,0) = \valsR_R(\xj)	 
\end{equation}
subject to 
$$\Phi(\xj,t) \geq -\valsR_A(\xj),$$
where the optimal Hamiltonian is given by
$$H\left(\xj,p\right) = \min_{u \in \mathbb{U}} \max_{d \in \mathbb{D}} p^T f(\xj,u,d).$$
By the argument presented in \cite{j:mitchell-TAC-2005} and \cite{mitchell-thesis}, the set of initial conditions from which the attackers are guaranteed to win within time $T$ is given by 
\bq
\mathcal{RA}_T(R,A) := \left\{\xj\in \R^n \mid \Phi (\xj,-\T) \leq 0\right\}.
\eq
Hence, $\Phi (\xj,-\T)$ is the level set representation of $\mathcal{RA}_T(R,A)$.

The optimal control input for the attacking team is given by \cite{j:Lygeros-automatica-1999, j:Tomlin-ProcIEEE-2000, Huang2011}:
\bq \label{eq:opt_ctrl_u}
  u^*(\xj,t) = \arg \min_{u \in \mathbb{U}} \max_{d \in \mathbb{D}} p(\xj,-t)^T f(\xj,u,d), \!\ t \in [0,T] 
\eq
where $p = \frac{\partial \Phi}{\partial \xj}$. 

Similarly, an initial player configuration outside $\mathcal{RA}_T(R,A)$ guarantees that the defenders will win by using the optimal control input
\bq \label{eq:opt_ctrl_d}
  d^*(\xj,t) = \arg \max_{d \in \mathbb{D}} p(\xj,-t)^T f(\xj,u^*,d), \ t \in [0,T].
\eq

Taking $T\rightarrow\infty$, we obtain the set of initial conditions from which the attackers are guaranteed to win. We denote this set $\mathcal{RA}_\infty(R,A)$. The set of initial conditions from which the defenders are guaranteed to win is given by all points not in $\mathcal{RA}_\infty(R,A)$. For an $\N$ vs. $\N$ game on a two-dimensional domain $\amb\subset\R^2$, the reachable set $\mathcal{RA}_\infty(R,A)$ is $4\N$-dimensional.

A highly accurate numerical solution to Equation (\ref{eq:HJ_PDE_reachavoid}) can be computed using the Level Set Toolbox for MATLAB \cite{LSToolbox}.

In the next two subsections, we will describe the terminal set and the avoid set for the multiplayer and the two player reach-avoid games.

\subsection{Hamilton-Jacobi Reachability for the Multiplayer Game}
\label{subsec:hj_multi}
%NEED TO INCLUDE OBSTACLES HERE TOO
In the multiplayer reach-avoid game, the goal of the attacking team is to send at least $\m$ attackers to the target set $\target$. For a particular set of $\m$ attackers, the goal set is given by each of the attackers being inside the target but outside of the capture radius of all the defenders. To obtain the terminal set $R$ for the game, we take the union of all $\begin{pmatrix} \N \\ \m \end{pmatrix}$ goal sets.
%$\begin{pmatrix} \N \\ \m \end{pmatrix}$ ($\N$ choose $\m$) sets. 
%To obtain the terminal set for at least $\m$ attackers reaching the target, one would take the union of the terminal sets for exactly $\m,\m+1,\ldots,\N$ attackers reaching the target without being captured.

The avoid set is defined by the losing condition of the attackers: the attackers lose the game when at least $\N-\m+1$ of them have been captured by the defender. For a particular attacker $\pam{i}$, the individual keep-out set is given by $\bigcup_{j=1}^\N \avoid_{ij}$, corresponding to $\pam{i}$ being captured by any of the defenders $\pbm{j}$. For a particular set of $\N-\m+1$ attackers, the joint keep-out set is characterized by each of these players being within the capture radius of some defender. The avoid set $\set_A$ for at least $\N-\m+1$ attackers being captured is the union of all $\begin{pmatrix} \N \\ \N-\m+1 \end{pmatrix}$ such joint keep-out sets. %The entire avoid set for at least $\N-\m+1$ attackers being captured would be the union of the avoid sets for $\N-\m+1, \N-\m+2, \ldots, \N$ attackers being captured.

The HJ reachability calculation for the multiplayer reach-avoid game is not only cumbersome to set up, but also intractable since the computation complexity scales exponentially with the number of dimensions. In general, an HJI PDE of dimensions higher than five cannot be solved practically. Thus, we are limited to only being able to solve the HJI PDE corresponding to a two player game in which each player's state space is two-dimensional. Instead of directly solving the HJI PDE corresponding to the $2\N$-player game, which is a $4\N$-dimensional problem, we will solve the multiplayer game by combining the solution to the two player game and maximum matching from graph theory. This approach is outlined below.

\subsection{Hamilton-Jacobi Reachability for the Two Player Game}
\label{subsec:hj_two}
In the two-player reach-avoid game, the goal of the attacker is to reach the target set $\target$ while avoiding capture by the defender. This goal is represented by the attacker being inside $\target$ but outside of the defender's capture radius.. En route to $\target$, the attacker must avoid capture by the defender. This is represented by the set $\avoid$. 

In addition, both players need to avoid the obstacles $\obs$, which can be considered as the locations in $\amb$ where the players have zero velocity. In particular, the defender wins if the attacker is in $\obs$, and vice versa. Therefore, we define the terminal set as 
\bq
\begin{aligned}
R = & \left\{\xj\in\amb^2 \mid \xa\in\target \land \|\xa-\xb\|_2 > \Rc \right\} \\  
&\cup \left\{\xj\in\amb^2\mid \xb\in\obs \right\}
\end{aligned}
\eq

Similarly, we define the avoid set as
\bq \begin{aligned}
A &= \left\{\xj\in\amb^2 \mid \|\xa-\xb\|_2\le \Rc \right\} \cup \left\{\xj\in\amb^2\mid \xa\in\obs \right\} \\
&= \avoid \cup \left\{\xj\in\amb^2\mid \xa\in\obs \right\}
\end{aligned} \eq

Given these sets, we can define the corresponding level set representations $\valsR_R,\valsR_A$, and solve (\ref{eq:HJ_PDE_reachavoid}). Assuming $\amb\subset\R^2$, the result is $\mathcal{RA}_\infty(R,A)\in\R^4$, a four dimensional reachable set with the level set representation $\Phi(\xj,-\infty)$. The attacker wins if and only if the joint initial condition is such that $(\xan,\xbn)=\xjn\in\mathcal{RA}_\infty(R,A)$.

If $\xjn\in\mathcal{RA}_\infty(R,A)$, then the attacker is guaranteed to win the game by using the optimal control input given in (\ref{eq:opt_ctrl_u}). Applying Equation (\ref{eq:opt_ctrl_u}) to the two player game, we have that the attacker winning strategy satisfies 

\bq \label{eq:opt_ctrl2_u_gen}
  \ca^*(\xa,\xb,t) = \arg \min_{\ca \in \mathbb{U}} \max_{\cb \in \mathbb{D}} p(\xa,\xb,-t)^T f(\xa,\xb,\ca,\cb) 
\eq

\noindent for $t \in [0,T]$. The explicit winning strategy satisfying (\ref{eq:opt_ctrl2_u_gen}) is given in \cite{Huang2011} as

\bq \label{eq:opt_ctrl2_a}
\ca^*(\xa,\xb,t) = -\vela \frac{p_u(\xa,\xb,-t)}{\|p_u(\xa,\xb,-t)\|_2}.
\eq

\noindent where $p = (p_u,p_d) = \frac{\partial \Phi}{\partial (\xa,\xb)}$.  

Similarly, if $\xjn\notin\mathcal{RA}_\infty(R,A)$, then the defender is guaranteed to win the game by using the optimal control input given in (\ref{eq:opt_ctrl_d}). Applying Equation (\ref{eq:opt_ctrl_d}) to the two player game, we have that the defender winning strategy satisfies 

\bq \label{eq:opt_ctrl2_d_gen}
  \cb^*(\xa,\xb,t) = \arg \max_{\cb \in \mathbb{D}} p(\xa,\xb,-t)^T f(\xa,\xb,\ca^*,\cb)
\eq

\noindent for $t \in [0,T]$. The explicit winning strategy satisfying (\ref{eq:opt_ctrl2_d_gen}) is given in \cite{Huang2011} as
\bq \label{eq:opt_ctrl2_d}
\cb^*(\xa,\xb,t) = \velb \frac{p_d(\xa,\xb,-t)}{\|p_d(\xa,\xb,-t)\|_2}.
\eq

\subsection{Maximum Matching}
\label{subsec:max_match}
We can determine whether the attacker can win the multiplayer reach-avoid game by combining the solution to the two player game, characterized by $\mathcal{RA}_\infty(R,A)$, and maximum matching \cite{Schrjiver2004, Karpinski1998} from graph theory as follows:

\begin{enumerate}
\item Compute $\mathcal{RA}_\infty(R,A)$
\item Construct a bipartite graph with two sets of nodes $\pas,\pbs$, where each node represents a player.
\item For each $\pbm{i}$, determine whether $\pbm{i}$ can win against $\pam{j}$, for all $j$. Given $\mathcal{RA}_\infty(R,A)$, we can determine the winner of the two player game for any given pair $(\xam{i},\xbm{j}) \forall (i,j)$.
\item Form a bipartite graph: Draw an edge between $\pbm{i}$ and $\pam{j}$ if $\pbm{i}$ wins against $\pam{j}$
\item Run any matching algorithm to find a maximum matching in the graph. This can be done using, for example, a linear program \cite{Schrjiver2004}, or the Hopcroft-Karp algorithm \cite{Karpinski1998}.
\end{enumerate}

After finding a maximum matching, we can determine whether the defending team can win as follows. After constructing the bipartite graph, if the maximum matching is of size $\mm$, then the defending team would be able to prevent \textit{at least} $\mm$ attackers from reaching the target. Alternatively, $\N-\mm$ is an upper bound on the number of attackers that can reach the target.

For intuition, consider the following specific cases of $\mm$. If the size of the maximum matching $\mm=\N$, then no attacker will be able to reach the target. If $\mm=0$, then there is no initial pairing that will prevent any attacker from reaching the target; however, the attackers are not guaranteed to all reach the target, as $\N-\mm=\N$ is only an upper bound on the number of attackers who can reach the target. Finally, $\mm=\N-\m+1$, then the attacking team would only be able to send at most $\N-\mm=\m-1$ attackers to the target. 

The optimal strategy for the defenders can be obtained from (\ref{eq:opt_ctrl2_d}). If the \ith defender $\pbm{i}$ is assigned to defend against the \jth attacker $\pam{j}$ by the maximum matching, then the strategy that guarantees that $\pam{j}$ never reaches the target satisfies (\ref{eq:opt_ctrl2_d_gen}): 

\bq \label{eq:opt_ctrl3_d_gen}
  \cbm{i}^*(\xam{j},\xbm{i},t) = \arg \max_{\cbm{i} \in \mathbb{D}} p(\xam{j},\xbm{i},-t)^T f(\xam{j},\xbm{i},\cam{j}^*,\cbm{i})
\eq

\noindent for $t\in [0,T]$, where $\cam{j}^*$ is given in (\ref{eq:opt_ctrl2_a}) as

\bq \label{eq:opt_ctrl2_a}
\cam{j}^*(\xam{j},\xbm{i},t) = -\vela \frac{p_u(\xam{j},\xbm{i},-t)}{\|p_u(\xam{j},\xbm{i},-t)\|_2}.
\eq

\noindent where $p = (p_u, p_d) = \frac{\partial \Phi}{\partial (\xam{j},\xbm{i})}$. The explicit strategy is then similar to (\ref{eq:opt_ctrl2_d}):
\bq \label{eq:opt_ctrl3_d}
\cb^*(\xam{j},\xbm{i},t) = \velb \frac{p_d(\xam{j},\xbm{i},-t)}{\|p_d(\xam{j},\xbm{i},-t)\|_2}
\eq

The entire procedure of applying maximum matching to the 4D HJ reachability calculation is illustrated in Figure \ref{fig:general_procedure}.

Our solution to the multiplayer reach-avoid game is an approximation to the optimal solution that would be obtained by directly solving the $4\N$ dimensional HJI PDE obtained in Section \ref{subsec:hj_multi}; it is conservative for the defending team because by creating defender-attacker pairs, each defender restricts its attention to only one opposing player. For example, if no suitable matching is found, the defending team is not guaranteed to allow all attackers to reach the target, as the defending team could potentially capture some attackers without using a strategy that creates defender-attacker pairs. Nevertheless, our solution is able to overcome the numerical intractibility to approximate a reachability calculation, and is useful in many game configurations.

\begin{figure}[h]
\centering
\includegraphics[width=0.5\textwidth]{"fig/general procedure"}
\caption{An illustration of using 4D HJ reachability and maximum matching to solve the multiplayer reach-avoid game. A bipartite graph is created based on results of the 4D HJ reachability calculation. Then, a maximum matching of the bipartite graph is found to optimally assign defender-attacker pairs. A maximum matching of size $\mm$ indicates that at most $\N-\mm$ attackers will be able to rach the target.}
\label{fig:general_procedure}
\end{figure}

\subsection{Time-Varying Defender-Attacker Pairings}
\label{subsec:tvarp}
The procedure outlined in Section \ref{subsec:max_match} assigns an attacker to each defender that is part of a maximum pairing in an open-loop manner: the assignment is done in the beginning of the game, and does not change during the course of the game. However, the bipartite graph and its corresponding maximum matching can be updated as the players change positions during the game. Because $\mathcal{RA}_\infty(R,A)$ captures the winning conditions for every joint defender-attacker configuration given $\amb,\obs,\target$, this update can be performed in real time by the following procedure:

\begin{enumerate}
\item Given the position of each defender $\xbm{i}$ and each attacker $\xam{j}$, determine whether $\xam{j}$ can win for all $j$. 
\item Construct the bipartite graph and find its maximum matching to assign an attacker to each defender that is part of the maximum matching.
\item For a short, chosen duration $\Delta$, compute the optimal control input and trajectory for each defender that is part of the maximum matching via Equation (\ref{eq:opt_ctrl2_d}). For the rest of the defenders and for all attackers, compute the trajectories assuming some control function.
\item Repeat the procedure with the new player positions.
\end{enumerate}

As $\Delta\rightarrow 0$, the above procedure computes a bipartite graph and its maximum matching as a function of time. Whenever the maximum matching is not unique, the defenders can choose a different maximum matching and still be guaranteed to prevent the same number of attackers from reaching the target. As long as each defender uses the optimal control input given in Equation (\ref{eq:opt_ctrl2_d}), the size of the maximum matching can never decrease as a function of time. 

On the other hand, it is possible for the size of the maximum matching to increase as a function of time. This occurs if the joint configuration of the players becomes such that the resulting bipartite graph has a bigger maximum matching than before, which may happen since the size of the maximum matching only gives an upper bound on the number of attackers that are able to reach the target. Furthermore, there is no numerically tractable way to compute the joint optimal control input for the attacking team, so a suboptimal strategy from the attacking team can be expected, making an increase of maximum matching size likely. Determining defender control strategies that optimally promote an increase in the size of the maximum matching would be an important step towards the investigation of cooperation, and will be part of our future work.