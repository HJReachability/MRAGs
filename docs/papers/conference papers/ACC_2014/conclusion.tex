\section{Conclusions and Future Work} \label{sec:conc}
By applying maximum matching to the solution to the two player reach-avoid game, we were able to approximate the solution to the multiplayer reach-avoid game without significant additional computation overhead over the two player computation. By solving a single 4D HJI PDE, we obtained all possible pairings between the defenders and attackers. Then, a maximum matching algorithm determines the pairing that prevents the maximum number of attackers from reaching the target. This way, we were able to analyze the multiplayer reach-avoid game without directly solving the corresponding high dimensional, numerically intractable HJI PDE. Calculating time-varying defender-attacker pairings allows the defending team to potentially increase the size of the maximum matching over time.

An immediate extension of this work is to investigate defender strategies that optimally promote an increase in the size of maximum matching. This would be a step towards real-time defending team cooperation, in which each defender is not only responsible for preventing a particular assigned attacker from reaching the target, but also enabling other defenders to defend previously seemingly undefendable attackers. Many other extensions of this work naturally arise, including analyzing reach-avoid games with unfair teams with different numbers of players on each team, more complex player dynamics, or partial observability. 