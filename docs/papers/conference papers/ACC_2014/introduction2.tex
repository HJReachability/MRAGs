\section{Introduction}
\label{sec:intro}
Differential games are powerful theoretical tools in robotics, aircraft control, security, and other domains \cite{OFTHEAIRFORCEWASHINGTON:2009p37, Erzberger:2006p44, kiva2009}. The multiplayer reach-avoid game (to be defined precisely in Section \ref{sec:formulation}) is a differential game between two adversarial teams of cooperating players, where one team attempts to reach a certain target quickly while the other team aims to delay, or if possible, prevent the opposing team from reaching the target. One example of a reach-avoid game is the popular game Capture-the-Flag (CTF) \cite{HThesis, Huang2011}. In robotics and automation, CTF has been explored most notably in the Cornell Roboflag competition, where two opposing teams of mobile robots are directed by human players to play the game \cite{DAndrea:2003p95}. A number of results related to motion planning and human-robot interactions have been reported from the competition \cite{Earl:2007p101, Campbell:2003p5, Waydo:2003p97, Parasuraman:2005p99}. 

A multiplayer reach-avoid game is a complex game due to both the conflicting goals of the two teams and the cooperation among the players within each team, rendering the optimal solution for each team nontrivial to obtain and visualize. Previous work \cite{HThesis, Huang2011} has shown that even in a 1 vs. 1 scenario, human agents are sometimes unable to find the optimal way to play, losing in situations in which an optimal winning strategy exists. For general multiplayer reach-avoid games, optimal solutions are extremely difficult to compute due to the intrinsic high dimensionality of the joint state space. Multiplayer differential games have been previously addressed using various techniques. In \cite{Earl:2007p101}, where a team of defenders assumes that the attackers move towards their target in straight lines, a mixed-integer linear programming approach was used. In \cite{Chasparis:2005p102}, optimal defender strategies are determined using a linear program, with the assumption that the attackers use a linear feedback control law. In complex pursuit-evasion games where players may change roles over time, nonlinear model-predictive control \cite{Sprinkle:2004p100} and approximate dynamic programming \cite{McGrew:2008p103} approaches have been investigated. In both cases, opponent strategies are estimated based on explicit prediction models.

The above-mentioned techniques tend to work well in the situations in which accurate models of the opponent team can be obtained. While those techniques have proven to be effective in their corresponding scenarios, they cannot be easily adapted to solve a general multiplayer reach-avoid game when no prior information on each side is known. The ideal framework to use for such a general multiplayer reach-avoid game is the Hamilton-Jacobi-Isaacs (HJI) approach \cite{b:isaacs-1967}, in which an HJI partial differential equation (PDE) is solved to obtain optimal strategies for both teams. In this case, if both teams play optimally, the result of the game is determined by the joint initial condition of the players. In addition, many numerical tools \cite{j:mitchell-TAC-2005, Sethian1996, b:osher-fedkiw-2002} have been developed to carry out the computations and leverage the power of the HJI framework when the dimensionality of the problem is low. These  tools have been employed to successfully solve a variety of differential games, path planning problems, and optimal control problems, including aircraft collision avoidance \cite{j:mitchell-TAC-2005}, automated in-flight refueling \cite{DSST08}, and two-player reach-avoid games \cite{Huang2011}. These tools offer tremendous flexibility in terms of the player dynamics and terrain, and do not explicitly assume any specific control strategy or prediction models for the players. 

Despite the power that the HJI framework and the numerical tools have to offer, solving a general multiplayer reach-avoid game is computationally intractable due to the curse of dimensionality: the joint state space of two $\N$-player teams in a two-dimensional (2D) domain is $4N$-dimensional ($4N$D). Numerically, when the state space is discretized, the number of nodes scales exponentially with the number of dimensions. Therefore, computing optimal solutions for a general multiplayer reach-avoid game in the HJI framework is out of reach. As a result, the inherent trade-off between optimality of the solutions and computational complexity must be considered and made. 

In the multiplayer reach-avoid game, we consider each defender-attacker pair and compute the optimal solutions for both players using the HJI framework. We then invoke the graph-theoretical maximum matching algorithm  \cite{Schrjiver2004, Karpinski1998} to determine optimal pairings. This procedure approximates the solution to the multiplayer game by combining the solutions from the $\N^2$ two-player games between each attacker-defender pair. This way, the computation complexity is reduced from solving a $4N$D HJI PDE to solving $\N^2$ 4D HJI PDEs. If we also assume that players on each team have the same dynamics, then only \textit{one} 4D HJI PDE needs to be solved. This is because an HJI calculation partitions the joint state space of the two players into a winning region for the attacker and a winning region for the defender. 

Our contributions can hence mainly be stated as follows. First, it is an extension of the work in \cite{Huang2011} on two-player reach-avoid games to multiplayer reach-avoid games. Our extension is easily implemented at almost no additional computational cost compared to the two-player games. Hence, our approach provides an appealing solution, especially when the number of players becomes large. In addition, some cooperation is taken into account by the maximum matching process. Second, as there are other ways to trade off optimality of the solutions and computational complexity, our extension provides an initial, in fact, the only baseline benchmark in the HJI framework to which other approaches can be compared. This is again due to the numerical intractibility of HJI computations, which are practically limited to at most five-dimensional systems. Hence, any sub-game one hopes to solve using the HJI framework can involve at most two players. Such a benchmark can also be viewed as an invitation for other approaches that do not require solving HJI PDEs to address the complexity-optimality trade-off. 

%we expand on the previous work on solving the two player reach-avoid game between an attacker and a defender via Hamilton Jacobi (HJ) reachability \cite{Huang2011}. Given the terrain and player dynamics, the HJ reachability calculation partitions the joint state space of the two players into a winning region for the attacker and a winning region for the defender. Because the calculation is done in the joint state space, HJ reachability determines the winner of the game under optimal strategies for any initial configuration of the two players. From this information, we can determine the winner of every defender-attacker pair in a multiplayer game, and then apply a maximum matching algorithm \cite{Schrjiver2004, Karpinski1998} to determine optimal pairings. This procedure approximates the HJ reachability solution to the multiplayer game by combining the solutions to two player games between each pair of players.

%In section \ref{sec:formulation}, we will mathematically formulate the multiplayer reach-avoid game. In section \ref{sec:solution}, we will first present how the full game can in principle be solved using Hamilton Jacobi (HJ) reachability, and then illustrate how maximum matching can overcome the game's high dimensionality  that makes it infeasible to directly solve. In addition, we will suggest how performing the maximum matching calculation in real time can sometimes allow defenders to win from an initially losing position. In section \ref{sec:results}, we will present numerical simulation results to demonstrate our algorithm. Finally, we conclude and suggest a mulititude of potential future research directions in section \ref{sec:conc}.