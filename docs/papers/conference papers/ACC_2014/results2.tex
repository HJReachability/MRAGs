\section{Computation Results}
\label{sec:results}
We illustrate our reachability and maximum matching approach in the example below. The HJ reach-avoid sets are calculated using the Level Set Toolbox \cite{LSToolbox} developed at the University of British Columbia. We calculated $\mathcal{RA}_\infty(R,A)$ by incrementing $T$ until $\mathcal{RA}_T(R,A)$ converges. The computation of $\mathcal{RA}_\infty(R,A)$, done on a $45\times45\times45\times45$ grid, took approximately 30 minutes on a Lenovo T420s laptop with a Core i7-2640M processor. 

The example is shown in Figures \ref{fig:results1}. There are four attackers and four defenders playing on a square domain with obstacles; the defenders have a capture radius of $0.1$ units. All players have equal speeds ($\velb=\vela$). 

$\mathcal{RA}_\infty(R,A)$ is a 4D set that represents the joint configurations in which the attacker wins the game. To visualize the 4D set in 2D, we view the reach-avoid set at the slices representing the positions of particular players. Figure \ref{subfig:fixed_d_1} shows boundaries of $\mathcal{RA}_\infty(R,A)$ with fixed defender positions. In each subplot, attackers which are closer to the target set than the reach-avoid set boundary win against the particular defender. For example, in the right top subplot of Figure $\ref{subfig:fixed_d_1}$, the defender at $(0.3,-0.5)$ loses to the attacker at $(0,0)$, but wins against the other three attackers. 

Similarly, Figures \ref{subfig:fixed_a_1} shows boundaries of $\mathcal{RA}_\infty(R,A)$ with fixed attacker positions. Defenders which are closer to the target set than the reach-avoid set boundary win against the particular attacker. For example, in the bottom left subplot, the attacker at $(0, 0)$ wins against the defender at $(0.3, 0.5)$ but loses against the other three defenders.

Figure \ref{fig:max_matching_1} shows the resulting bipartite graph (thin solid blue lines) and the maximum matching (thick dashed blue lines) after applying the algorithm described in Section \ref{subsec:max_match}. The maximum matching is of size 3, which means that at most 1 attacker will be able to reach the target. 

Figure \ref{fig:tvarg} shows the result of a 4 vs. 4 reach-avoid game simulation being played out in a course of 0.6 time units to illustrate the potential usefulness of time-dependent defender-attacker pairings. Every $\Delta=0.005$ time units, a bipartite graph and its maximum matching are computed according to the algorithm in Section \ref{subsec:tvarp}. The defender that is not part of the maximum matching plays optimally according to (\ref{eq:opt_ctrl3_d}) against the closest attacker. The attackers use the suboptimal strategy of taking the shortest path to the target while steering $0.125$ units clear of the obstacles and disregarding the control inputs of other players in the game.

The initial size of the maximum matching is 3. At $t=0.4$, because the attackers have not been playing optimally, the attacker at $(-0.50,0.16)$ becomes in a losing position against the defender at $(-0.26, -0.34)$. Thus, the maximum matching now assigns the attacker at $(-0.50,0.16)$ to the defender at $(-0.26, -0.34)$. At the same time, the defender at $(-0.09, -0.22)$ switches from defending the attacker at $(-0.50,0.16)$ to defending the attacker at $(-0.27,0.14)$, creating a perfect matching and preventing any attacker from reaching the target.
\begin{figure}[h]
\centering
	\begin{subfigure}{0.5\textwidth}
	\centering
	\includegraphics[width=\textwidth]{"fig/fixed defender 1"}
	\caption{Slices of reach-avoid set at the four defender positions. The defender wins against any attacker who is farther away from the target than the reachable set boundary is.}
	\label{subfig:fixed_d_1}
	\end{subfigure}	
	
	\begin{subfigure}{0.5\textwidth}
	\centering
	\includegraphics[width=\textwidth]{"fig/fixed attacker 1"}
	\caption{Slices of reach-avoid set at the the four attacker positions. Each attacker is wins against any defender who is farther away from the target than the reachable set boundary is.}
	\label{subfig:fixed_a_1}
	\end{subfigure}
\caption{A 4 vs. 4 reach-avoid game in which all players have equal maximum speeds.}
\label{fig:results1}
\end{figure}

\begin{figure}[h]
	%\ContinuedFloat
	\centering
	%\begin{subfigure}{0.4\textwidth}
%	\centering
	\includegraphics[width=0.5\textwidth]{"fig/max matching 1"}
	\caption{Bipartite graph and maximum matching results. Each edge (solid blue line) connects a defender to an attacker against whom the defender is guaranteed to win, creating a bipartite graph. A maximum matching (dashed blue line) of size 3 indicates at most one attacker can reach the target.}
	\label{fig:max_matching_1}
	%\end{subfigure}
	
\end{figure}

\begin{figure}[h]
\centering
\includegraphics[width=0.5\textwidth]{"fig/time varying graph"}
\caption{Real-time maximum matching updates. The defenders can update the bipartite graph and maximum matching via the procedure described in Section \ref{subsec:tvarp} in real time. Because the attacking team is not playing optimally, the defending team finds a perfect matching after $t=0.4$.}
\label{fig:tvarg}
\end{figure}